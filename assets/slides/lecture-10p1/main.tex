\documentclass{beamer}

\usepackage{graphicx}
\usepackage{textpos}
\usepackage{listings}
\usepackage{lstautogobble}

\usetheme{Madrid}
\useoutertheme{miniframes} % Alternatively: miniframes, infolines, split

% Setup the university's color pallette
\definecolor{UIUCorange}{RGB}{19, 41, 75} % UBC Blue (primary)
\definecolor{UIUCblue}{RGB}{232, 74, 39} % UBC Grey (secondary)

\definecolor{codegreen}{rgb}{0,0.6,0}
\definecolor{codegray}{rgb}{0.5,0.5,0.5}
\definecolor{codepurple}{rgb}{0.58,0,0.82}
\definecolor{backcolour}{rgb}{0.95,0.95,0.92}

\lstdefinestyle{python}{
  backgroundcolor=\color{backcolour},   
  commentstyle=\color{codegreen},
  keywordstyle=\color{magenta},
  numberstyle=\tiny\color{codegray},
  stringstyle=\color{codepurple},
  basicstyle=\ttfamily\footnotesize,
  breakatwhitespace=false,         
  belowskip=-0.5em,
  breaklines=true,                 
  captionpos=b,                    
  keepspaces=true,                 
  numbers=left,                    
  numbersep=5pt,                  
  showspaces=false,                
  showstringspaces=false,
  showtabs=false,                  
  tabsize=2
}

\lstset{style=python}

\AtBeginSection[]{
  \begin{frame}
    \vfill
    \centering
    \begin{beamercolorbox}[sep=8pt,center,shadow=true,rounded=true]{title}
      \usebeamerfont{title}\insertsectionhead\par%
    \end{beamercolorbox}
    \vfill
  \end{frame}
}
% Setup the university's color pallette
\definecolor{UIUCorange}{RGB}{19, 41, 75} % UBC Blue (primary)
\definecolor{UIUCblue}{RGB}{232, 74, 39} % UBC Grey (secondary)


\setbeamercolor{palette primary}{bg=UIUCorange,fg=white}
\setbeamercolor{palette secondary}{bg=UIUCblue,fg=white}
\setbeamercolor{palette tertiary}{bg=UIUCblue,fg=white}
\setbeamercolor{palette quaternary}{bg=UIUCblue,fg=white}
\setbeamercolor{structure}{fg=UIUCorange} % itemize, enumerate, etc
\setbeamercolor{section in toc}{fg=UIUCblue} % TOC sections

\setbeamercolor{subsection in head/foot}{bg=UIUCorange,fg=UIUCblue}
\setbeamercolor{subsection in head/foot}{bg=UIUCorange,fg=UIUCblue}

\usepackage[utf8]{inputenc}
\usepackage{graphicx}

%Information to be included in the title page:
\title{\textbf{Lists}}
\author{\textbf{David H Smith IV}}
\institute[\textbf{UIUC}]{\textbf{University of Illinois Urbana-Champaign}}
\date{\textbf{Tues, Thu 21 2021}}

\setbeamertemplate{title page}[default][colsep=-4bp,rounded=true]
\addtobeamertemplate{title page}{\vspace{3\baselineskip}}{}
\addtobeamertemplate{title page}{
  \begin{textblock*}{\paperwidth}(-1.0em, -1.2em)
    \includegraphics[width=\paperwidth, height=\paperheight]{imgs/uiuc.png}
  \end{textblock*} 
}{}

\begin{document}

\frame{\titlepage}

\section{Reminders}

%
% Slide 1
%
\begin{frame}
  \frametitle{Reminders}
  \begin{itemize}
    \item 
  \end{itemize}
\end{frame}

\section{Lists: General Review}

%
% Slide 2
%
\begin{frame}[fragile]
  \frametitle{Poll Question: List Function}
  What is the value of x?
  \begin{lstlisting}[language=Python, autogobble]
  x = list('abc')
  \end{lstlisting}
  \vfill
  \begin{enumerate}[A]
    \item \lstinline|'abc'|
    \item \lstinline|['abc']|
    \item \lstinline|['a', 'b','c']|
    \item \lstinline|[]|
  \end{enumerate}
\end{frame}

%
% Slide
%
\begin{frame}[fragile]
  \frametitle{List Methods}
  Use \lstinline|help(list.<method name>)| for information on a given method:
  \begin{itemize}
    \item \lstinline|L.append(elem)| \textrightarrow \ Add element to the end of L.
    \item \lstinline|L.extend(lst)| \textrightarrow \ Add all elements of lst to the end of L.
    \item \lstinline|L.insert(index, elem)| \textrightarrow \ Insert element at index of L pushing other elements forward.
    \item \lstinline|L.pop()| \textrightarrow \ Remove and return the element at the end of L.
    \item \lstinline|L.pop(index)| \textrightarrow \ Remove and return the element at index of L.
    \item \lstinline|L.remove(elem)| \textrightarrow \ Remove first occurrence of element from L.
    \item \lstinline|L.sort(elem)| \textrightarrow \ Sort the elements of L.
  \end{itemize}
\end{frame}

%
% Slide 2
%
\begin{frame}[fragile]
  \frametitle{Poll Question: List Functions}
  What is the value of \lstinline|a| after this code is run?
  \begin{lstlisting}[language=Python, autogobble]
  a = [2, 4, 6, 8]
  a.remove(4)
  a.pop(2)
  \end{lstlisting}
  \vfill
  \begin{enumerate}[A]
    \item \lstinline|[2, 4]|
    \item \lstinline|[6, 8]|
    \item \lstinline|[2, 6]|
    \item \lstinline|[2, 8]|
  \end{enumerate}
\end{frame}

%
% Slide 2
%
\begin{frame}[fragile]
  \frametitle{Poll Question: List Slicing/Indexing}
  What is the value of A after this code executes?
  \begin{lstlisting}[language=Python, autogobble]
  A = [1, 2, 3]
  B = A
  C = A[:]
  B[1] = "pirate"
  C[2] = "scurvy"
  \end{lstlisting}
  \vfill
  \begin{enumerate}[A]
    \item \lstinline|[1, 2, 3]|
    \item \lstinline|[1, "pirate", 3]|
    \item \lstinline|[1, "pirate", "scurvy"]|
    \item \lstinline|["pirate", "scurvy"]|
  \end{enumerate}
\end{frame}

%
% Slide 2
%
\begin{frame}[fragile]
  \frametitle{Poll Question: More List Functions}
  Which will cause \lstinline|x = [1, 2, 3, 4]|
  \begin{lstlisting}[language=Python, autogobble]
  x = [1, 2]
  \end{lstlisting}
  \vfill
  \begin{enumerate}[A]
    \item \lstinline|x.append([3, 4])|
    \item \lstinline|x += [3, 4]|
    \item \lstinline|x.extend([3, 4])|
    \item \lstinline|A and B|
    \item \lstinline|A and C|
    \item \lstinline|B and C|
  \end{enumerate}
\end{frame}

%
% Slide 2
%
\begin{frame}[fragile]
  \frametitle{Poll Question: Loops}
  What will this output?\\
  \begin{minipage}{0.48\textwidth}
    \begin{lstlisting}[language=Python, autogobble]
    A = [5, 10, 15]
    for i in range(len(A)):
      A[i] = A[i] + 1
    print("1: ", A, end=" ")

    A = [5, 10, 15]
    for e in A:
      e = e + 1
    print("2: ", A)
    \end{lstlisting}
  \end{minipage}
  \hfill
  \begin{minipage}{0.48\textwidth}
    \begin{enumerate}[A]
      \item \lstinline|1: [5,10,15] 2: [5,10,15]|
      \item \lstinline|1: [6,11,16] 2: [6,11,16]|
      \item \lstinline|1: [5,10,15] 2: [6,11,16]|
      \item \lstinline|1: [6,11,16] 2: [5,10,15]|
    \end{enumerate}
  \end{minipage}
\end{frame}

%
% Slide 2
%
\begin{frame}[fragile]
  \frametitle{Poll Question: Removal Functions}
  We want a function that removes all spaces form strings. Which is correct?\\
  \hfill
  \begin{minipage}{0.48\textwidth}
    \begin{lstlisting}[language=Python, autogobble, basicstyle=\tiny]
  def A(s):
    for c in s:
      if c == "":
        c = ""
    return s

  def B(s):
    new_s = ""
    for c in s:
      if c != " ":
        new_s = new_s + c
    return new_s

  def C(s):
    for i in range(len(s)):
      if s[i] == " ":
        s[i] = ""
    return s\end{lstlisting}
  \end{minipage}
  \hfill
  \begin{minipage}{0.48\textwidth}
    \hfill
    \begin{enumerate}[A]
      \item A
      \item B
      \item C
      \item A and C
      \item B and C
    \end{enumerate}
  \end{minipage}
\end{frame}

\section{List Creation vs Modification}

%
% Slide 2
%
\begin{frame}[fragile]
  \frametitle{Creating vs Modifying Lists}
  Imagine lists \lstinline|x| and \lstinline|y|.\\
  \vfill
  \begin{minipage}{0.49\textwidth}
    \textbf{Creates new list:}
    \begin{enumerate}[A]
      \item \lstinline|z = x.copy()|
      \item \lstinline|z = x[:]|
      \item \lstinline|z = x + y|
      \item \lstinline|z = sorted(x)|
      \item \lstinline|z = reversed(x)|
    \end{enumerate}
  \end{minipage}
  \begin{minipage}{0.49\textwidth}
    \textbf{Modifies a list:}
    \begin{enumerate}[A]
      \item \lstinline|x.sort()|
      \item \lstinline|x.append(num)|
      \item \lstinline|x.remove(num)|
      \item \lstinline|x.extend([num1, num2, ...])|
      \item \lstinline|x.pop(index)|
      \item \lstinline|x.insert(num)|
    \end{enumerate}
  \end{minipage}
\end{frame}



\section{Nested Lists}

\begin{frame}[fragile]
  \frametitle{Poll Question: Nested Lists}
  Which of the following is used to access `e'?
  \begin{lstlisting}[language=Python, autogobble]
  my_list = [
      ['a', 'b', 'c'],
      ['d', 'e', 'f'],
      ['g', 'h', 'i']
    ]
  x = my_list[1][2]
  \end{lstlisting}
  \vfill
  \begin{enumerate}[A]
    \item \lstinline|my_list[2][2]|
    \item \lstinline|my_list[1][1]|
    \item \lstinline|my_list[1][2]|
    \item None of the above
  \end{enumerate}
\end{frame}


%
% Slide 2
%
\begin{frame}[fragile]
  \frametitle{Poll Question: Nested Lists}
  What is the result of x?
  \begin{lstlisting}[language=Python, autogobble]
  my_list = [
      ['a', 'b', 'c'],
      ['d', 'e', 'f'],
      ['g', 'h', 'i']
    ]
  x = my_list[1][2]
  \end{lstlisting}
  \vfill
  \begin{enumerate}[A]
    \item \lstinline|'b'|
    \item \lstinline|'d'|
    \item \lstinline|'f'|
    \item \lstinline|'h'|
  \end{enumerate}
\end{frame}

%
% Slide 2
%
\begin{frame}[fragile]
  \frametitle{Poll Question: Nested Lists}
  What is the resulting value of z after this code runs?
  \begin{lstlisting}[language=Python, autogobble]
  my_list = [
      ['a', 'b', 'c'],
      ['d', 'e', 'f'],
      ['g', 'h', 'i']
    ]
  z = ""
  for y in my_list:
    for x in y:
      total += x
  \end{lstlisting}
  \vfill
  \begin{enumerate}[A]
    \item \lstinline|`abcdefghi'|
    \item \lstinline|[`abc',`def',`ghi']|
    \item \lstinline|[`[abc]',`[def]',`[ghi]']|
    \item \lstinline|`[abcdefghi]'|
  \end{enumerate}
\end{frame}


\section{numpy}
%
% Slide 2
%
\begin{frame}[fragile]
  \frametitle{numpy.array vs lists}
  \begin{enumerate}
    \item Shape in numpy:
      \begin{itemize}
        \item Imagine we have a numpy array \lstinline|a|.
        \item \lstinline|a.shape[0]| height of the array (y dimesion).
        \item \lstinline|a.shape[1]| width of the array (x dimension).
      \end{itemize}
      \pause
    \item To index into a 2d array: \lstinline|regular_python_list[y][x]| \textrightarrow \ \lstinline|np_array[y, x]|
      \pause
    \item To get a chunk of an numpy array: \lstinline|np_array[y_start: y_end, x_start: x_end]|
      \pause
    \item \lstinline|numpy.zeros((y, x))| \textrightarrow Produces a 2d numpy array of x by y dimensions
      \pause 
    \item Other than that, they behave very similar to regular python lists but come with a ton of helpful functions (See \lstinline|help(numpy.array))!
  \end{enumerate}
\end{frame}

\end{document}
