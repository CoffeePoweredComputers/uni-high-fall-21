\documentclass{beamer}

\usepackage{graphicx}
\usepackage{textpos}
\usepackage{listings}
\usepackage{lstautogobble}

\usetheme{Madrid}
\useoutertheme{miniframes} % Alternatively: miniframes, infolines, split

% Setup the university's color pallette
\definecolor{UIUCorange}{RGB}{19, 41, 75} % UBC Blue (primary)
\definecolor{UIUCblue}{RGB}{232, 74, 39} % UBC Grey (secondary)

\definecolor{codegreen}{rgb}{0,0.6,0}
\definecolor{codegray}{rgb}{0.5,0.5,0.5}
\definecolor{codepurple}{rgb}{0.58,0,0.82}
\definecolor{backcolour}{rgb}{0.95,0.95,0.92}

\lstdefinestyle{python}{
    backgroundcolor=\color{backcolour},   
    commentstyle=\color{codegreen},
    keywordstyle=\color{magenta},
    numberstyle=\tiny\color{codegray},
    stringstyle=\color{codepurple},
    basicstyle=\ttfamily\footnotesize,
    breakatwhitespace=false,         
    belowskip=-0.5em,
    breaklines=true,                 
    captionpos=b,                    
    keepspaces=true,                 
    %numbers=left,                    
    numbersep=5pt,                  
    showspaces=false,                
    showstringspaces=false,
    showtabs=false,                  
    tabsize=2
}

\lstset{style=python}

\AtBeginSection[]{
    \begin{frame}
        \vfill
        \centering
        \begin{beamercolorbox}[sep=8pt,center,shadow=true,rounded=true]{title}
            \usebeamerfont{title}\insertsectionhead\par%
        \end{beamercolorbox}
        \vfill
    \end{frame}
}
% Setup the university's color pallette
\definecolor{UIUCorange}{RGB}{19, 41, 75} % UBC Blue (primary)
\definecolor{UIUCblue}{RGB}{232, 74, 39} % UBC Grey (secondary)


\setbeamercolor{palette primary}{bg=UIUCorange,fg=white}
\setbeamercolor{palette secondary}{bg=UIUCblue,fg=white}
\setbeamercolor{palette tertiary}{bg=UIUCblue,fg=white}
\setbeamercolor{palette quaternary}{bg=UIUCblue,fg=white}
\setbeamercolor{structure}{fg=UIUCorange} % itemize, enumerate, etc
\setbeamercolor{section in toc}{fg=UIUCblue} % TOC sections

\setbeamercolor{subsection in head/foot}{bg=UIUCorange,fg=UIUCblue}
\setbeamercolor{subsection in head/foot}{bg=UIUCorange,fg=UIUCblue}

\usepackage[utf8]{inputenc}
\usepackage{graphicx}

%Information to be included in the title page:
\title{\textbf{Try-Except and Testing}}
\author{\textbf{David H Smith IV}}
\institute[\textbf{UIUC}]{\textbf{University of Illinois Urbana-Champaign}}
\date{\textbf{Tues, Dec 07 2021}}

\setbeamertemplate{title page}[default][colsep=-4bp,rounded=true]
\addtobeamertemplate{title page}{\vspace{3\baselineskip}}{}
\addtobeamertemplate{title page}{
    \begin{textblock*}{\paperwidth}(-1.0em, -1.2em)
        \includegraphics[width=\paperwidth, height=\paperheight]{imgs/uiuc.png}
    \end{textblock*} 
}{}

\begin{document}

\frame{\titlepage}

\section{Reminders}

%
% Slide 1
%
\begin{frame}
    \frametitle{Reminders}
    \begin{itemize}
        \item Quiz Thursday
        \item Attempt Practice Quiz before Thursday 10am
        \item Topic 15 Participation due Wednesday
    \end{itemize}
\end{frame}

\section{Try-Except}

%
% Slide 2
%
\begin{frame}[fragile]
    \frametitle{Making a Class: Poll Question}
    What is the result of the following code?
    \begin{lstlisting}[language=Python, autogobble]
    strings = ["This", "Is", "A", "String", "For", "Testing"]
    for string in strings:
        print(string[3], end=" ")
    \end{lstlisting}
    \vfill
    \begin{enumerate}[A]
        \item IndexError
        \item s i t
        \item This String Testing
        \item This Stri Test
    \end{enumerate}
\end{frame}


%
% Slide 2
%
\begin{frame}[fragile]
    \frametitle{Making a Class: Poll Question}
    What is the result of the following code?
    \begin{lstlisting}[language=Python, autogobble]
    strings = ["This", "Is", "A", "String", "For", "Testing"]
    for string in strings:
        try:
            print(string[3], end=" ")
        except IndexError:
            continue
    \end{lstlisting}
    \vfill
    \begin{enumerate}[A]
        \item IndexError
        \item s i t
        \item This String Testing
        \item This Stri Test
    \end{enumerate}
\end{frame}


%
% Slide 2
%
\begin{frame}[fragile]
    \frametitle{Types of Exceptions}
    \vfill
    \begin{enumerate}[A]
        \item \textbf{EOFError: } input() hits an end-of-file condition (EOF) without reading any input.
        \item \textbf{KeyError: } A dictionary key is not found in the set of keys.
        \item \textbf{ZeroDivisionError: } Divide by zero error.
        \item \textbf{ValueError: } Invalid value.
        \item \textbf{IndexError: } Index out of bounds.
    \end{enumerate}
    \vfill
\end{frame}

%
% Slide 2
%
\begin{frame}[fragile]
    \frametitle{Try-Catch Template}
    \vfill
    \begin{lstlisting}[language=Python, autogobble]
    try:
        print(string[3], end=" ")
    except <ExceptionType>:
        # Some code ...
    \end{lstlisting}
    \vfill
    or
    \vfill
    \begin{lstlisting}[language=Python, autogobble]
    try:
        print(string[3], end=" ")
    except <ExceptionType> as e:
        # Other code....
        print(e)
    \end{lstlisting}
    \vfill
\end{frame}

%
% Slide 2
%
\begin{frame}[fragile]
    \frametitle{A Warning Against Bare Excepts}
    Will this ever exit if the user trys to hit press Ctrl+C?
    \begin{lstlisting}[language=Python, autogobble]
    while True:
        try:
            print("Hi")
        except:
            continue
    \end{lstlisting}
    \pause
    \vfill
    No, it wont. So be sure to always have at least:
    \vfill
    \begin{lstlisting}[language=Python, autogobble]
    while True:
        try:
            print("Hi")
        except Exception:
            continue
    \end{lstlisting}
    \vfill
\end{frame}

\section{Modules}

%
% Slide 2
%
\begin{frame}[fragile]
    \frametitle{Modules and the Standard Library}
    Let's import some modules we've used in the class...
    \begin{lstlisting}[language=Python, autogobble]
    import sys
    import numpy
    import requests
    import math
    import pygame

    for module in sys.modules:
        print(module)
    \end{lstlisting}
    \pause
    \vfill
    \begin{enumerate}
        \item There's lots of other modules:
        \begin{enumerate}[A]
            \item \lstinline|flask| \textrightarrow http server
            \item \lstinline|bottle| \textrightarrow a simple http server
            \item \lstinline|BeautifulSoup| \textrightarrow an html processing library.
            \item \lstinline|pandas| \textrightarrow a data processing library.
        \end{enumerate}
        \item \textbf{Standard Library: } Modules that have been imported for you.
    \end{enumerate}
\end{frame}

\section{unittest}

%
% Slide 2
%
\begin{frame}[fragile]
    \frametitle{Meet how PrairieLearn grades your code}
    \begin{lstlisting}[language=Python, autogobble]
    import unittest 

    class TestFoo(unittest.TestCase):
        def test0(self):
            # assert something 
        def test1(self):
            # assert something 
        # ...
        def testn(self):
            # assert something 
    \end{lstlisting}
    \vfill
    \begin{enumerate}
        \pause
        \item \lstinline|unittest| \textrightarrow \ This is a very large module with lots of classes and functionality.
        \pause
        \item \lstinline|class TestFoo(unittest.TestCase)| \textrightarrow Our new class \textit{extends} unittest.TestCase and thus gets all of it's functionality plus whatever test cases we add..
    \end{enumerate}
\end{frame}


%
% Slide 2
%
\begin{frame}[fragile]
    \frametitle{\lstinline|unittest| Assertions}
    \begin{table}[]
        \small
        \begin{tabular}{l|l}
            \hline
            assertEqual(a, b)        & assert a == b               \\ \hline
            assertNotEqual(a,b)      & assert a != b               \\ \hline
            assertTrue(x)            & assert bool(x) is True      \\ \hline
            assertFalse(x)           & assert bool(x) is False     \\ \hline
            assertIs(a, b)           & assert a is b               \\ \hline
            assertIsNot(a,b)         & assert a is not b           \\ \hline
            assertIsNone(x)          & assert x is None            \\ \hline
            assertIsNotNone(x)       & assert x is not None        \\ \hline
            assertIn(a, b)           & assert a in b               \\ \hline
            assertNotIn(a, b)        & assert a not in b           \\ \hline
            assertAlmostEqual(a, b)  & assert round(a - b, 7) == 0 \\ \hline
            assertGreater(a, b)      & assert a \textgreater b     \\ \hline
            assertGreaterEqual(a, b) & assert a \textgreater{}= b  \\ \hline
            assertLess(a, b)         & assert a \textless b        \\ \hline
            assertLessEqual(a, b)    & assert a \textless{}= b     \\ \hline
        \end{tabular}
    \end{table}
\end{frame}

%
% Slide 2
%
\begin{frame}[fragile]
    \frametitle{Meet how PrairieLearn grades your code}
    \centering
    \vfill
    Lets see how this works in practice. Look at class info for a Colab link.
    \vfill
\end{frame}


\end{document}
