\documentclass{beamer}

\usepackage{graphicx}
\usepackage{textpos}
\usepackage{listings}
\usepackage{lstautogobble}

\usetheme{Madrid}
\useoutertheme{miniframes} % Alternatively: miniframes, infolines, split

\newcommand{\xdownarrow}[1]{%
  {\left\downarrow\vbox to #1{}\right.\kern-\nulldelimiterspace}
}

% Setup the university's color pallette
\definecolor{UIUCorange}{RGB}{19, 41, 75} % UBC Blue (primary)
\definecolor{UIUCblue}{RGB}{232, 74, 39} % UBC Grey (secondary)

\definecolor{codegreen}{rgb}{0,0.6,0}
\definecolor{codegray}{rgb}{0.5,0.5,0.5}
\definecolor{codepurple}{rgb}{0.58,0,0.82}
\definecolor{backcolour}{rgb}{0.95,0.95,0.92}

\lstdefinestyle{python}{
  backgroundcolor=\color{backcolour},   
  commentstyle=\color{codegreen},
  keywordstyle=\color{magenta},
  numberstyle=\tiny\color{codegray},
  stringstyle=\color{codepurple},
  basicstyle=\ttfamily\footnotesize,
  breakatwhitespace=false,         
  belowskip=-0.5em,
  breaklines=true,                 
  captionpos=b,                    
  keepspaces=true,                 
  numbers=left,                    
  numbersep=5pt,                  
  showspaces=false,                
  showstringspaces=false,
  showtabs=false,                  
  tabsize=2
}

\lstset{style=python}

\AtBeginSection[]{
  \begin{frame}
    \vfill
    \centering
    \begin{beamercolorbox}[sep=8pt,center,shadow=true,rounded=true]{title}
      \usebeamerfont{title}\insertsectionhead\par%
    \end{beamercolorbox}
    \vfill
  \end{frame}
}
% Setup the university's color pallette
\definecolor{UIUCorange}{RGB}{19, 41, 75} % UBC Blue (primary)
\definecolor{UIUCblue}{RGB}{232, 74, 39} % UBC Grey (secondary)


\setbeamercolor{palette primary}{bg=UIUCorange,fg=white}
\setbeamercolor{palette secondary}{bg=UIUCblue,fg=white}
\setbeamercolor{palette tertiary}{bg=UIUCblue,fg=white}
\setbeamercolor{palette quaternary}{bg=UIUCblue,fg=white}
\setbeamercolor{structure}{fg=UIUCorange} % itemize, enumerate, etc
\setbeamercolor{section in toc}{fg=UIUCblue} % TOC sections

\setbeamercolor{subsection in head/foot}{bg=UIUCorange,fg=UIUCblue}
\setbeamercolor{subsection in head/foot}{bg=UIUCorange,fg=UIUCblue}

\usepackage[utf8]{inputenc}
\usepackage{graphicx}

%Information to be included in the title page:
\title{\textbf{Topic 2: Vars and Expressions}}
\author{\textbf{David H Smith IV}}
\institute[\textbf{UIUC}]{\textbf{University of Illinois Urbana-Champaign}}
\date{\textbf{Thur, Aug 26 2021}}

\setbeamertemplate{title page}[default][colsep=-4bp,rounded=true]
\addtobeamertemplate{title page}{\vspace{3\baselineskip}}{}
\addtobeamertemplate{title page}{
  \begin{textblock*}{\paperwidth}(-1.0em, -1.2em)
    \includegraphics[width=\paperwidth, height=\paperheight]{imgs/uiuc.png}
  \end{textblock*} 
}{}

\begin{document}

\frame{\titlepage}

\section{Announcements}

\begin{frame}
  \begin{enumerate}
    \item Homework 2 (Part 2) and Post-Reading for Topic 3 (Part 1) are posted and due Friday.
    \item Participation Topic 3 (Part 1) is due Friday
    \item Challenge Activities for Topic 2 are due Sunday.
  \end{enumerate}
\end{frame}

\section{Review Poll Questions}

%
%
%
\begin{frame}[fragile]
  \frametitle{Poll Question: }
  How many of the following characters are visible on the screen?\\
  \begin{lstlisting}[language=Python, autogobble]
  print("\t\\n\\\t")
  \end{lstlisting}
  \vfill
  \begin{minipage}{.48\textwidth}
    \begin{enumerate}[A]
      \item 1
      \item 2
      \item 3
      \item 4
      \item 5
    \end{enumerate}
  \end{minipage}
  \begin{minipage}{.48\textwidth}
  \end{minipage}
\end{frame}


\section{Math Operators}

%
%
%
\begin{frame}[fragile]
  \frametitle{Poll Question: Multiplication}
  \vfill
  What is the result of the following?
  \begin{lstlisting}[language=Python, autogobble]
  x = 4(10 + 2)
  print(x)
  \end{lstlisting}
  \vfill
  \begin{enumerate}[A]
    \item 24
    \item SyntaxError
    \item TypeError
    \item ValueError
  \end{enumerate}
\end{frame}

%
%
%
\begin{frame}[fragile]
  \frametitle{Poll Question: Multiplication}
  \vfill
  What is the value of \lstinline|y| after this code executes?
  \begin{lstlisting}[language=Python, autogobble]
  x = 2
  y = x + 3
  x = 3
  \end{lstlisting}
  \vfill
  \begin{enumerate}[A]
    \item 2 
    \item 3
    \item 5
    \item 8
    \item 10
  \end{enumerate}
\end{frame}


%
%
%
\begin{frame}[fragile]
  \frametitle{Poll Question: Multiplication}
  \vfill
  What is the value of \lstinline|y| after this code executes?
  \begin{lstlisting}[language=Python, autogobble]
  x = 7
  y = x
  x = x + 2
  \end{lstlisting}
  \vfill
  \begin{enumerate}
    \item 2
    \item 5 
    \item 7
    \item 9
  \end{enumerate}
\end{frame}

%
%
%
\begin{frame}[fragile]
  \frametitle{Poll Question: Multiplication}
  \vfill
  What is the value of this expression?
  \begin{lstlisting}
  -3 ** 2
  \end{lstlisting}
  \vfill
  \begin{enumerate}
    \item -9
    \item -8
    \item 8
    \item 9
  \end{enumerate}
\end{frame}


%
%
%
\begin{frame}[fragile]
  \frametitle{Poll Question: More Math Operators}
  \vfill
  Which computes how many (whole) apples I can give to each friend?
  \begin{enumerate}
    \item \lstinline|num_apples / num_friends|
    \item \lstinline|num_friends // num_apples|
    \item \lstinline|num_apples // num_friends|
    \item \lstinline|num_friends % num_appples|
    \item \lstinline|num_apples % num_friends|
  \end{enumerate}
\end{frame}

%
%
%
\begin{frame}[fragile]
  \frametitle{Poll Question: More Math Operators}
  \vfill
  Which computes how many (whole) apples you have left over if you give \lstinline|num_apples| to \lstinline|num_friends|?
  \begin{enumerate}
    \item \lstinline|num_apples / num_friends|
    \item \lstinline|num_friends // num_apples|
    \item \lstinline|num_apples // num_friends|
    \item \lstinline|num_friends % num_appples|
    \item \lstinline|num_apples % num_friends|
  \end{enumerate}
\end{frame}


%
%
%
\begin{frame}[fragile]
  \frametitle{Division, Floor Division, and Modulo}
  \begin{enumerate}
    \item Division operator (/) gives best approximation to true result and \textit{always return a float}.
    \item Floor division (//) rounds down the closet whole number. The type of the result will follow the normal rules.
    \item Modulo operator(\%) performs a division and returns the remainder. The type of the result will always be the same.
    \item For any numbers x and y, the following equality holds: \lstinline|(y == (y // x) * x + (y \% x))|
  \end{enumerate}
\end{frame}


\section{Orders of Operation}

\begin{frame}
  \frametitle{Order of Operations in Python}
  \centering
  \begin{minipage}{0.49\textwidth}
    \begin{itemize}
      \item Parentheses 
      \item Exponentiation
      \item Positive and negative
      \item Multiplication, Division, Modulo
      \item Addition, Subtraction 
    \end{itemize}
  \end{minipage}
  \begin{minipage}{0.2\textwidth}
    \centering
    Highest\\
    $\xdownarrow{2cm}$\\
    Lowest\\
  \end{minipage}
  \vfill
  \textbf{Note:} Python evaluates from left to right within a precedence level
\end{frame}

\section{Math Module}

%
%
%
\begin{frame}[fragile]
  \frametitle{Poll Question: Rounding}
  What is the result of this code if the user types in \lstinline|4.51| and \lstinline|5.9|?
  \begin{lstlisting}[language=Python, autogobble]
x = math.ceil(float(input()))
y = math.floor(float(input()))
print(x + y)
  \end{lstlisting}
  \vfill
  \begin{enumerate}[A]
    \item SyntaxError
    \item NameError
    \item 10
    \item 10.0
  \end{enumerate}
\end{frame}

%
%
%
\begin{frame}[fragile]
  \frametitle{Poll Question: Rounding}
  What is the result of this code if the user types in \lstinline|4.1| and \lstinline|5.9|?
  \begin{lstlisting}[language=Python, autogobble]
import math
x = math.ceil(float(input()))
y = math.floor(float(input()))
print(x + y)
  \end{lstlisting}
  \vfill
  \begin{enumerate}[A]
    \item 10
    \item 10.0
  \end{enumerate}
\end{frame}

%
%
%
\begin{frame}[fragile]
  \frametitle{Poll Question: More Math Operators}
  \vfill
  Which of the following will print the value of $\pi$?
  \begin{enumerate}[A]
    \item \lstinline|print(math.pi)|
    \item \lstinline|print(pi)|
    \item import math.pi\\print(math.pi)
    \item import math\\print(math.pi)
  \end{enumerate}
\end{frame}


\begin{frame}[fragile]
  \frametitle{Math Module}
  \begin{enumerate}[A]
    \item Should I memorize (operators, function, modules, module functions, etc.)?
    \item Yes and no. 
    \item Google + \lstinline|help()| function are your friends
    \item Modules vs Scripts: Modules are just (for the purposes of what we've discussed so far) scripts that someone else wrote that you can use in your own scripts.
  \end{enumerate}
  \vfill
  To get information on a module:
  \begin{lstlisting}
  import math
  help(math)
  \end{lstlisting}
\end{frame}

\begin{frame}
  \frametitle{\lstinline|__name__| and \lstinline|"__main__"|}

  \begin{enumerate}
    \item When you run a script in python it gets a few "environment variates".
    \item For the script you run (e.g, \lstinline|test.py|) the \lstinline|__name__| variable will always be \lstinline|"__main__"|.
    \item For any scripts/modules you import \lstinline|__name__| variable will always be the name of that script/module.
  \end{enumerate}
\end{frame}

\section{Lab}


\section{Lab 1}

\begin{frame}
  \frametitle{Section 1 - Python + TextEditor}
\end{frame}

\begin{frame}
  \frametitle{Section 2  - Git Setup}
\end{frame}

\begin{frame}
  \frametitle{Section 2  - GitHub Setup}
\end{frame}

%
%
%
\begin{frame}
  \begin{enumerate}
    \item Homework 2 (Part 2) and Post-Reading for Topic 3 (Part 1) are posted and due Friday.
    \item Participation Topic 3 (Part 1) is due Friday
    \item Challenge Activities for Topic 2 are due Sunday.
  \end{enumerate}
\end{frame}


\end{document}
