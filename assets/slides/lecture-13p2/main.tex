\documentclass{beamer}

\usepackage{graphicx}
\usepackage{textpos}
\usepackage{listings}
\usepackage{lstautogobble}

\usetheme{Madrid}
\useoutertheme{miniframes} % Alternatively: miniframes, infolines, split

% Setup the university's color pallette
\definecolor{UIUCorange}{RGB}{19, 41, 75} % UBC Blue (primary)
\definecolor{UIUCblue}{RGB}{232, 74, 39} % UBC Grey (secondary)

\definecolor{codegreen}{rgb}{0,0.6,0}
\definecolor{codegray}{rgb}{0.5,0.5,0.5}
\definecolor{codepurple}{rgb}{0.58,0,0.82}
\definecolor{backcolour}{rgb}{0.95,0.95,0.92}

\lstdefinestyle{python}{
  backgroundcolor=\color{backcolour},   
  commentstyle=\color{codegreen},
  keywordstyle=\color{magenta},
  numberstyle=\tiny\color{codegray},
  stringstyle=\color{codepurple},
  basicstyle=\ttfamily\footnotesize,
  breakatwhitespace=false,         
  belowskip=-0.5em,
  breaklines=true,                 
  captionpos=b,                    
  keepspaces=true,                 
  %numbers=left,                    
  numbersep=5pt,                  
  showspaces=false,                
  showstringspaces=false,
  showtabs=false,                  
  tabsize=2
}

\lstset{style=python}

\AtBeginSection[]{
  \begin{frame}
    \vfill
    \centering
    \begin{beamercolorbox}[sep=8pt,center,shadow=true,rounded=true]{title}
      \usebeamerfont{title}\insertsectionhead\par%
    \end{beamercolorbox}
    \vfill
  \end{frame}
}
% Setup the university's color pallette
\definecolor{UIUCorange}{RGB}{19, 41, 75} % UBC Blue (primary)
\definecolor{UIUCblue}{RGB}{232, 74, 39} % UBC Grey (secondary)


\setbeamercolor{palette primary}{bg=UIUCorange,fg=white}
\setbeamercolor{palette secondary}{bg=UIUCblue,fg=white}
\setbeamercolor{palette tertiary}{bg=UIUCblue,fg=white}
\setbeamercolor{palette quaternary}{bg=UIUCblue,fg=white}
\setbeamercolor{structure}{fg=UIUCorange} % itemize, enumerate, etc
\setbeamercolor{section in toc}{fg=UIUCblue} % TOC sections

\setbeamercolor{subsection in head/foot}{bg=UIUCorange,fg=UIUCblue}
\setbeamercolor{subsection in head/foot}{bg=UIUCorange,fg=UIUCblue}

\usepackage[utf8]{inputenc}
\usepackage{graphicx}

%Information to be included in the title page:
\title{\textbf{Adv. Functions}}
\author{\textbf{David H Smith IV}}
\institute[\textbf{UIUC}]{\textbf{University of Illinois Urbana-Champaign}}
\date{\textbf{Tues, Nov 16 2021}}

\setbeamertemplate{title page}[default][colsep=-4bp,rounded=true]
\addtobeamertemplate{title page}{\vspace{3\baselineskip}}{}
\addtobeamertemplate{title page}{
  \begin{textblock*}{\paperwidth}(-1.0em, -1.2em)
    \includegraphics[width=\paperwidth, height=\paperheight]{imgs/uiuc.png}
  \end{textblock*} 
}{}

\begin{document}

\frame{\titlepage}

\section{Reminders}

%
% Slide 1
%
\begin{frame}
  \frametitle{Reminders}
  The folowing are due on Friday:
  \begin{itemize}
    \item \textbf{PrairieLearn:} Homework 13p2, Post-reading 14p1
    \item \textbf{zyBooks:} Participation 14p1
  \end{itemize}
  \vfill
  Due Monday:
  \begin{itemize}
    \item \textbf{zyBooks:} Topic 13 - Challenge Activities 
  \end{itemize}
  \vfill
  Lab will be due December 3rd.
\end{frame}

\section{Namespaces and Scope Resolution}

%
% Slide 2
%
\begin{frame}[fragile]
  \frametitle{Namespaces, Scope, and Scope Resolution}
  \begin{itemize}
    \item \textbf{Namespaces: } A mapping between names and objects.
    \pause 
    \item \textbf{Scope: } The hierarchy that defines where we have access to what variables.
    \pause
    \item \textbf{Scope Resolution (LEGB rule):}
      \begin{enumerate}
        \item \textit{Local:} Things defined in a function.
        \item \textit{Enclosed:} Things defined in a nested function.
        \item \textit{Global:} Things defined in the program as a whole.
        \item \textit{Built-in:} Names that are built-in to Python like \lstinline|int()|.
      \end{enumerate}
    \pause
    \item Searches up the levels of the hierarchy.
  \end{itemize}
\end{frame}


%
% Slide 2
%
\begin{frame}[fragile]
  \frametitle{Poll Question: Function Scope}
  What is printed after the following function runs?
  \begin{lstlisting}[language=Python, autogobble]
  x = [1, 2, 3]
  def foo():
    x = []
  foo()
  print(x)
  \end{lstlisting}
  \vfill
  \begin{enumerate}[A]
    \item \lstinline|[]|
    \item \lstinline|[1, 2, 3]|
    \item NameError
  \end{enumerate}
\end{frame}

%
% Slide 2
%
\begin{frame}[fragile]
  \frametitle{Poll Question: Function Scope}
  What is printed after the following function runs?
  \begin{lstlisting}[language=Python, autogobble]
  x = [1, 2, 3]
  def foo():
    x.append(4)
  foo()
  print(x)
  \end{lstlisting}
  \vfill
  \begin{enumerate}[A]
    \item \lstinline|[1, 2, 3, 4]|
    \item \lstinline|[1, 2, 3]|
    \item NameError
  \end{enumerate}
\end{frame}

%
% Slide 2
%
\begin{frame}[fragile]
  \frametitle{Poll Question: Function Scope}
  What is printed after the following function runs?
  \begin{lstlisting}[language=Python, autogobble]
  x = [1, 2, 3]
  def foo():
    global x
    x = []
  foo()
  print(x)
  \end{lstlisting}
  \vfill
  \begin{enumerate}[A]
    \item \lstinline|[]|
    \item \lstinline|[1, 2, 3]|
    \item NameError
  \end{enumerate}
\end{frame}

%
% Slide 2
%
\begin{frame}[fragile]
  \frametitle{Poll Question: Function Scope}
  What is printed after the following function runs?
  \begin{lstlisting}[language=Python, autogobble]
  x = 1
  def foo():
    print("x" in locals(), end=" ")
    x = 2
    print("x" in locals(), end=" ")
  foo()
  \end{lstlisting}
  \vfill
  \begin{enumerate}[A]
    \item \lstinline|True True|
    \item \lstinline|True False|
    \item \lstinline|False True|
    \item NameError
  \end{enumerate}
\end{frame}

%
% Slide 2
%
\begin{frame}[fragile]
  \frametitle{Poll Question: Function Scope}
  What is printed after the following function runs?
  \begin{lstlisting}[language=Python, autogobble]
  x = 1
  def foo():
    x += 1
  print(x)
  foo()
  print(x)
  \end{lstlisting}
  \vfill
  \begin{enumerate}[A]
    \item \lstinline|1 2|
    \item \lstinline|1 1|
    \item \lstinline|2 2|
    \item UnboundedLocal
  \end{enumerate}
\end{frame}

%
% Slide 2
%
\begin{frame}[fragile]
  \frametitle{Poll Question: Function Scope}
  Why can we do this without using \lstinline|global|$\cdots$
  \vfill
  \begin{lstlisting}[language=Python, autogobble]
  x = [1, 2, 3]
  def foo():
    x.append(4)
  foo()
  \end{lstlisting}
  \vfill
  and not this without \lstinline|global|?
  \vfill
  \begin{lstlisting}[language=Python, autogobble]
  x = 1
  def foo():
    x -= 2
  foo()
  \end{lstlisting}
\end{frame}


%
% Slide 2
%
\begin{frame}[fragile]
  \frametitle{Poll Question: Function Scope}
  What is printed after the following function runs?
  \begin{lstlisting}[language=Python, autogobble]
  x = 1
  def foo():
    print(x)
    x = 2
    print(x)
  foo()
  \end{lstlisting}
  \vfill
  \begin{enumerate}[A]
    \item \lstinline|1 2|
    \item \lstinline|1 1|
    \item \lstinline|2 2|
    \item UnboundedLocal
  \end{enumerate}
\end{frame}


\section{Default Arguments}

%
% Slide 2
%
\begin{frame}[fragile]
  \frametitle{Default Arguments: Poll Question}
  \begin{lstlisting}[language=Python, autogobble]
  def foo(sep=",", num):
    x = []
    for i in range(num):
      x.append(str(i))
    return sep.join(x)
  foo(5)
  \end{lstlisting}
  \vfill
  \begin{enumerate}[A]
    \item \lstinline|'0,1,2,3,4'|
    \item \lstinline|'0,1,2,3,4,5'|
    \item \lstinline|'1,2,3,4,5'|
    \item SyntaxError
  \end{enumerate}
\end{frame}

%
% Slide 2
%
\begin{frame}[fragile]
  \frametitle{Default Arguments: Poll Question}
  \begin{lstlisting}[language=Python, autogobble]
  def foo(num, sep=",", mult=2):
    x = []
    for i in range(num):
      x.append(str(i * mult))
    return sep.join(x)
  foo(5, mult=3, sep=".")
  \end{lstlisting}
  \vfill
  \begin{enumerate}[A]
    \item \lstinline|'0.3.6.9.12'|
    \item \lstinline|'0,3,6,9,12'|
    \item AttributeError
    \item SyntaxError
  \end{enumerate}
\end{frame}

%
% Slide 2
%
\begin{frame}[fragile]
  \frametitle{Default Arguments: Poll Question}
  \begin{lstlisting}[language=Python, autogobble]
  def foo(num, sep=",", mult=2):
    x = []
    for i in range(num):
      x.append(str(i * mult))
    return sep.join(x)
  foo(5, ".", 3)
  \end{lstlisting}
  \vfill
  \begin{enumerate}[A]
    \item \lstinline|'0.3.6.9.12'|
    \item \lstinline|'0,3,6,9,12'|
    \item AttributeError
    \item SyntaxError
  \end{enumerate}
\end{frame}

%
% Slide 2
%
\begin{frame}[fragile]
  \frametitle{Default Arguments: Poll Question}
  \begin{lstlisting}[language=Python, autogobble]
  def foo(num, step=1, mult=2):
    x = []
    for i in range(num, step=step):
      x.append(str(i * mult))
    return ",".join(x)
  foo(5, mult=3)
  \end{lstlisting}
  \vfill
  \begin{enumerate}[A]
    \item \lstinline|'0,3,6,9,12'|
    \item \lstinline|NameError|
    \item AttributeError
    \item SyntaxError
  \end{enumerate}
\end{frame}


%
% Slide 2
%
\begin{frame}[fragile]
  \frametitle{Key Take Aways}
  \begin{enumerate}
      \pause
    \item Default arguments must follow non-default arguments (e.g., \lstinline|def qux(a, b=3)|.
      \pause
    \item You can use position to pass values in for default arguments.
      \pause
    \item You can switch positions of default arguments (or arguments in general) if you use their names when calling the function.
  \end{enumerate}
\end{frame}


\section{*args}

%
% Slide 2
%
\begin{frame}[fragile]
  \frametitle{Default Arguments: Poll Question}
  \begin{lstlisting}[language=Python, autogobble]
  def foo(*args):
    print(type(args))
  \end{lstlisting}
  \vfill
  \begin{enumerate}[A]
    \item list
    \item tuple
    \item set
    \item something else?
  \end{enumerate}
\end{frame}

%
% Slide 2
%
\begin{frame}[fragile]
  \frametitle{Default Arguments: Poll Question}
  What is returned and printed by the function call at the bottom?
  \begin{lstlisting}[language=Python, autogobble]
  def foo(*things):
    x = []
    for thing in things:
      if type(thing) is int:
        x.append(thing)
    return x
  print(foo(1, 2, 4.5, 3.4, "bar", "baz"))    
  \end{lstlisting}
  \vfill
  \begin{enumerate}[A]
    \item \lstinline|[1, 2]|
    \item \lstinline|[1, 2, 4.5, 3.4]|
    \item SyntaxError
    \item NameError
  \end{enumerate}
\end{frame}


%
% Slide 2
%
\begin{frame}[fragile]
  \frametitle{Default Arguments: Poll Question}
  What is returned by the function call at the bottom?
  \begin{lstlisting}[language=Python, autogobble]
  def foo(*stuff, num):
    x = []
    for thing in stuff:
      if thing % num == 0 and type(thing) is int:
        x.append(thing)
    return x
  foo(5, 10, 3, "hello", "World", num=5)
  \end{lstlisting}
  \vfill
  \begin{enumerate}[A]
    \item \lstinline|[5, 10]|
    \item \lstinline|[5, 10, "hello", "World"]|
    \item SyntaxError
    \item TypeError
  \end{enumerate}
\end{frame}

%
% Slide 2
%
\begin{frame}[fragile]
  \frametitle{Default Arguments: Poll Question}
  What about now?
  \begin{lstlisting}[language=Python, autogobble]
  def foo(*stuff, num):
    x = []
    for thing in stuff:
      if type(thing) is int and thing % num == 0:
        x.append(thing)
    return x
  foo(5, 10, 3, "hello", "World", num=5)
  \end{lstlisting}
  \vfill
  \begin{enumerate}[A]
    \item \lstinline|[5, 10]|
    \item \lstinline|[5, 10, "hello", "World"]|
    \item SyntaxError
    \item TypeError
  \end{enumerate}
\end{frame}


%
% Slide 2
%
\begin{frame}[fragile]
  \frametitle{Default Arguments: Poll Question}
  What is returned by the function call in the code below?
  \begin{lstlisting}[language=Python, autogobble]
  def foo(total, *vals):
    return total == sum(vals)
  foo(15, 1, 2, 3, 4, 5)
  \end{lstlisting}
  \vfill
  \begin{enumerate}[A]
    \item \lstinline|[5, 10]|
    \item \lstinline|[5, 10, "hello", "World"]|
    \item SyntaxError
    \item TypeError
  \end{enumerate}
\end{frame}

%
% Slide 2
%
\begin{frame}[fragile]
  \frametitle{Default Arguments}
  This$\cdots$
  \vfill
  \begin{lstlisting}[language=Python, autogobble]
  def foo(*vals):
    return sum(vals)
  foo(1, 2, 3)
  \end{lstlisting}
  \vfill
  is functionally equivalent to this$\cdots$
  \vfill
  \begin{lstlisting}[language=Python, autogobble]
  def foo(vals):
    return sum(vals)
  foo([1, 2, 3])
  \end{lstlisting}
  \vfill
  So arbitrary arguments are more syntactic sugar added by Python to make your code more versatile.
\end{frame}

%
% Slide 2
%
\begin{frame}[fragile]
  \frametitle{Default Arguments: Why?}
  This$\cdots$
  \vfill
  \begin{lstlisting}[language=Python, autogobble]
  def foo(*vals):
    total = 0
    for val in vals:
      total += val
    return total
  foo(1, 2)
  foo(1, 2, 3)
  \end{lstlisting}
  \vfill
  avoids the need for this\dots
  \vfill
  \begin{lstlisting}[language=Python, autogobble]
  def foo1(a,b):
    return a + b 
  def foo2(a,b,c):
    return a + b + c
  foo1(1, 2)
  foo1(1, 2, 3)
  \end{lstlisting}
  \vfill
\end{frame}


%
% Slide 2
%
\begin{frame}[fragile]
  \frametitle{Key Take Aways}
  \begin{enumerate}[A]
    \item The * operator is the important part. *args is only used by convention.
    \pause
    \item *args can precede other parameters in the function definition however the other parameters must be called as named variables. For example:
    \pause
    \begin{lstlisting}[language=Python, autogobble]
    def foo(*vals, num):
      ...
    foo(1, 2, 3, num=100)
    \end{lstlisting}
    \vfill
    vs
    \vfill
    \begin{lstlisting}[language=Python, autogobble]
    def foo(num, *vals):
      ...
    foo(100, 1, 2, 3)
    \end{lstlisting}
  \end{enumerate}
\end{frame}


\section{**kwargs}

%
% Slide 2
%
\begin{frame}[fragile]
  \frametitle{Default Arguments: Poll Question}
  What is printed out when the code below runs?
  \begin{lstlisting}[language=Python, autogobble]
  def foo(**kwargs):
    print(type(kwargs))
  foo(a="thing", b="thing2", c=3)
  \end{lstlisting}
  \vfill
  \begin{enumerate}[A]
    \item list
    \item tuple
    \item set
    \item dict
  \end{enumerate}
\end{frame}


%
% Slide 2
%
\begin{frame}[fragile]
  \frametitle{Default Arguments: Poll Question}
  \begin{lstlisting}[language=Python, autogobble]
  def foo(**named_stuff, num):
    x = []
    for name, value in named_stuff.items():
      if value > num:
        x.append(value)
    return x
  print(foo(a=10, b=4, c=1, d=15, num=5))
  \end{lstlisting}
  \vfill
  \begin{enumerate}[A]
    \item \lstinline|[10, 15]|
    \item \lstinline|None|
    \item SyntaxError
    \item NameError
  \end{enumerate}
  \pause
  \vfill
  Why?
\end{frame}

%
% Slide 2
%
\begin{frame}[fragile]
  \frametitle{Default Arguments: Poll Question}
  \begin{lstlisting}[language=Python, autogobble]
  def foo(num, **more_named_stuff):
    x = ""
    for key, value in more_named_stuff.items():
      if value % num == 0:
        x += key
    return x
  foo(2, a=2, b=4, c=5, d=6)
  \end{lstlisting}
  \vfill
  \begin{enumerate}[A]
    \item \lstinline|'abd'|
    \item TypeError
    \item ValueError
    \item AttributeError
  \end{enumerate}
\end{frame}

%
% Slide 2
%
\begin{frame}[fragile]
  \frametitle{Default Arguments: Poll Question}
  \begin{lstlisting}[language=Python, autogobble]
  def foo(arg1, arg2=2, **args):
    return arg1, arg2, args
  foo(5, a=1, b=2, c=3)
  \end{lstlisting}
  \vfill
  \begin{enumerate}[A]
    \item \lstinline|(5, 2, {'a': 1, 'b': 2, 'c': 3})|
    \item \lstinline|(5, 1, {'a': 1, 'b': 2, 'c': 3})|
    \item SyntaxError
    \item NameError
  \end{enumerate}
\end{frame}

%
% Slide 2
%
\begin{frame}[fragile]
  \frametitle{Default Arguments: Poll Question}
  \begin{lstlisting}[language=Python, autogobble]
  def foo(arg1, arg2=2, **args):
    return arg1, arg2, args
  foo(5, a=1, b=2, c=3, arg2=-1)
  \end{lstlisting}
  \vfill
  \begin{enumerate}[A]
    \item \lstinline|(5, 2, {'a': 1, 'b': 2, 'c': 3})|
    \item \lstinline|(5, -1, {'a': 1, 'b': 2, 'c': 3})|
    \item SyntaxError
    \item NameError
  \end{enumerate}
\end{frame}


%
% Slide 2
%
\begin{frame}[fragile]
  \frametitle{Key Take Aways}
  \begin{enumerate}[A]
    \item As with *args, the ** operator is the important part. **kwargs is only used by convention.
      \pause
    \item **kwargs CANNOT precede other parameters in the function definition.
      \pause
      \begin{lstlisting}[language=Python, autogobble]
    def foo(**vals, num):
      ...
    foo(val1=1, val2=2, val3=3, num=100)
      \end{lstlisting}
      \vfill
      \pause
    \item **kwargs CANNOT precede *args:
      \begin{lstlisting}[language=Python, autogobble]
    def foo(num, **vals, *args):
      ...
    foo(num=100, 1, 2, 3)
      \end{lstlisting}
      \vfill
      \pause
    \item \textbf{Generally, the valid order is: \lstinline|foo(val1, val2=10, *args, **kwargs)|}
  \end{enumerate}
\end{frame}

\end{document}
