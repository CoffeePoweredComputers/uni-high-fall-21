\documentclass[xcolor=table]{beamer}

\usepackage[table,xcdraw]{xcolor}
\usepackage{graphicx}
\usepackage{textpos}
\usepackage{listings}
\usepackage{lstautogobble}

\usetheme{Madrid}
\useoutertheme{miniframes} % Alternatively: miniframes, infolines, split

% Setup the university's color pallette
\definecolor{UIUCorange}{RGB}{19, 41, 75} % UBC Blue (primary)
\definecolor{UIUCblue}{RGB}{232, 74, 39} % UBC Grey (secondary)

\definecolor{codegreen}{rgb}{0,0.6,0}
\definecolor{codegray}{rgb}{0.5,0.5,0.5}
\definecolor{codepurple}{rgb}{0.58,0,0.82}
\definecolor{backcolour}{rgb}{0.95,0.95,0.92}

\lstdefinestyle{python}{
  backgroundcolor=\color{backcolour},   
  commentstyle=\color{codegreen},
  keywordstyle=\color{magenta},
  numberstyle=\tiny\color{codegray},
  stringstyle=\color{codepurple},
  basicstyle=\ttfamily\footnotesize,
  breakatwhitespace=false,         
  belowskip=-0.5em,
  breaklines=true,                 
  captionpos=b,                    
  keepspaces=true,                 
  numbers=left,                    
  numbersep=5pt,                  
  showspaces=false,                
  showstringspaces=false,
  showtabs=false,                  
  tabsize=2
}

\lstset{style=python}

\AtBeginSection[]{
    \begin{frame}
        \vfill
        \centering
        \begin{beamercolorbox}[sep=8pt,center,shadow=true,rounded=true]{title}
            \usebeamerfont{title}\insertsectionhead\par%
        \end{beamercolorbox}
        \vfill
    \end{frame}
}
% Setup the university's color pallette
\definecolor{UIUCorange}{RGB}{19, 41, 75} % UBC Blue (primary)
\definecolor{UIUCblue}{RGB}{232, 74, 39} % UBC Grey (secondary)


\setbeamercolor{palette primary}{bg=UIUCorange,fg=white}
\setbeamercolor{palette secondary}{bg=UIUCblue,fg=white}
\setbeamercolor{palette tertiary}{bg=UIUCblue,fg=white}
\setbeamercolor{palette quaternary}{bg=UIUCblue,fg=white}
\setbeamercolor{structure}{fg=UIUCorange} % itemize, enumerate, etc
\setbeamercolor{section in toc}{fg=UIUCblue} % TOC sections

\setbeamercolor{subsection in head/foot}{bg=UIUCorange,fg=UIUCblue}
\setbeamercolor{subsection in head/foot}{bg=UIUCorange,fg=UIUCblue}

\usepackage[utf8]{inputenc}
\usepackage{graphicx}

%Information to be included in the title page:
\title{\textbf{Review(ish)}}
\author{\textbf{David H Smith IV}}
\institute[\textbf{UIUC}]{\textbf{University of Illinois Urbana-Champaign}}
\date{\textbf{Thurs, Oct 21 2021}}

\setbeamertemplate{title page}[default][colsep=-4bp,rounded=true]
\addtobeamertemplate{title page}{\vspace{3\baselineskip}}{}
\addtobeamertemplate{title page}{
  \begin{textblock*}{\paperwidth}(-1.0em, -1.2em)
    \includegraphics[width=\paperwidth, height=\paperheight]{imgs/uiuc.png}
  \end{textblock*} 
}{}

\begin{document}

\frame{\titlepage}

\section{Reminders}

%
% Slide 1
%
\begin{frame}
  \frametitle{Reminders}
  \begin{itemize}
    \item Last homework of the quarter is due Thursday by the end of the day.
  \end{itemize}
\end{frame}

\section{User Input}

%
%
%
\begin{frame}[fragile]
  \frametitle{Poll Question: Data Types}
  What is the type of x after the following segment of code runs if the user inputs a 5 and then a 4?
  \begin{lstlisting}[language=Python,autogobble]
  a = input()
  b = input()
  x = a + b\end{lstlisting}
  \hfill
  \begin{enumerate}[A]
    \item String
    \item Integer
  \end{enumerate}
\end{frame}

%
%
%
\begin{frame}[fragile]
  \frametitle{Poll Question: Data Types}
  What is the result of running the following code if the user types in \lstinline|world| and then \lstinline|hello|?
  \begin{lstlisting}[language=Python,autogobble]
  a = int(input())
  b = int(input())
  x = a + b\end{lstlisting}
  \vfill
  \begin{enumerate}[A]
    \item SyntaxError
    \item NameError
    \item ValueError
    \item TypeError
  \end{enumerate}
\end{frame}



%
%
%
\begin{frame}[fragile]
  \frametitle{Poll Question: Data Types}
  What is the result of the following code?
  \begin{lstlisting}[language=Python, autogobble]
  def foo(a, b):
    a = int(input())
    b = int(input())
    return a + b
  foo(5, 6)
  \end{lstlisting}
  \vfill
  \begin{enumerate}[A]
    \item Error
    \item 11
    \item "11"
    \item Nothing
  \end{enumerate}
\end{frame}

\section{Data Types}

%
% Slide 6
%
\begin{frame}[fragile]
  \frametitle{Poll Question: Changing Strings}
  What is the value of \lstinline|str1| after the following segment of code executes?
  \begin{lstlisting}[language=Python, autogobble] 
  str1 = "urbana"
  str1[0] = "e"
  \end{lstlisting}
  \vfill
  \begin{enumerate}[A] 
    \item SyntaxError
    \item IndexError
    \item TypeError
    \item \lstinline{'erbana'}
    \item \lstinline{'eurbana'}
  \end{enumerate}
\end{frame}

%
% Slide 7
%
\begin{frame}[fragile]
  \frametitle{Poll Question: Indexing with Lists}
  What is the result of the following code?
  \begin{lstlisting}[language=Python, autogobble]
  l = [1, 2, 3, 4]
  x = l.pop(1)
  y = l.pop(1)
  z = l.remove(1)
  print(x, y, z)
  \end{lstlisting}
  \vfill
  \begin{enumerate}[A] 
    \item \lstinline|2 3 None|
    \item \lstinline|None None 4|
    \item \lstinline|2 3 4|
    \item IndexError
  \end{enumerate}
\end{frame}

%
% Slide 8
%
\begin{frame}[fragile]
  \frametitle{Poll Question: Appending}
  What is the the value of \lstinline|x| after the following code runs?
  \begin{lstlisting}[language=Python, autogobble]
  x = [1, 2, 3]
  x = x.append(4)
  \end{lstlisting}
  \vfill
  \begin{enumerate}[A] 
    \item \lstinline|[1, 2, 3, 4]|
    \item AttributeError
    \item \lstinline|[1, 2, 3, [4]]|
    \item \lstinline|None|
  \end{enumerate}
\end{frame}

%
% Slide 8
%
\begin{frame}[fragile]
  \frametitle{Poll Question: Extending}
  What is the result of the following code if the user types in
  \lstinline|hello| and \lstinline|world|?
  \begin{lstlisting}[language=Python, autogobble]
  x = list(input())
  y = list(input())
  x.extend(y)
  \end{lstlisting}
  \vfill
  \begin{enumerate}[A] 
    \item \lstinline|["hello", "world"]|
    \item \lstinline|["h", "e", "l", "l", "o", "w", "o", "r", "l", "d"]|
    \item \lstinline|[["hello"], ["world"]]|
    \item \lstinline|[["h", "e", "l", "l", "o"], ["w", "o", "r", "l", "d"]]|
  \end{enumerate}
\end{frame}

%
% Slide 9
%
\begin{frame}[fragile]
  \frametitle{Poll Question: Modify Tuple}
  What is the result of the following code?
  \begin{lstlisting}[language=Python, autogobble]
  x = (1, 2, 3, 4)
  x[1] = 5
  print(x)
  \end{lstlisting}
  \vfill
  \begin{enumerate}[A] 
    \item \lstinline|(5, 2, 3, 4)|
    \item \lstinline|(1, 5, 3, 4)|
    \item TypeError
    \item SyntaxError
  \end{enumerate}
\end{frame}

%
% Slide 9
%
\begin{frame}[fragile]
  \frametitle{Poll Question: Creating Tuples} 
  What is the result of the following code?
  \begin{lstlisting}[language=Python, autogobble]
  x = (1, 2, 3, 4)
  y = (1, 2)
  z = x + y
  print(z)
  \end{lstlisting}
  \vfill
  \begin{enumerate}[A] 
    \item \lstinline|((1, 2, 3, 4), (1, 2))|
    \item \lstinline|(1, 2, 3, 4, 1, 2)|
    \item \lstinline|[(1, 2, 3, 4), (1, 2)]|
    \item TypeError
  \end{enumerate}
\end{frame}

%
% Slide 11
%
\begin{frame}[fragile]
  \frametitle{Poll Question: Updating Sets}
  \begin{lstlisting}[language=Python, autogobble]
  x = {1, 2, 3}
  y = set("aeiou")
  x.append(4)
  z = x + y
  print(z)
  \end{lstlisting}
  \vfill
  \begin{enumerate}[A] 
    \item \lstinline|{1, 2, 3, 4, "aeiou"}|
    \item \lstinline|{1, 2, 3, 4, "a", "e", "i", "o", "u"}|
    \item ValueError
    \item AttributeError
  \end{enumerate}
\end{frame}

%
% Slide 9
%
\begin{frame}[fragile]
  \frametitle{Order and Mutability}
  % Please add the following required packages to your document preamble:
  \begin{table}[]
    \begin{tabular}{l|l|l}
      \hline
      \multicolumn{1}{c|}{}       & \multicolumn{1}{c|}{Ordered}                  & \multicolumn{1}{c}{Mutable}                  \\ \hline
      \multicolumn{1}{c|}{String} & \multicolumn{1}{c|}{\cellcolor[HTML]{34FF34}} & \multicolumn{1}{c}{\cellcolor[HTML]{FE0000}} \\ \hline
      List                        & \cellcolor[HTML]{34FF34}                      & \cellcolor[HTML]{34FF34}                     \\ \hline
      Tuple                       & \cellcolor[HTML]{34FF34}                      & \cellcolor[HTML]{FE0000}                     \\ \hline
      Set                         & \cellcolor[HTML]{FE0000}                      & \cellcolor[HTML]{34FF34}                     \\ \hline
      Dict                        & \cellcolor[HTML]{FE0000}                      & \cellcolor[HTML]{34FF34}                     \\ \hline
    \end{tabular}
  \end{table}
  \pause
  This is a lot to remember. So memorize it through practice rather than through standard memorization. 
\end{frame}


\section{Conditionals}

%
% Slide 2
%
\begin{frame}[fragile]
	\frametitle{Poll Question: If Statements}
	What's the result of running the following code?
	\begin{lstlisting}[language=Python, autogobble]
	x = 5
  y = bool(x == 3 or 4)
	\end{lstlisting}
	\vfill
	\begin{enumerate}[A]
		\item True
		\item False
		\item SyntaxError
	\end{enumerate}
	\pause
  Will it ever be false?
\end{frame}

%
% Slide 2
%
\begin{frame}[fragile]
	\frametitle{Boolean Operators}
	\begin{enumerate}[A]
		\item Why is \lstinline|x == 3 or 4| always True?
		\item Alternatives:
			\begin{enumerate}
				\item \lstinline|x == 3 or x == 4|
				\item \lstinline|x in [3, 4]|
			\end{enumerate}
		\item Types of operators:
			\begin{enumerate}
				\item \textbf{Binary operators:} and, or
				\item \textbf{Unary Operators: } not
			\end{enumerate}
	\end{enumerate}
\end{frame}

\section{Loops}


%
% Slide
%
\begin{frame}[fragile]
  \frametitle{Task: Validate User Input}
  \textbf{Problem Statement:} Create a function that gets 10 words that contain the letter "e", stores them in a list, then returns them. Note that this problems uses nested loops but not break or enumerate.
  \vfill
  \pause
  \begin{lstlisting}[language=Python, autogobble, basicstyle=\tiny]
  def no_e():
    l = []
    for i in range(0, 10):
      word = input("Enter a word with the letter e: ")
      while "e" not in word:
        word = input("Enter a word with the letter e: ")
      l.append(word)
    return l
  \end{lstlisting}
\end{frame}

%
% Slide
%
\begin{frame}[fragile]
  \frametitle{Task: Validate User Input}
  \textbf{Problem Statement:} Create a function that keeps asking the user for strings of an even length and adding them to a list until the user enters a string of an odd length. Then return the final list. You'll want to use a "while True:" loop here.
  \vfill
  \pause
  \begin{lstlisting}[language=Python, autogobble, basicstyle=\tiny]
  def get_even_words():
    l = []
    while True:
      user_in = input("Enter a word with an even number of vowels: ")
      if len(user_in) % 2 != 0:
        print("That word has an odd number of letters. Terminating!!")
        break
      l.append(user_in)
  \end{lstlisting}
\end{frame}


\section{OS and Files}

%
% Slide 2
%
\begin{frame}[fragile]
  \frametitle{Poll Question: Opening a File}
  What is returned by os.getcwd() when called in myfile.py?
  \begin{lstlisting}
  home
  |__tracertong
     |---mydata.csv
     |___dev_website
         |---index.html
         |___elements
             |--- myfile.py   <-- You are here
  \end{lstlisting} 
  \vfill
  \begin{enumerate}[A]
    \item \lstinline|"/home/tracertong/dev_website/elements"|
    \item \lstinline|"/home/tracertong/dev_website/elements/myfile.py"|
    \item \lstinline|"dev_website/elements/myfile.py"|
    \item \lstinline|"elements/myfile.py"|
  \end{enumerate}
\end{frame}

%
% Slide 2
%
\begin{frame}[fragile]
  \frametitle{Poll Question: Opening a File}
  My data.csv contains three rows. What is the result of: \lstinline|len(open("mydata.csv").readlines())|.
  \begin{lstlisting}
  home
  |__tracertong
     |---mydata.csv
     |___dev_website
         |---index.html
         |___elements
             |___ myfile.py   <-- You are here
  \end{lstlisting} 
  \vfill
  \begin{enumerate}[A]
    \item 2
    \item 3
    \item 4
    \item FileNotFoundError
    \item SyntaxError
  \end{enumerate}
\end{frame}

%
% Slide 2
%
\begin{frame}[fragile]
  \frametitle{Poll Question: Opening a File}
  We are script myfile.py and we want to open mydata.csv.
  \begin{lstlisting}
  home
  |__tracertong
     |---mydata.csv
     |___dev_website
         |---index.html
         |___elements
             |___ myfile.py   <-- You are here
  \end{lstlisting} 
  \vfill
  \begin{enumerate}[A]
    \item \lstinline|fo = open("../../mydata.csv")|
    \item \lstinline|fo = open("/../../mydata.csv")|
    \item \lstinline|fo = open("/tracertong/mydata.csv")|
    \item \lstinline|fo = open("../../../mydata.csv")|
  \end{enumerate}
\end{frame}



\end{document}
