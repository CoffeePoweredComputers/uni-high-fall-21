\documentclass{beamer}

\usepackage{graphicx}
\usepackage{textpos}
\usepackage{listings}
\usepackage{lstautogobble}

\usetheme{Madrid}
\useoutertheme{miniframes} % Alternatively: miniframes, infolines, split

% Setup the university's color pallette
\definecolor{UIUCorange}{RGB}{19, 41, 75} % UBC Blue (primary)
\definecolor{UIUCblue}{RGB}{232, 74, 39} % UBC Grey (secondary)

\definecolor{codegreen}{rgb}{0,0.6,0}
\definecolor{codegray}{rgb}{0.5,0.5,0.5}
\definecolor{codepurple}{rgb}{0.58,0,0.82}
\definecolor{backcolour}{rgb}{0.95,0.95,0.92}

\lstdefinestyle{python}{
  backgroundcolor=\color{backcolour},   
  commentstyle=\color{codegreen},
  keywordstyle=\color{magenta},
  numberstyle=\tiny\color{codegray},
  stringstyle=\color{codepurple},
  basicstyle=\ttfamily\footnotesize,
  breakatwhitespace=false,         
  belowskip=-0.5em,
  breaklines=true,                 
  captionpos=b,                    
  keepspaces=true,                 
  numbers=left,                    
  numbersep=5pt,                  
  showspaces=false,                
  showstringspaces=false,
  showtabs=false,                  
  tabsize=2
}

\lstset{style=python}

\AtBeginSection[]{
    \begin{frame}
        \vfill
        \centering
        \begin{beamercolorbox}[sep=8pt,center,shadow=true,rounded=true]{title}
            \usebeamerfont{title}\insertsectionhead\par%
        \end{beamercolorbox}
        \vfill
    \end{frame}
}


\setbeamercolor{palette primary}{bg=UIUCorange,fg=white}
\setbeamercolor{palette secondary}{bg=UIUCblue,fg=white}
\setbeamercolor{palette tertiary}{bg=UIUCblue,fg=white}
\setbeamercolor{palette quaternary}{bg=UIUCblue,fg=white}
\setbeamercolor{structure}{fg=UIUCorange} % itemize, enumerate, etc
\setbeamercolor{section in toc}{fg=UIUCblue} % TOC sections

\setbeamercolor{subsection in head/foot}{bg=UIUCorange,fg=UIUCblue}
\setbeamercolor{subsection in head/foot}{bg=UIUCorange,fg=UIUCblue}

\usepackage[utf8]{inputenc}
\usepackage{graphicx}


%Information to be included in the title page:
\title{\textbf{Topic 5: Conditionals}}
\author{\textbf{David H Smith IV}}
\institute[\textbf{UIUC}]{\textbf{University of Illinois Urbana-Champaign}}
\date{\textbf{Mon, July 05 2021}}

\setbeamertemplate{title page}[default][colsep=-4bp,rounded=true]
\addtobeamertemplate{title page}{\vspace{3\baselineskip}}{}
\addtobeamertemplate{title page}{
  \begin{textblock*}{\paperwidth}(-1.0em, -1.2em)
    \includegraphics[width=\paperwidth, height=\paperheight]{imgs/uiuc.png}
  \end{textblock*} 
}{}

\begin{document}

\frame{\titlepage}

%
% Slide 2
%
\begin{frame}[fragile]
	\frametitle{Truth Table to Expressions}
	\vfill
	The As and Bs in the truth tables correspond to \textit{the result of boolean expressions}.
	\vfill
	\begin{minipage}{0.4\textwidth}
		\begin{lstlisting}[language=Python,autogobble]
	# Get some variables 
	x = int(input())
	y = int(input())

	# Construct the expr
  # and assign to A or B
	A = (x == 3)
	B = (y > 5)
		\end{lstlisting}
	\end{minipage}
	\begin{minipage}{0.59\textwidth}
		\begin{table}[]
			\begin{tabular}{|cc|c|c|c|}
				{\color[HTML]{CD9934} \textbf{A}} & {\color[HTML]{3531FF} \textbf{B}} & \textbf{A OR B} & \textbf{A AND B} & \textbf{NOT A} \\ \hline
				{\color[HTML]{CD9934} T}          & {\color[HTML]{3531FF} T}          & T               & T                & F              \\
				{\color[HTML]{CD9934} T}          & {\color[HTML]{3531FF} F}          & T               & F                & T              \\
				{\color[HTML]{CD9934} F}          & {\color[HTML]{3531FF} T}          & T               & F                & F              \\
				{\color[HTML]{CD9934} F}          & {\color[HTML]{3531FF} F}          & F               & F                & T             
			\end{tabular}
		\end{table}
	\end{minipage}
	\vfill
	\begin{itemize}
		\item You can have \textbf{as many operands (e.g., A, B, C, $\cdots$) as you like}.
		\item The truth tables get BIG as you have to consider more permutations.
	\end{itemize}
\end{frame}

\section{\lstinline|is| vs \lstinline|==|}

%
% Slide 1
%
\begin{frame}[fragile]
	\frametitle{Muddiest Points: \lstinline|is| vs \lstinline|==|}
  \begin{itemize}
    \item \lstinline|x is y| translates to \lstinline|id(x) == id(y)|
    \begin{itemize}
      \item \lstinline|==| is for comparing literal values
      \item \lstinline|is| is for comparing to see \textbf{two variables are referencing the same object}.
    \end{itemize}
    \item And there's a caveat\ldots 
    \begin{itemize}
      \item Python pre-instantiate the first few numbers. This isn't true for large numbers.
    \end{itemize}
  \end{itemize}
\end{frame}


\section{Poll Questions: Conditional Syntax Check}

%
% Slide 2
%
\begin{frame}[fragile]
	\frametitle{Poll Question: Constructing Conditionals}
  Which of the following will correctly check if a number (int or float) is between 1 and 10 inclusive.
	\begin{minipage}{0.69\textwidth}
    1)
		\begin{lstlisting}[language=Python, autogobble,basicstyle=\tiny,numbers=none]
    1 <= x <= 10
		\end{lstlisting}
    \vspace{1cm}
    2)
		\begin{lstlisting}[language=Python, autogobble,basicstyle=\tiny,numbers=none]
    1 =< x =< 10
		\end{lstlisting}
    \vspace{1cm}
    3)
		\begin{lstlisting}[language=Python, autogobble,basicstyle=\tiny,numbers=none]
    x >= 1 and x <= 10
		\end{lstlisting}
	\end{minipage}
	\begin{minipage}{0.29\textwidth}
    \begin{enumerate}[A]
      \item 1
      \item 2
      \item 3
      \item 1 and 2
      \item Another combo...
    \end{enumerate}
	\end{minipage}
\end{frame}

%
% Slide 2
%
\begin{frame}[fragile]
	\frametitle{Poll Question: Constructing Conditionals}
  Which of the following evaluates to \lstinline|True| if a set contains an even number of elements?\\ \hfill

	\begin{minipage}{0.69\textwidth}
    1)
		\begin{lstlisting}[language=Python, autogobble,basicstyle=\tiny,numbers=none]
    0 == len(set1) % 2
		\end{lstlisting}
    \vspace{1cm}
    2)
		\begin{lstlisting}[language=Python, autogobble,basicstyle=\tiny,numbers=none]
    len(set1) % 2 == 0
		\end{lstlisting}
    \vspace{1cm}
    3)
		\begin{lstlisting}[language=Python, autogobble,basicstyle=\tiny,numbers=none]
    not (len(set1) % 2)
		\end{lstlisting}
	\end{minipage}
	\begin{minipage}{0.29\textwidth}
    \begin{enumerate}[A]
      \item 1
      \item 2
      \item 3
      \item 1 and 2
      \item All of the above
    \end{enumerate}
	\end{minipage}
\end{frame}

%
% Slide 2
%
\begin{frame}[fragile]
	\frametitle{Poll Question: Constructing Conditionals}
  Which of the following evaluates to \lstinline|True| if x is greater than 10?
	\begin{minipage}{0.69\textwidth}
    1)
		\begin{lstlisting}[language=Python, autogobble,basicstyle=\tiny,numbers=none]
    x not <= 10 
		\end{lstlisting}
    \vspace{1cm}
    2)
		\begin{lstlisting}[language=Python, autogobble,basicstyle=\tiny,numbers=none]
    not x <= 10
		\end{lstlisting}
    \vspace{1cm}
    3)
		\begin{lstlisting}[language=Python, autogobble,basicstyle=\tiny,numbers=none]
    x > 10
		\end{lstlisting}
	\end{minipage}
	\begin{minipage}{0.29\textwidth}
    \begin{enumerate}[A]
      \item 1
      \item 2
      \item 3
      \item 2 and 3
      \item 1 and 3
    \end{enumerate}
	\end{minipage}
\end{frame}

%
% Slide 2
%
\begin{frame}[fragile]
	\frametitle{Poll Question: Constructing Conditionals}
  Which of the following will correctly check to make sure a list \lstinline|x| does not contain a variable \lstinline|y|.

	\begin{minipage}{0.69\textwidth}
    1)
		\begin{lstlisting}[language=Python, autogobble,basicstyle=\tiny,numbers=none]
    y not in x
		\end{lstlisting}
    \vspace{.5cm}
    2)
		\begin{lstlisting}[language=Python, autogobble,basicstyle=\tiny,numbers=none]
    not (x in y)
		\end{lstlisting}
    \vspace{.5cm}
    3)
		\begin{lstlisting}[language=Python, autogobble,basicstyle=\tiny,numbers=none]
    x not in y
		\end{lstlisting}
	\end{minipage}
	\begin{minipage}{0.29\textwidth}
    \begin{enumerate}[A]
      \item 1
      \item 2
      \item 3
      \item 1 and 2
      \item All of the above
    \end{enumerate}
	\end{minipage}
  \vfill
  \pause
  With \lstinline|in|, position of operands matters. With \lstinline|is| and \lstinline|==|, it does not.
\end{frame}

%
% Slide 2
%
\begin{frame}[fragile]
	\frametitle{Poll Question: Constructing Conditionals}
  Which of the following will check if a value \lstinline|y| is in a dictionary \lstinline|x|?
	\begin{minipage}{0.69\textwidth}
    1)
		\begin{lstlisting}[language=Python, autogobble,basicstyle=\tiny,numbers=none]
    y in x
		\end{lstlisting}
    \vspace{1cm}
    2)
		\begin{lstlisting}[language=Python, autogobble,basicstyle=\tiny,numbers=none]
    y == x
		\end{lstlisting}
    \vspace{1cm}
    3)
		\begin{lstlisting}[language=Python, autogobble,basicstyle=\tiny,numbers=none]
    y is x
		\end{lstlisting}
	\end{minipage}
	\begin{minipage}{0.29\textwidth}
    \begin{enumerate}[A]
      \item 1
      \item 2
      \item 3
      \item All of the above
      \item None of the above
    \end{enumerate}
	\end{minipage}
\end{frame}

%
% Slide 2
%
\begin{frame}[fragile]
	\frametitle{Poll Question: Constructing Conditionals}
  Does this segment of code evaluate to True or False in the end?
  \begin{lstlisting}[language=Python, autogobble,basicstyle=\tiny,numbers=none]
  y = list("hello")
  x = y
  y.append("!")
  print(y is x)
  \end{lstlisting}
  \vfill
  \begin{enumerate}[A]
    \item True
    \item False
    \item SyntaxError
    \item TypeError
  \end{enumerate}
\end{frame}

%
% Slide 2
%
\begin{frame}[fragile]
	\frametitle{Poll Question: Constructing Conditionals}
  Does this segment of code evaluate to True or False in the end?
  \begin{lstlisting}[language=Python, autogobble,basicstyle=\tiny,numbers=none]
  y = 5
  x = y
  x = x + 1
  print(y is x)
  \end{lstlisting}
  \vfill
  \begin{enumerate}[A]
    \item True
    \item False
    \item SyntaxError
    \item TypeError
  \end{enumerate}
\end{frame}

%
% Slide 2
%
\begin{frame}[fragile]
	\frametitle{Poll Question: Constructing Conditionals}
  What is the result of the following code?
  \begin{lstlisting}[language=Python, autogobble,basicstyle=\tiny,numbers=none]
  x = 5
  y = 5
  x += 1
  y += 1
  print(x is y)
  \end{lstlisting}
  \vfill
  \begin{enumerate}[A]
    \item True
    \item False
    \item SyntaxError
    \item TypeError
  \end{enumerate}
  \pause
  What is I replace 5 with 1000?
\end{frame}

%
% Slide 2
%
\begin{frame}[fragile]
	\frametitle{Poll Question: Constructing Conditionals}
  What is the result of the following code?
  \begin{lstlisting}[language=Python, autogobble,basicstyle=\tiny,numbers=none]
  x = None
  y = None
  print(y is x)
  \end{lstlisting}
  \vfill
  \begin{enumerate}[A]
    \item True
    \item False
    \item SyntaxError
    \item TypeError
  \end{enumerate}
\end{frame}

%
% Slide 2
%
\begin{frame}[fragile]
	\frametitle{Poll Question: Constructing Conditionals}
  What is the result of the following code?
  \begin{lstlisting}[language=Python, autogobble,basicstyle=\tiny,numbers=none]
  x = {"baz" : 2, "bar": 3}
  y = {"foo": 1}
  z = y
  y.update(x)
  print(y is z)
  \end{lstlisting}
  \vfill
  \begin{enumerate}[A]
    \item True
    \item False
    \item SyntaxError
    \item TypeError
  \end{enumerate}
\end{frame}

\section{Poll Questions: Correct Functions}

%
% Slide 2
%
\begin{frame}[fragile]
	\frametitle{Poll Question: Conditionals in Functions}
  Which of the following functions will ask the user for a sentence and return that sentence only if it is of even length?
	\vfill
	\begin{minipage}{0.32\textwidth}
		\begin{lstlisting}[language=Python, autogobble,basicstyle=\tiny,numbers=none]
    def get_even_sentence():
      x = input("Enter a string with an even number of letters: ")
      if x % 2 == 0:
        return x
		\end{lstlisting}
	\end{minipage}
	\begin{minipage}{0.32\textwidth}
		\begin{lstlisting}[language=Python, autogobble,basicstyle=\tiny,numbers=none]
    def get_even_sentence():
      x = input("Enter a string with an even number of letters: ")
      if len(x) % 2 == 0:
        return x
		\end{lstlisting}
	\end{minipage}
	\begin{minipage}{0.32\textwidth}
		\begin{lstlisting}[language=Python, autogobble,basicstyle=\tiny,numbers=none]
    def get_even_sentence():
      x = input("Enter a string with an even number of letters: ")
      if len(x) % 2 is 0:
        return x
		\end{lstlisting}
	\end{minipage}
	\vfill
	\begin{enumerate}[A]
    \item 1
    \item 2
    \item 3
    \item 2 and 3
    \item All of the above
	\end{enumerate}
\end{frame}

%
% Slide 2
%
\begin{frame}[fragile]
	\frametitle{Poll Question: Conditionals in Functions}
  Which fuction(s) return true if both elem1 and elem2 are in some\_list?
	\begin{minipage}{0.69\textwidth}
    1)
		\begin{lstlisting}[language=Python, autogobble,basicstyle=\tiny,numbers=none]
    def are_in_list(elem1, elem2, some_list):
      return elem1 and elem2 in some_list
		\end{lstlisting}
    \vspace{1cm}
    2)
		\begin{lstlisting}[language=Python, autogobble,basicstyle=\tiny,numbers=none]
    def are_in_list(elem1, elem2, some_list):
      return (elem1 in some_list) and (elem2 in some_list)
		\end{lstlisting}
    \vspace{1cm}
    3)
		\begin{lstlisting}[language=Python, autogobble,basicstyle=\tiny,numbers=none]
    def are_in_list(elem1, elem2, some_list):
      return (elem1 in some_list) or (elem2 in some_list)
		\end{lstlisting}
	\end{minipage}
	\begin{minipage}{0.29\textwidth}
    \begin{enumerate}[A]
      \item 1
      \item 2
      \item 3
      \item 2 and 3
      \item All of the above
    \end{enumerate}
	\end{minipage}
\end{frame}

%
% Slide 2
%
\begin{frame}[fragile]
	\frametitle{Poll Question: Conditionals in Functions}
  Which function will return the kth word in a list of words given any list of strings only if that kth word contains an even number of letters?
	\vfill
	\begin{minipage}{0.7\textwidth}
    1)
		\begin{lstlisting}[language=Python, autogobble,basicstyle=\tiny,numbers=none]
    def get_kth_word_if_even_len(k, words):
      if len(words) > k:
        if len(words[k]) % 2 == 0:
          return words[k]
		\end{lstlisting}
    2)
		\begin{lstlisting}[language=Python, autogobble,basicstyle=\tiny,numbers=none]
    def get_kth_word_if_even_len(k, words):
      if len(words[k]) % 2 == 0 and len(words) > k:
          return words[k]
		\end{lstlisting}
    3)
		\begin{lstlisting}[language=Python, autogobble,basicstyle=\tiny,numbers=none]
    def get_kth_word_if_even_len(k, words):
      if len(words) > k and len(words[k]) % 2 == 0:
          return words[k]
		\end{lstlisting}
	\end{minipage}
  \begin{minipage}{0.29\textwidth}
    \begin{enumerate}[A]
      \item 1
      \item 2
      \item 3
      \item 1 and 2
      \item Another option
    \end{enumerate}
	\end{minipage}
\end{frame}

%
% Slide 2
%
\begin{frame}[fragile]
	\frametitle{Poll Question: Conditionals in Functions}
  Which of the functions will correctly execute the task?
	\begin{minipage}{0.7\textwidth}
    1)
		\begin{lstlisting}[language=Python, autogobble,basicstyle=\tiny,numbers=none]
    def is_one_through_five(x):
      return 1 <= x <=5
		\end{lstlisting}
    2)
		\begin{lstlisting}[language=Python, autogobble,basicstyle=\tiny,numbers=none]
    def is_one_through_five(x):
      return x in [1, 2, 3, 4, 5]
		\end{lstlisting}
    3)
		\begin{lstlisting}[language=Python, autogobble,basicstyle=\tiny,numbers=none]
    def is_one_through_five(x):
      return x == (1 or 2 or 3 or 4 or 5)
		\end{lstlisting}
	\end{minipage}
  \begin{minipage}{0.29\textwidth}
    \begin{enumerate}[A]
      \item 1
      \item 2
      \item 3
      \item 1 and 2
      \item All of the above
    \end{enumerate}
	\end{minipage}
\end{frame}




%
% Slide 2
%
\begin{frame}[fragile]
  \frametitle{Poll Question: Nesting}
  What does the code on the right print?
  \vfill
  \begin{minipage}{0.49\textwidth}
    \begin{enumerate}[A]
      \item 2
      \item 5
      \item 8
      \item SyntaxError
    \end{enumerate}
  \end{minipage}
  \begin{minipage}{0.49\textwidth}
    \begin{lstlisting}[language=Python, autogobble]
    x = 2
    if x < 8:
      if x > 5:
        print(8)
      else:
        print(5)
    else:
      print(2)
    \end{lstlisting}
  \end{minipage}
\end{frame}

%
% Slide 2
%
\section{Code Blocks}
\begin{frame}[fragile]
  \frametitle{Code Blocks}

  \begin{itemize}
    \item Fancy term for defining a unit of execution.
  \end{itemize}
  \begin{lstlisting}[language=Python, autogobble]
  model = input('Enter car model: ')
  year = int(input('Enter year of car manufacture: '))

  antique = False
  domestic = False

  if year < 1970:
    antique = True

  if model in ['Ford', 'Chevrolet', 'Dodge']:
    domestic = True

  if antique:
    if domestic:
      print('My own model-T still runs like a charm...')
  \end{lstlisting}
\end{frame}

%
% Slide 2
%
\section{Conditional Expressions}
\begin{frame}[fragile]
  \frametitle{Conditional Expressions vs if-else}

  \begin{itemize}
    \item Follows this template: \lstinline|x if <cond> else y|
    \item Useful for item assignment where a condition must be met in order to avoid errors
    \item More concise
  \end{itemize}
  \vfill
  \begin{minipage}{0.49\textwidth}
    \begin{lstlisting}[language=Python, autogobble]
    x = input()
    if len(x) > 9:
      y = x[9]
    else:
      y = None
    \end{lstlisting}
  \end{minipage}
  \begin{minipage}{0.49\textwidth}
    \begin{lstlisting}[language=Python, autogobble]
    x = input()
    y = x[9] if len(x) > 9 else None
    \end{lstlisting}
  \end{minipage}
\end{frame}

%
% Slide 2
%
\begin{frame}[fragile]
  \frametitle{Poll Questions: Conditional Expressions}
  What is the value of \lstinline|x| after this code runs and the user attempts to enter the value 10?
  \begin{lstlisting}[language=Python, autogobble]
  x = "Odd" if int(input("Enter a number")) % 2 != 0 else "Even"
  \end{lstlisting}
  \vfill
  \begin{enumerate}[A]
    \item SyntaxError
    \item "Odd"
    \item "Even"
    \item This code contains another error
  \end{enumerate}
\end{frame}

%
% Slide 2
%
\begin{frame}[fragile]
  \frametitle{Poll Questions: Conditional Expressions}
  What is the the resulting value of \lstinline|x|?
  \begin{lstlisting}[language=Python, autogobble]
  y = 1.0
  x = "It's a float!" if type(y) == float
  \end{lstlisting}
  \vfill
  \begin{enumerate}[A]
    \item SyntaxError
    \item "It's a float!"
    \item None
    \item This code contains another error
  \end{enumerate}
\end{frame}



\end{document}
