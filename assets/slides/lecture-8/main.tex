\documentclass{beamer}

\usepackage{graphicx}
\usepackage{textpos}
\usepackage{listings}
\usepackage{lstautogobble}

\usetheme{Madrid}
\useoutertheme{miniframes} % Alternatively: miniframes, infolines, split

% Setup the university's color pallette
\definecolor{UIUCorange}{RGB}{19, 41, 75} % UBC Blue (primary)
\definecolor{UIUCblue}{RGB}{232, 74, 39} % UBC Grey (secondary)

\definecolor{codegreen}{rgb}{0,0.6,0}
\definecolor{codegray}{rgb}{0.5,0.5,0.5}
\definecolor{codepurple}{rgb}{0.58,0,0.82}
\definecolor{backcolour}{rgb}{0.95,0.95,0.92}

\lstdefinestyle{python}{
  backgroundcolor=\color{backcolour},   
  commentstyle=\color{codegreen},
  keywordstyle=\color{magenta},
  numberstyle=\tiny\color{codegray},
  stringstyle=\color{codepurple},
  basicstyle=\ttfamily\footnotesize,
  breakatwhitespace=false,         
  belowskip=-0.5em,
  breaklines=true,                 
  captionpos=b,                    
  keepspaces=true,                 
  numbers=left,                    
  numbersep=5pt,                  
  showspaces=false,                
  showstringspaces=false,
  showtabs=false,                  
  tabsize=2
}

\lstset{style=python}

\AtBeginSection[]{
    \begin{frame}
        \vfill
        \centering
        \begin{beamercolorbox}[sep=8pt,center,shadow=true,rounded=true]{title}
            \usebeamerfont{title}\insertsectionhead\par%
        \end{beamercolorbox}
        \vfill
    \end{frame}
}
% Setup the university's color pallette
\definecolor{UIUCorange}{RGB}{19, 41, 75} % UBC Blue (primary)
\definecolor{UIUCblue}{RGB}{232, 74, 39} % UBC Grey (secondary)


\setbeamercolor{palette primary}{bg=UIUCorange,fg=white}
\setbeamercolor{palette secondary}{bg=UIUCblue,fg=white}
\setbeamercolor{palette tertiary}{bg=UIUCblue,fg=white}
\setbeamercolor{palette quaternary}{bg=UIUCblue,fg=white}
\setbeamercolor{structure}{fg=UIUCorange} % itemize, enumerate, etc
\setbeamercolor{section in toc}{fg=UIUCblue} % TOC sections

\setbeamercolor{subsection in head/foot}{bg=UIUCorange,fg=UIUCblue}
\setbeamercolor{subsection in head/foot}{bg=UIUCorange,fg=UIUCblue}

\usepackage[utf8]{inputenc}
\usepackage{graphicx}

%Information to be included in the title page:
\title{\textbf{Functions}}
\author{\textbf{David H Smith IV}}
\institute[\textbf{UIUC}]{\textbf{University of Illinois Urbana-Champaign}}
\date{\textbf{Mon, Sept 28 2021}}

\setbeamertemplate{title page}[default][colsep=-4bp,rounded=true]
\addtobeamertemplate{title page}{\vspace{3\baselineskip}}{}
\addtobeamertemplate{title page}{
  \begin{textblock*}{\paperwidth}(-1.0em, -1.2em)
    \includegraphics[width=\paperwidth, height=\paperheight]{imgs/uiuc.png}
  \end{textblock*} 
}{}

\begin{document}

\frame{\titlepage}

%
% Slide 1
%
\section{Reminders}
\begin{frame}
  \frametitle{Announcements}
  \begin{itemize}
    \item \textbf{No post-reading or participation since we have a quiz.}
    \item Homework 7 is a review homework that is useful as practice for the quiz.
    \item Chromakey lab due on Sunday.
    \item Quiz 2 is Thursday.
  \end{itemize}
\end{frame}

\section{More on Functions}

%
% Slide 2
%
\begin{frame}[fragile]
  \frametitle{Dynamic Typing}
  Python is a dynamically typed language
  \begin{itemize}
    \item The \lstinline|+| does several different things:
      \begin{enumerate}
        \item Integer addition
        \item Floating point addition
        \item String concatenation
      \end{enumerate}
    \item At \textbf{runtime} Python looks at the \lstinline|+| operator and determines the correct behaviours based on the types or it's operands.
    \item \textbf{Polymorphism} \textrightarrow A single piece of code can do several things depending on the type of data it's working with.
      \begin{enumerate}
        \item You can write less code.
        \item Can be harder to find bugs.
      \end{enumerate}

  \end{itemize}
\end{frame}

%
% Slide 2
%
\begin{frame}[fragile]
  \frametitle{Poll Question: Polymorphic Functions}
  Which function produces an error?\\
  \vfill
  \begin{minipage}{0.69\textwidth}
    \begin{lstlisting}[language=Python, autogobble]
    def print_all(collection):
      for item in collection:
        print(item)

    print_all({ 'k': 'v', 'CS': '105' }) #A
    print_all(7)                         #B
    print_all('a string')                #C
    \end{lstlisting}
  \end{minipage}
  \hfill
  \begin{minipage}{0.29\textwidth}
    \begin{enumerate}[A]
      \item A error
      \item B error
      \item C error
      \item A \& B error
      \item A \& C error
      \item B \& C error
    \end{enumerate}
  \end{minipage}
\end{frame}


%
% Slide 2
%
\section{Scoping}
\begin{frame}[fragile]
  \frametitle{Poll Question: Function Scoping}
  \begin{minipage}{0.69\textwidth}
    What is produced by the following code?
    \begin{lstlisting}[language=Python, autogobble]
    my_var = 11
    def my_print(my_var):
      print(my_var)

    my_print(22)
    print(my_var)
    \end{lstlisting}
  \end{minipage}
  \hfill
  \begin{minipage}{0.29\textwidth}
    \begin{enumerate}[A]
      \item 11\\11
      \item 11\\22
      \item 22\\11
      \item 22\\22
      \item NameError
    \end{enumerate}
  \end{minipage}
\end{frame}

%
%
%
\begin{frame}[fragile]
  \frametitle{Poll Question: Function Scoping}
  \begin{minipage}{0.69\textwidth}
    What is produced by the following code?
    \begin{lstlisting}[language=Python, autogobble]
    my_var = 11
    def change_my_var():
      my_var = 12

    change_my_var()
    print(my_var)
    \end{lstlisting}
  \end{minipage}
  \hfill
  \begin{minipage}{0.29\textwidth}
    \begin{enumerate}[A]
      \item 11
      \item 12
      \item NameError
      \item None
    \end{enumerate}
  \end{minipage}
\end{frame}

%
%
%
\begin{frame}[fragile]
  \frametitle{Poll Question: Function Scoping}
  \begin{minipage}{0.69\textwidth}
    What is produced by the following code?
    \begin{lstlisting}[language=Python, autogobble]
    my_var = 11
    def print_my_var():
      print(my_var)

    change_my_var()
    print(print_my_var)
    \end{lstlisting}
  \end{minipage}
  \hfill
  \begin{minipage}{0.29\textwidth}
    \begin{enumerate}[A]
      \item 11
      \item 12
      \item NameError
      \item None
    \end{enumerate}
  \end{minipage}
  \pause
  \vfill
  \begin{enumerate}
    \item We have read access but not write access in the function's scope. 
    \item How do we get write access to the global scope from within a function?
  \end{enumerate}
\end{frame}


%
%
%
\begin{frame}[fragile]
  \frametitle{Poll Question: Function Scoping}
  What goes where the ?? is in order to (1) change the \textbf{global} value of my\_var and (2) such that the value 12 is printed to the screen when the code finishes running?
  \begin{lstlisting}[language=Python, autogobble]
    my_var = 11
    def print_my_var(new_my_var):
      ??

    change_my_var(int(input("Enter a new number: ")))
    print(print_my_var)
  \end{lstlisting}
  \vfill
  \textbf{For this activity, have one person connect their laptop to the monitor and work on a solution as a group.}
\end{frame}

%
% Slide 2
%
\begin{frame}[fragile]
  \frametitle{Scoping}
  \begin{itemize}
    \item Every function is given a clean slate.
    \item Any variables written in a function are defined in the function's scope.
    \item The scope is destroyed when the function returns.
    \item If a name is read that doesn't exist in the function's scope, it tries the scope the function was defined in.
  \end{itemize}
\end{frame}

\section{Docstrings}
%
% Slide 2
%
\begin{frame}[fragile]
  \frametitle{Docstrings!}
  \centering
  Off to Repl.it we go\ldots
\end{frame}

\section{Generator Functions}

%
% Slide 2
%
\begin{frame}[fragile]
  \frametitle{Generators}
  \begin{lstlisting}[language=Python, autogobble]
  def foo(x):
    while x > 0:
      yield x
      x -= 1
  foo_gen = foo(10)
  next(foo_gen)
  \end{lstlisting}
  \vfill
  \begin{enumerate}[A]
    \item Generator objects: \lstinline|yield| multiple values until they have finish running.
    \item Are defined like functions but are noteably different:
      \begin{enumerate}
        \item They \lstinline|yield| instead of return.
        \item You use \lstinline|next()| to get each successive yield.
        \item Perserve their internal state until they terminate.
        \item Throws StopIteration error if you try calling next() after the generator has already finished.
        \item Almost always involve iteration (i.e., at least one for or while loop).
      \end{enumerate}
    \item This is how the \lstinline|enumerate()| function is implemented.
  \end{enumerate}
\end{frame}



%
% Slide 2
%
\begin{frame}[fragile]
  \frametitle{Poll Question: Function Scoping}
  \begin{minipage}{0.69\textwidth}
    What is produced by the following code?
    \begin{lstlisting}[language=Python, autogobble]
    def foo(x):
      while x > 0:
        yield x
        x -= 2
    foo_gen = foo(10)
    a = next(foo_gen)
    b = next(foo_gen)
    c = next(foo_gen)
    print(a, b, c)
    \end{lstlisting}
  \end{minipage}
  \hfill
  \begin{minipage}{0.29\textwidth}
    \begin{enumerate}[A]
      \item \lstinline|10 8 6|
      \item \lstinline|10 10 10|
      \item \lstinline|10 8 6 4 2|
      \item This code contains an error
    \end{enumerate}
  \end{minipage}
\end{frame}

%
% Slide 2
%
\begin{frame}[fragile]
  \frametitle{Poll Question: Function Scoping}
  \begin{minipage}{0.6\textwidth}
    What is produced by the following code?
    \begin{lstlisting}[language=Python, autogobble]
    def foo(x):
      while x > 0:
        yield x
        x -= 2
    foo_gen = foo(10)
    foo_gen_list = list(foo_gen)
    print(foo_gen_list)
    \end{lstlisting}
  \end{minipage}
  \hfill
  \begin{minipage}{0.39\textwidth}
    \begin{enumerate}[A]
      \item \lstinline|[10, 8, 6, 4, 2]|
      \item \lstinline|[10, 8, 6, 4, 2, 0]|
      \item None
      \item StopIteration error
    \end{enumerate}
  \end{minipage}
  \pause
  \vfill
  Starting to look familiar?
\end{frame}

%
% Slide 2
%
\begin{frame}[fragile]
  \frametitle{Poll Question: Function Scoping}
  Fill in the ??? in this function\ldots
  \begin{lstlisting}[language=Python, autogobble]
    def my_enumerate(x):
      ???

    my_enumerate_gen = my_enumerate([1, 2, 3, 4])
    my_enumerate_list = list(my_enumerate_gen)
    print(my_enumerate_list)
  \end{lstlisting}
  \vfill
  So that the above code behaves like this code. 
  \begin{lstlisting}[language=Python, autogobble]
    enumerate_class = enumerate([1, 2, 3])
    foo_gen_list = list(enumerate_class)
    print(foo_gen_list)
  \end{lstlisting}
  \vfill
  Once you finish enumerate do the \lstinline|my_range()| function so that it replicates the behaviour of the builtin \lstinline|range_)| function.
\end{frame}


\section{General Loop Practice}

%
% Slide
%
\begin{frame}[fragile]
  \frametitle{Task: Validate User Input}
  \textbf{Problem Statement:} Create a function that gets 10 words that contain the letter "e", stores them in a list, then returns them. Note that this problems uses nested loops but not break or enumerate.
  \vfill
  \pause
  \begin{lstlisting}[language=Python, autogobble, basicstyle=\tiny]
  def no_e():
    l = []
    for i in range(0, 10):
      word = input("Enter a word with the letter e: ")
      while "e" not in word:
        word = input("Enter a word with the letter e: ")
      l.append(word)
    return l
  \end{lstlisting}
\end{frame}

%
% Slide
%
\begin{frame}[fragile]
  \frametitle{Task: Validate User Input}
  \textbf{Problem Statement:} Create a function that keeps asking the user for strings of an even length and adding them to a list until the user enters a string of an odd length. Then return the final list. You'll want to use a "while True:" loop here.
  \vfill
  \pause
  \begin{lstlisting}[language=Python, autogobble, basicstyle=\tiny]
  def get_even_words():
    l = []
    while True:
      user_in = input("Enter a word with an even number of vowels: ")
      if len(user_in) % 2 != 0:
        print("That word has an odd number of letters. Terminating!!")
        break
      l.append(user_in)
  \end{lstlisting}
\end{frame}

%
% Slide 1
%
\section{Reminders}
\begin{frame}
  \frametitle{Announcements}
  \begin{itemize}
    \item \textbf{No post-reading or participation since we have a quiz.}
    \item Homework 7 is a review homework that is useful as practice for the quiz.
    \item Chromakey lab due on Sunday.
    \item Quiz 2 is Thursday.
  \end{itemize}
\end{frame}

\end{document}
