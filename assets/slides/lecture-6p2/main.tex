\documentclass{beamer}

\usepackage{graphicx}
\usepackage{textpos}
\usepackage{listings}
\usepackage{lstautogobble}

\usetheme{Madrid}
\useoutertheme{miniframes} % Alternatively: miniframes, infolines, split

% Setup the university's color pallette
\definecolor{UIUCorange}{RGB}{19, 41, 75} % UBC Blue (primary)
\definecolor{UIUCblue}{RGB}{232, 74, 39} % UBC Grey (secondary)

\definecolor{codegreen}{rgb}{0,0.6,0}
\definecolor{codegray}{rgb}{0.5,0.5,0.5}
\definecolor{codepurple}{rgb}{0.58,0,0.82}
\definecolor{backcolour}{rgb}{0.95,0.95,0.92}

\lstdefinestyle{python}{
  backgroundcolor=\color{backcolour},   
  commentstyle=\color{codegreen},
  keywordstyle=\color{magenta},
  numberstyle=\tiny\color{codegray},
  stringstyle=\color{codepurple},
  basicstyle=\ttfamily\footnotesize,
  breakatwhitespace=false,         
  belowskip=-0.5em,
  breaklines=true,                 
  captionpos=b,                    
  keepspaces=true,                 
  numbers=left,                    
  numbersep=5pt,                  
  showspaces=false,                
  showstringspaces=false,
  showtabs=false,                  
  tabsize=2
}

\lstset{style=python}

\AtBeginSection[]{
    \begin{frame}
        \vfill
        \centering
        \begin{beamercolorbox}[sep=8pt,center,shadow=true,rounded=true]{title}
            \usebeamerfont{title}\insertsectionhead\par%
        \end{beamercolorbox}
        \vfill
    \end{frame}
}
% Setup the university's color pallette
\definecolor{UIUCorange}{RGB}{19, 41, 75} % UBC Blue (primary)
\definecolor{UIUCblue}{RGB}{232, 74, 39} % UBC Grey (secondary)


\setbeamercolor{palette primary}{bg=UIUCorange,fg=white}
\setbeamercolor{palette secondary}{bg=UIUCblue,fg=white}
\setbeamercolor{palette tertiary}{bg=UIUCblue,fg=white}
\setbeamercolor{palette quaternary}{bg=UIUCblue,fg=white}
\setbeamercolor{structure}{fg=UIUCorange} % itemize, enumerate, etc
\setbeamercolor{section in toc}{fg=UIUCblue} % TOC sections

\setbeamercolor{subsection in head/foot}{bg=UIUCorange,fg=UIUCblue}
\setbeamercolor{subsection in head/foot}{bg=UIUCorange,fg=UIUCblue}

\usepackage[utf8]{inputenc}
\usepackage{graphicx}


%Information to be included in the title page:
\title{\textbf{Topic 6: Loops}}
\author{\textbf{David H Smith IV}}
\institute[\textbf{UIUC}]{\textbf{University of Illinois Urbana-Champaign}}
\date{\textbf{Wed, Sep 22 2021}}

\setbeamertemplate{title page}[default][colsep=-4bp,rounded=true]
\addtobeamertemplate{title page}{\vspace{3\baselineskip}}{}
\addtobeamertemplate{title page}{
  \begin{textblock*}{\paperwidth}(-1.0em, -1.2em)
    \includegraphics[width=\paperwidth, height=\paperheight]{imgs/uiuc.png}
  \end{textblock*} 
}{}

\begin{document}

\frame{\titlepage}

\section{Enumerate}

%
% Slide 2
%
\begin{frame}[fragile]
  \frametitle{Poll Question: Unpacking}
  What are the values of foo?
  \begin{lstlisting}[language=Python, autogobble]
  foo, bar = (1, 2)
  \end{lstlisting}
  \vfill
  \begin{enumerate}[A]
    \item 1
    \item 2
    \item \lstinline|(1, 2)|
    \item Error
  \end{enumerate}
\end{frame}

%
% Slide 2
%
\begin{frame}[fragile]
  \frametitle{Poll Question: Unpacking}
  What is the value of z?
  \begin{lstlisting}[language=Python, autogobble]
  x, y, z = [22, [33, 44], [66]]
  \end{lstlisting}
  \vfill
  \begin{enumerate}[A]
    \item 22
    \item 33
    \item \lstinline|[33, 44]|
    \item \lstinline|[66]|
  \end{enumerate}
\end{frame}


%
% Slide 2
%
\begin{frame}[fragile]
  \frametitle{Poll Question: Enumerate}
  What is the value of \lstinline|x| after this code runs?
  \begin{lstlisting}[language=Python, autogobble]
  orig_list = ["I", "am", "Groot"]
  x = enumerate(orig_list)
  \end{lstlisting}
  \vfill
  \begin{enumerate}[A]
    \item Error
    \item \lstinline|[(1, "I"), (2, "am"), (3, "Groot")]|
    \item \lstinline|[(0, "I"), (1, "am"), (2, "Groot")]|
    \item \lstinline|[[0, "I"], [1, "am"], [2, "Groot"]]|
    \item Something else
  \end{enumerate}
  \pause
  Enumerate is like \lstinline|range()|. On it's own it's just an object that we can iterate over.
\end{frame}

%
% Slide 2
%
\begin{frame}[fragile]
  \frametitle{Poll Question: Enumerate}
  What is the value of \lstinline|y| after this code runs?
  \begin{lstlisting}[language=Python, autogobble]
  orig_list = ["I", "am", "Groot"]
  x = enumerate(orig_list)
  y = list(x)
  \end{lstlisting}
  \vfill
  \begin{enumerate}[A]
    \item Error
    \item \lstinline|[(1, "I"), (2, "am"), (3, "Groot")]|
    \item \lstinline|[(0, "I"), (1, "am"), (2, "Groot")]|
    \item \lstinline|[[0, "I"], [1, "am"], [2, "Groot"]]|
  \end{enumerate}
\end{frame}


%
% Slide 2
%
\begin{frame}[fragile]
  \frametitle{Poll Question: Enumerate}
  For this code, what is the varible \lstinline|type| of item at each iteration?
  \begin{lstlisting}[language=Python, autogobble]
  orig_list = [3, 7, 22, 90]
  for item in enumerate(orig_list):
    print(item)
  \end{lstlisting}
  \vfill
  \begin{enumerate}[A]
    \item \lstinline|tuple|
    \item \lstinline|list|
    \item \lstinline|int|
    \item This code has an error
  \end{enumerate}
\end{frame}


%
% Slide 2
%
\begin{frame}[fragile]
  \frametitle{Poll Question: Enumerate}
  \begin{minipage}{0.3\textwidth}
    \begin{lstlisting}[language=Python, autogobble, basicstyle=\tiny]
    for item in enumerate(x):
      print(item)
    \end{lstlisting}
  \end{minipage}
  \hfill
  +
  \hfill
  \begin{minipage}{0.2\textwidth}
    \begin{lstlisting}[language=Python, autogobble, basicstyle=\tiny]
    i, val = (0, 2)
    \end{lstlisting}
  \end{minipage}
  \hfill
  =
  \hfill
  \begin{minipage}{0.32\textwidth}
    \begin{lstlisting}[language=Python, autogobble, basicstyle=\tiny]
    for i, val in enumerate(x):
      print(i, val)
    \end{lstlisting}
  \end{minipage}
\end{frame}

%
% Slide 2
%
\begin{frame}[fragile]
  \frametitle{Poll Question: Enumerate}
  What is the contents of new list?
  \begin{lstlisting}[language=Python, autogobble]
  orig_list = [3, 7, 22, 90]
  new_list = []
  for index, value in enumerate(orig_list):
    if (index % 2) == 0:
      new_list.append(value)
  \end{lstlisting}
  \vfill
  \begin{enumerate}[A]
    \item \lstinline|[3, 7]|
    \item \lstinline|[3, 22]|
    \item \lstinline|[3, 7, 22, 90]|
    \item \lstinline|[7, 90]|
  \end{enumerate}
\end{frame}

%
% Slide 2
%
\begin{frame}[fragile]
  \frametitle{Why?!?}
  \textbf{Why would we ever want to use this enumerate?}
\end{frame}


%
% Slide 2
%
\begin{frame}[fragile]
  \frametitle{Consider the Following}
  I want to replace all even numbers in a list with the word "Even" and odd numbers with the word "Odd". Will this code do it?
  \begin{lstlisting}[language=Python, autogobble]
  x = [1, 2, 3, 4]
  for item in x:
    item = "Even" if item % 2 == 0 else "Odd"
  \end{lstlisting}
  \vfill
  \begin{enumerate}[A]
    \item Yes :D
    \item No D:
  \end{enumerate}
  \pause
  How could we accomplish this task?
\end{frame}


%
% Slide 2
%
\begin{frame}[fragile]
  \frametitle{Consider the Following}
  Will this work?
  \begin{lstlisting}[language=Python, autogobble]
  x = [1, 2, 3, 4]
  for i, item in enumerate(x):
    x[i] = "Even" if item % 2 == 0 else "Odd"
  \end{lstlisting}
\end{frame}


%
% Slide 2
%
\begin{frame}[fragile]
  \frametitle{Now Side by Side}
  1) Doesn't update the list because \lstinline|item| is a different variable that simply references a value in the list. Setting it equal to a new value only updates the value it is references and doesn't change the original list.
  \begin{lstlisting}[language=Python, autogobble]
  x = [1, 2, 3, 4]
  for item in x:
    item = "Even" if item % 2 == 0 else "Odd"
  \end{lstlisting}
  \vfill
  2) This directly references the list via the index \lstinline|i| and updates the value in the list.
  \begin{lstlisting}[language=Python, autogobble]
  x = [1, 2, 3, 4]
  for i, item in enumerate(x):
    x[i] = "Even" if item % 2 == 0 else "Odd"
  \end{lstlisting}
\end{frame}

\section{Break vs Continue}
%
% Slide 2
%
\begin{frame}[fragile]
  \frametitle{Poll Question: Break}
  How many chars are printed?
  \begin{lstlisting}[language=Python, autogobble]
  for c in "sleepy":
    if c == "e":
      break
    print(c)
  \end{lstlisting}
  \vfill
  \begin{enumerate}[A]
    \item 4
    \item 1
    \item 2
    \item 6
  \end{enumerate}
\end{frame}

%
% Slide 2
%
\begin{frame}[fragile]
  \frametitle{Poll Question: Break vs Return}
  On which inputs do these functions behave differently?
  \centering
  \begin{minipage}{0.45\textwidth}
    \begin{lstlisting}[language=Python, autogobble]
    def func(a_list):
      for item in a_list:
        if item == "":
          break
        print(item)
      print("done")
    \end{lstlisting}
  \end{minipage}
  \hfill
  \begin{minipage}{0.45\textwidth}
    \begin{lstlisting}[language=Python, autogobble]
    def func(a_list):
      for item in a_list:
        if item == "":
          return 
        print(item)
      print("done")
    \end{lstlisting}
  \end{minipage}
  \vfill
  \begin{enumerate}[A]
    \item \lstinline|["a", "b", "", "d"]|
    \item \lstinline|["a", "b", "c", ""]|
    \item both
    \item neither
  \end{enumerate}
\end{frame}

%
% Slide 2
%
\begin{frame}[fragile]
  \frametitle{Poll Question: Continue}
  How many items are printed?
  \begin{lstlisting}[language=Python, autogobble]
  mixed_list = ['hi', '3', math.pi, 'there', ['CS', 437]]
  for item in mixed_list:
    if type(item) != str:
      continue
    print(item)
  \end{lstlisting}
  \vfill
  \begin{enumerate}[A]
    \item 0
    \item 3
    \item 2
    \item 6
  \end{enumerate}
\end{frame}

%
% Slide
%
\begin{frame}[fragile]
  \frametitle{Break vs Return vs Continue}
  \begin{itemize}
    \item \textbf{continue} \textrightarrow Skips everything below it and goes back to the beginning of the loop.
    \item \textbf{return} \textrightarrow Leaves function with return value.
    \item \textbf{break} \textrightarrow Exits loop it is apart of.
    \end{itemize}
\end{frame}

%
% Slide
%
\begin{frame}[fragile]
  \frametitle{When would we want to use these?}
  \begin{enumerate}
    \item When would we want to use continue?
      \pause
      \begin{enumerate}
        \item Concatenating all strings in a list of things.
        \item Summing all integers/floats in a list of things.
        \item Validating user input.
      \end{enumerate}
      \pause
    \item When would we want to use break?
      \pause
      \begin{enumerate}
        \item Searching for the occurance of an item that meats a condition in the list.
        \item Exiting an otherwise infinite loop when the user wants to exit.
      \end{enumerate}
  \end{enumerate}
\end{frame}

%
% Slide 2
%
\begin{frame}[fragile]
  \frametitle{Poll Question: Summing Nums in List}
  What should we replace the question marks with?
  \begin{lstlisting}[language=Python, autogobble]
  def sum_nums(x):
    s = 0
    for item in x:
      if type(item) != int or float:
        ???
      s += item
  \end{lstlisting}
  \vfill
  \begin{enumerate}[A]
    \item break
    \item continue
    \item There's another issue with this code
  \end{enumerate}
  \pause
  \vfill
  What should we replace this line of code with?
\end{frame}

%
% Slide 2
%
\begin{frame}[fragile]
  \frametitle{Poll Question: Summing Nums in List}
  What should we replace the question marks with?
  \begin{lstlisting}[language=Python, autogobble]
  def sum_nums(x):
    s = 0
    for item in x:
      if type(item) != int or type(item) != float:
        ???
      s += item
  \end{lstlisting}
  \vfill
  \begin{enumerate}[A]
    \item break
    \item continue
    \item There's another issue with this code
  \end{enumerate}
  \pause
  \vfill
  \textbf{Key Takeaway:} Break leaves the loop. Continue skips to the next iteration. 
\end{frame}



\section{General Loop Practice}

%
% Slide
%
\begin{frame}[fragile]
  \frametitle{Task: Validate User Input}
  \textbf{Problem Statement:} Create a function that gets 10 words that contain the letter "e", stores them in a list, then returns them. Note that this problems uses nested loops but not break or enumerate.
  \vfill
  \pause
  \begin{lstlisting}[language=Python, autogobble]
  def no_e():
    l = []
    for i in range(0, 10):
      word = input("Enter a word with the letter e: ")
      while "e" not in word:
        word = input("Enter a word with the letter e: ")
      l.append(word)
    return l
  \end{lstlisting}
\end{frame}

%
% Slide
%
\begin{frame}[fragile]
  \frametitle{Task: Validate User Input}
  \textbf{Problem Statement:} Create a function that keeps asking the user for strings of an even length and adding them to a list until the user enters a string of an odd length. Then return the final list. You'll want to use a "while True:" loop here.
  \vfill
  \pause
  \begin{lstlisting}[language=Python, autogobble]
  def get_even_words():
    l = []
    while True:
      user_in = input("Enter a word with an even number of vowels: ")
      if len(user_in) % 2 != 0:
        print("That word has an odd number of letters. Terminating!!")
        break
      l.append(user_in)
  \end{lstlisting}
\end{frame}


\end{document}
