\documentclass{beamer}

\usepackage{graphicx}
\usepackage{textpos}
\usepackage{listings}
\usepackage{lstautogobble}

\usetheme{Madrid}
\useoutertheme{miniframes} % Alternatively: miniframes, infolines, split

% Setup the university's color pallette
\definecolor{UIUCorange}{RGB}{19, 41, 75} % UBC Blue (primary)
\definecolor{UIUCblue}{RGB}{232, 74, 39} % UBC Grey (secondary)

\definecolor{codegreen}{rgb}{0,0.6,0}
\definecolor{codegray}{rgb}{0.5,0.5,0.5}
\definecolor{codepurple}{rgb}{0.58,0,0.82}
\definecolor{backcolour}{rgb}{0.95,0.95,0.92}

\lstdefinestyle{python}{
    backgroundcolor=\color{backcolour},   
    commentstyle=\color{codegreen},
    keywordstyle=\color{magenta},
    numberstyle=\tiny\color{codegray},
    stringstyle=\color{codepurple},
    basicstyle=\ttfamily\footnotesize,
    breakatwhitespace=false,         
    belowskip=-0.5em,
    breaklines=true,                 
    captionpos=b,                    
    keepspaces=true,                 
    numbers=left,                    
    numbersep=5pt,                  
    showspaces=false,                
    showstringspaces=false,
    showtabs=false,                  
    tabsize=2
}

\lstset{style=python}

\AtBeginSection[]{
    \begin{frame}
        \vfill
        \centering
        \begin{beamercolorbox}[sep=8pt,center,shadow=true,rounded=true]{title}
            \usebeamerfont{title}\insertsectionhead\par%
        \end{beamercolorbox}
        \vfill
    \end{frame}
}
% Setup the university's color pallette
\definecolor{UIUCorange}{RGB}{19, 41, 75} % UBC Blue (primary)
\definecolor{UIUCblue}{RGB}{232, 74, 39} % UBC Grey (secondary)


\setbeamercolor{palette primary}{bg=UIUCorange,fg=white}
\setbeamercolor{palette secondary}{bg=UIUCblue,fg=white}
\setbeamercolor{palette tertiary}{bg=UIUCblue,fg=white}
\setbeamercolor{palette quaternary}{bg=UIUCblue,fg=white}
\setbeamercolor{structure}{fg=UIUCorange} % itemize, enumerate, etc
\setbeamercolor{section in toc}{fg=UIUCblue} % TOC sections

\setbeamercolor{subsection in head/foot}{bg=UIUCorange,fg=UIUCblue}
\setbeamercolor{subsection in head/foot}{bg=UIUCorange,fg=UIUCblue}

\usepackage[utf8]{inputenc}
\usepackage{graphicx}

%Information to be included in the title page:
\title{\textbf{HTML}}
\author{\textbf{David H Smith IV}}
\institute[\textbf{UIUC}]{\textbf{University of Illinois Urbana-Champaign}}
\date{\textbf{Mon, July 21 2021}}

\setbeamertemplate{title page}[default][colsep=-4bp,rounded=true]
\addtobeamertemplate{title page}{\vspace{3\baselineskip}}{}
\addtobeamertemplate{title page}{
    \begin{textblock*}{\paperwidth}(-1.0em, -1.2em)
        \includegraphics[width=\paperwidth, height=\paperheight]{imgs/uiuc.png}
    \end{textblock*} 
}{}

\begin{document}

\frame{\titlepage}

\section{Reminders}

%
% Slide 1
%
\begin{frame}
    \frametitle{Reminders}
    \begin{itemize}
        \item Homework 12 is due Friday
        \item Game of Life is due Friday
        \item Topic 12 participation is due tommorow
        \item Topic 12 post-reading is due tommorowa
    \end{itemize}
\end{frame}

\section{Basics of Website Construction}
%
% Slide 2
%
\begin{frame}[fragile]
    \frametitle{HTML5 = HTML, CSS, JS}
    \pause
    \begin{itemize}
        \item \textbf{Separation of Concerns: }
            \begin{itemize}
                \item HTML = content
                \item CSS = styling
                \item JS = interactivity
            \end{itemize}
            \pause
        \item \textbf{HTML Documents are Heirarchical: }
            \begin{itemize}
                \item Elements have begin/end tags (e.g., \lstinline|<body></body>|)
                \item Elements can be nested in other elements
                \item Newlines and indentation in HTML are ignored.
            \end{itemize}
            \pause
        \item \textbf{CSS consists of:}
            \begin{itemize}
                \item \{attribute : value\} pairs
                \item Like a Python dictionary
            \end{itemize}
            \pause
        \item \textbf{JS}
            \begin{itemize}
                \item variables, expressions, function, conditionals, loops, etc\ldots
            \end{itemize}
    \end{itemize}
\end{frame}
%
% Slide 2
%
\begin{frame}[fragile]
    \frametitle{Generating Web pages Dynamically}
    \begin{enumerate}[A]
        \item Amazon.com doesn't have people write web pages for each product. They're generated on the fly by a computer program according to a template.
        \item \textbf{Templates: } Basically like Python format strings, like \lstinline|"Product: {} Price: {}".format(productname, product[productname])|.
        \item Looping through collections
    \end{enumerate}
\end{frame}

\section{HTML}
%
% Slides
%
\begin{frame}[fragile]
    \frametitle{The Starting Template}
    \begin{lstlisting}[language=html,autogobble]
    <!DOCTYPE html>
    <html>
        <body>
            <h1>This is the header!</h1>
            <p> And this is my first paragraph :D </p>
        </body>
    </html>
    \end{lstlisting} 
    \vfill
    \begin{enumerate}
        \item \lstinline|<!DOCTYPE html>| \textrightarrow This defines the type of document we are making (html5) so the browser knows how to interpret it.
        \item \lstinline|<html>| \textrightarrow Defines the bounds of the HTML document.
        \item \lstinline|<body>| \textrightarrow Defines the visible portion of the html document.
        \item Headers:
            \begin{itemize}
                \item \lstinline|<h1>| \textrightarrow Largest header
                \item \lstinline|<h2>| \textrightarrow Second largest header
                \item \lstinline|<h3>| \textrightarrow You get the idea...
            \end{itemize}
        \item \lstinline|<p>| \textrightarrow Encapsulates a paragraph and formats the text as just plain text
    \end{enumerate}
\end{frame}


%
% Slides
%
\begin{frame}[fragile]
    \frametitle{Ordered vs Unordered Lists}

    \begin{minipage}{0.49\textwidth}
        Unordered list:
        \begin{lstlisting}[language=html,autogobble]
          <ul>
            <li>CS 105</li>
            <li>CS 125</li>
          </ul>
        \end{lstlisting} 
    \end{minipage}
    \begin{minipage}{0.49\textwidth}
        Ordered list:
        \begin{lstlisting}[language=html,autogobble]
          <ol>
            <li>CS 105</li>
            <li>CS 125</li>
          </ol>
        \end{lstlisting} 
    \end{minipage}
    \vfill
    \begin{enumerate}
        \item \lstinline|<ul></ul>| \textrightarrow Encapsulates the items and renders them as an unordered list.
        \item \lstinline|<ol></ol>| \textrightarrow Encapsulates the items and renders them as an ordered list.
        \item \lstinline|<li></li>| \textrightarrow Goes inside either the \lstinline|<ul></ul>| or \lstinline|<ol></ul>| and encapsulates a list item.
    \end{enumerate}
\end{frame}

%
% Slides
%
\begin{frame}[fragile]
    \frametitle{Tables}
    What are the visible column header(s) in the table produced by this HTML?\\
    \hfill
    \begin{minipage}{0.49\textwidth}
        \begin{lstlisting}[language=html]
<table>
  <caption>CS 105</caption>
  <tr>
    <th></th>
    <th>Assigned</th>
    <th>Due</th>
  </tr>
  <tr>
    <th>Homework 1</th>
    <td>06/14/2021</td>
    <td>06/21/2021</td>
  </tr>
</table>
        \end{lstlisting} 
    \end{minipage}
    \hfill
    \begin{minipage}{0.45\textwidth}
        {\footnotesize
            \begin{enumerate}[A]
                \item \lstinline|<table></table>| \textrightarrow Defines a new table and encapsulates every item below
                \item \lstinline|<caption></caption>| \textrightarrow Defines a caption for the table
                \item \lstinline|<tr></tr>| \textrightarrow Defines a new table row and encapsulates a series of \lstinline|td|.
                \item \lstinline|<td></td>| \textrightarrow Defines an entry in the table row.
                \item \lstinline|<th></th>| \textrightarrow Same thing as td but bold to indicate it's a header.
        \end{enumerate}}
    \end{minipage}
\end{frame}

%
% Slides
%
\begin{frame}[fragile]
    \frametitle{Poll Question: HTML}
    How many rows are produced by this table?\\
    \begin{minipage}{0.49\textwidth}
        \begin{lstlisting}[language=html, basicstyle=\scriptsize]
<table>
  <tr>
    <th>Assigned</th>
    <th>Due</th>
  </tr>
  <tr>
    <td>06/14/2021</td>
    <td>06/21/2021</td>
  </tr>
  <tr>
    <td>07/04/2021</td>
    <td>07/14/2021</td>
  </tr>
  <tr>
    <td>07/04/2021</td>
    <td>07/14/2021</td>
  </tr>
</table>
        \end{lstlisting} 
    \end{minipage}
    \hfill
    \begin{minipage}{0.39\textwidth}
        \begin{enumerate}[A]
            \item 4
            \item 3
            \item 2
            \item 1
        \end{enumerate}
    \end{minipage}
\end{frame}

%
% Slides
%
\begin{frame}[fragile]
    \frametitle{Text with Links}
    \begin{lstlisting}[language=html, basicstyle=\scriptsize]
<a href="https://hamiltonfour.tech/#/cv">CV</a>
<a href="https://hamiltonfour.tech/#/">About</a>
<a href="https://hamiltonfour.tech/#/publications">Publications</a>
    \end{lstlisting} 
    \hfill
    \begin{enumerate}[A]
        \item \lstinline|<a></a>| \textrightarrow Encapsulates the text we wish to include the hyperlink.
        \item \lstinline|href=""| \textrightarrow The link to either an external link to a page on another website (e.g., the examples above) or another internal page on the website. 
    \end{enumerate}
\end{frame}

%
% Slides
%
\begin{frame}[fragile]
    Given this directory structure and this sample HTML, which document do you think this HTML is in?\\
    \begin{minipage}{0.68\textwidth}
    \begin{lstlisting}[autogobble]
    dev_website
    |---index.html
    |___elements
        |--- about.html
        |--- cv.html
        |___ projects.html
        \end{lstlisting} 
        \hfill
        \begin{lstlisting}[language=html,autogobble, basicstyle=\scriptsize]
    <a href="elements/about.html">About</a>
    <a href="elements/cv.html">CV</a>
    <a href="elements/projects.html">Projects</a>
        \end{lstlisting} 
    \end{minipage}
    \hfill
    \begin{minipage}{0.25\textwidth}
        \begin{enumerate}[A]
            \item index.html
            \item about.html
            \item cv.html
            \item projects.html
        \end{enumerate}
    \end{minipage}
\end{frame}


%
% Slides
%
\begin{frame}[fragile]
    \frametitle{HTML Images}
    \begin{lstlisting}[language=html, basicstyle=\scriptsize]
  <img src="img.jpg" alt="This is an image">
    \end{lstlisting} 
    \vfill
    \begin{enumerate}[A]
        \item \lstinline|src=""| Link to that image either locally or online.
        \item \lstinline|alt=""| Text to display instead of the image incase a browser does not support that image or the image link becomes broken.
        \item One of the only html tags that is self contained and therefore does not have an end tag.
    \end{enumerate}
\end{frame}


%
% Slide
%
\begin{frame}[fragile]
    \frametitle{The div tag}
    \begin{lstlisting}[language=html, basicstyle=\scriptsize, autogobble]
    <div class=""> 
    ...
    <div>
    \end{lstlisting} 
    \vfill
    \begin{enumerate}[A]
        \item Encapsulates an arbitrary amount of other html tags
        \item Defines how the stuff it encapsulates is rendered using class + css
    \end{enumerate}
\end{frame}

\section{CSS}
%
% Slides
%
\begin{frame}[fragile]
    \frametitle{CSS Attributes for Tag Types}
    \begin{minipage}{0.44\textwidth}
        \begin{lstlisting}[language=python, basicstyle=\scriptsize, autogobble]
    body{
        background-color: orange;
    }

    h1{
        color: navy;
        text-align: center;
        font-size: 50;
    }

    p{
        color: blue;
    }

    li{
        font-variant: small-caps;
    }
        \end{lstlisting} 
    \end{minipage}
    \hfill
    \begin{minipage}{0.44\textwidth}
        {\footnotesize
        \begin{enumerate}[A]
            \item A file containing attributes wrapped in curly brackets.
            \item Attributes change how the tag they're attached to get rendered on the webpage.
            \item Each attribute value pair ends with a \lstinline|;|.
            \item There's a lot of attributes you can change. These are just a few...
        \end{enumerate}}
    \end{minipage}
\end{frame}

%
% Slides
%
\begin{frame}[fragile]
    \frametitle{Defining your own classes}
    \begin{minipage}{0.44\textwidth}
        \begin{lstlisting}[language=css, basicstyle=\scriptsize, autogobble]
        .myclass{
            background-color: orange;
            padding: 10px;
            font-size: 10px;
        }
        \end{lstlisting} 
    \end{minipage}
    \hfill
    \begin{minipage}{0.44\textwidth}
        \begin{enumerate}[A]
            \item Starts with a .
            \item Otherwise identical to the previous example
        \end{enumerate}
    \end{minipage}
\end{frame}

%
% Slides
%
\begin{frame}[fragile]
    \frametitle{div + css classes = unlimited power}
    \begin{minipage}{0.48\textwidth}
    \begin{lstlisting}[language=html, basicstyle=\tiny, autogobble]
        <div class="myclass">
            <p> This is a paragraph </p>
            <p> This is another paragraph </p>
            <ul>
                <li> I </li>
                <li> am </li>
                <li> Groot </li>
            </ul>
        </div>
    \end{lstlisting}
    \end{minipage}
    \hfill
    \begin{minipage}{0.44\textwidth}
    \begin{lstlisting}[language=html, basicstyle=\tiny, autogobble]
        .myclass{
            background-color: orange;
            padding: 10px;
            font-size: 10px;
        }
    \end{lstlisting} 
    \end{minipage}
    \vfill
    \begin{enumerate}[A]
        \item A file containing attributes wrapped in curly brackets.
        \item Attributes change how the tag they're attached to get rendered on the webpage.
        \item Each attribute value pair ends with a \lstinline|;|
        \item There's a lot of attributes you can change. These are just a few...
    \end{enumerate}
\end{frame}


\section{Next bat time. Next bat channel.}
%
% Slide 2
%
\begin{frame}[fragile]
    \frametitle{Reading data from the internet}
    \begin{lstlisting}[language=Python, autogoggle]
    import requests

    response = requests.get("https://www.google.com")
    \end{lstlisting} 
    \vfill
    \begin{enumerate}[A]
        \item From a Python program
        \item resquests module: Given a URL, returns the document at that URL
    \end{enumerate}
\end{frame}

\end{document}
