\documentclass{beamer}

\usepackage{graphicx}
\usepackage{textpos}
\usepackage{listings}
\usepackage{lstautogobble}

\usetheme{Madrid}
\useoutertheme{miniframes} % Alternatively: miniframes, infolines, split

% Setup the university's color pallette
\definecolor{UIUCorange}{RGB}{19, 41, 75} % UBC Blue (primary)
\definecolor{UIUCblue}{RGB}{232, 74, 39} % UBC Grey (secondary)

\definecolor{codegreen}{rgb}{0,0.6,0}
\definecolor{codegray}{rgb}{0.5,0.5,0.5}
\definecolor{codepurple}{rgb}{0.58,0,0.82}
\definecolor{backcolour}{rgb}{0.95,0.95,0.92}

\lstdefinestyle{python}{
  backgroundcolor=\color{backcolour},   
  commentstyle=\color{codegreen},
  keywordstyle=\color{magenta},
  numberstyle=\tiny\color{codegray},
  stringstyle=\color{codepurple},
  basicstyle=\ttfamily\footnotesize,
  breakatwhitespace=false,         
  belowskip=-0.5em,
  breaklines=true,                 
  captionpos=b,                    
  keepspaces=true,                 
  numbers=left,                    
  numbersep=5pt,                  
  showspaces=false,                
  showstringspaces=false,
  showtabs=false,                  
  tabsize=2
}

\lstset{style=python}

\AtBeginSection[]{
    \begin{frame}
        \vfill
        \centering
        \begin{beamercolorbox}[sep=8pt,center,shadow=true,rounded=true]{title}
            \usebeamerfont{title}\insertsectionhead\par%
        \end{beamercolorbox}
        \vfill
    \end{frame}
}
% Setup the university's color pallette
\definecolor{UIUCorange}{RGB}{19, 41, 75} % UBC Blue (primary)
\definecolor{UIUCblue}{RGB}{232, 74, 39} % UBC Grey (secondary)


\setbeamercolor{palette primary}{bg=UIUCorange,fg=white}
\setbeamercolor{palette secondary}{bg=UIUCblue,fg=white}
\setbeamercolor{palette tertiary}{bg=UIUCblue,fg=white}
\setbeamercolor{palette quaternary}{bg=UIUCblue,fg=white}
\setbeamercolor{structure}{fg=UIUCorange} % itemize, enumerate, etc
\setbeamercolor{section in toc}{fg=UIUCblue} % TOC sections

\setbeamercolor{subsection in head/foot}{bg=UIUCorange,fg=UIUCblue}
\setbeamercolor{subsection in head/foot}{bg=UIUCorange,fg=UIUCblue}

\usepackage[utf8]{inputenc}
\usepackage{graphicx}

%Information to be included in the title page:
\title{\textbf{HTML}}
\author{\textbf{David H Smith IV}}
\institute[\textbf{UIUC}]{\textbf{University of Illinois Urbana-Champaign}}
\date{\textbf{Mon, July 21 2021}}

\setbeamertemplate{title page}[default][colsep=-4bp,rounded=true]
\addtobeamertemplate{title page}{\vspace{3\baselineskip}}{}
\addtobeamertemplate{title page}{
  \begin{textblock*}{\paperwidth}(-1.0em, -1.2em)
    \includegraphics[width=\paperwidth, height=\paperheight]{imgs/uiuc.png}
  \end{textblock*} 
}{}

\begin{document}

\frame{\titlepage}

\section{Reminders}

%
% Slide 1
%
\begin{frame}
  \frametitle{Reminders}
  \begin{itemize}
    \item Please remember to sign-up or submit a conflict request for the quiz
    \item Practice quiz is up, be sure to attempt it before the actual quiz
    \item Homework 7 is due Friday 
    \item The grace period for homework 6 ends Friday
    \item The grace period for homework 7 will only be until Aug. 5
  \end{itemize}
\end{frame}

\section{Hyper Text Markup Language}

%
% Slide 2
%
\begin{frame}[fragile]
  \frametitle{HTML5 = HTML, CSS, JS}
  \pause
  \begin{itemize}
    \item \textbf{Separation of Concerns: }
    \begin{itemize}
      \item HTML = content
      \item CSS = styling
      \item JS = interactivity
    \end{itemize}
    \pause
    \item \textbf{HTML Documents are Heirarchical: }
    \begin{itemize}
      \item Elements have begin/end tags
      \item Elements can be nested in other elements
    \end{itemize}
    \pause
    \item \textbf{CSS consists of:}
    \begin{itemize}
      \item \{attribute : value\} pairs
      \item Like a Python dictionary
    \end{itemize}
    \pause
    \item \textbf{JS}
    \begin{itemize}
      \item variables, expressions, function, conditionals, loops, etc\ldots
    \end{itemize}
  \end{itemize}
\end{frame}
%
% Slide 2
%
\begin{frame}[fragile]
  \frametitle{Generating Web pages Dynamically}
  \begin{enumerate}[A]
    \item Amazon.com doesn't have people write web pages for each product. They're generated on the fly by a computer program according to a template.
    \item \textbf{Templates: } Basically like Python format strings, like \lstinline|"Product: {} Price: {}".format(productname, product[productname])|.
    \item Looping through collections
  \end{enumerate}
\end{frame}

%
% Slides
%
\begin{frame}[fragile]
  \frametitle{Poll Question: HTML}
  What will the following code segment produce?\\
  \begin{lstlisting}[language=html,autogobble]
  <ul>
    <li>CS 105<li>
    <li>CS 125<li>
  </ul>
  \end{lstlisting} 
  \vfill
  \begin{enumerate}[A]
    \item Unordered list with CS 105 and CS 125
    \item Ordered list with CS 105 and CS 125
    \item Error
  \end{enumerate}
\end{frame}

%
% Slides
%
\begin{frame}[fragile]
  \frametitle{Poll Question: HTML}
  What are the visible column header(s) in the table produced by this HTML?\\
  \begin{minipage}{0.49\textwidth}
    \begin{lstlisting}[language=html]
<table>
  <caption>CS 105</caption>
  <tr>
    <th></th>
    <th>Assigned</th>
    <th>Due</th>
  </tr>
  <tr>
    <th>Homework 1</th>
    <td>06/14/2021</td>
    <td>06/21/2021</td>
  </tr>
</table>
    \end{lstlisting} 
  \end{minipage}
  \hfill
  \begin{minipage}{0.45\textwidth}
    \begin{enumerate}[A]
      \item Assigned
      \item Assigned \& Due
      \item Homework 1
      \item 06/14/2021 \& 06/21/2021
    \end{enumerate}
  \end{minipage}
\end{frame}

%
% Slides
%
\begin{frame}[fragile]
  \frametitle{Poll Question: HTML}
  How many rows are produced by this table?\\
  \begin{minipage}{0.49\textwidth}
  \begin{lstlisting}[language=html, basicstyle=\scriptsize]
<table>
  <tr>
    <th>Assigned</th>
    <th>Due</th>
  </tr>
  <tr>
    <td>06/14/2021</td>
    <td>06/21/2021</td>
  </tr>
  <tr>
    <td>07/04/2021</td>
    <td>07/14/2021</td>
  </tr>
  <tr>
    <td>07/04/2021</td>
    <td>07/14/2021</td>
  </tr>
</table>
    \end{lstlisting} 
  \end{minipage}
  \hfill
  \begin{minipage}{0.39\textwidth}
    \begin{enumerate}[A]
      \item 4
      \item 3
      \item 2
      \item 1
    \end{enumerate}
  \end{minipage}
\end{frame}

%
% Slides
%
\begin{frame}[fragile]
  Given this directory structure and this sample HTML, which document do you think this HTML is in?\\
  \begin{minipage}{0.65\textwidth}
    \begin{lstlisting}
    dev_website
    |---index.html
    |___elements
        |--- about.html
        |--- cv.html
        |___ projects.html
    \end{lstlisting} 
    \hfill
    \begin{lstlisting}[language=html, basicstyle=\scriptsize]
    <a href="elements/about.html">About</a>
    <a href="elements/cv.html">CV</a>
    <a href="elements/projects.html">Projects</a>
    \end{lstlisting} 
  \end{minipage}
  \hfill
  \begin{minipage}{0.3\textwidth}
    \begin{enumerate}[A]
      \item index.html
      \item about.html
      \item cv.html
      \item projects.html
    \end{enumerate}
  \end{minipage}
\end{frame}
%
% Slides
%
\begin{frame}[fragile]
  \begin{minipage}{0.49\textwidth}
  What is produced on a webpage assuming the image referenced by href is broken or missing?
  \begin{lstlisting}[language=html, basicstyle=\scriptsize]
  <img src="img.jpg" alt="This is an image">
  \end{lstlisting} 
  \end{minipage}
  \hfill
  \begin{minipage}{0.39\textwidth}
    \begin{enumerate}[A]
      \item img.jpg
      \item This is an image
      \item Nothing, there is an error with the html
    \end{enumerate}
  \end{minipage}
\end{frame}

%
% Slide
%
\begin{frame}[fragile]
  \frametitle{Poll Question: HTML}
  Which of these statements is false?
  \begin{enumerate}[A]
    \item Newlines and indentation in HTML are ignored.
    \item There is more than one type of list in HTML.
    \item Image tags don't require closing tags.
    \item ``alt attributes'' for images help make web pages accessible.
    \item ``favicon'' is a small picture to represent your web page, often shown in the browser tab.
    \item None of the above
  \end{enumerate}
\end{frame}


%
% Slide 2
%
\begin{frame}[fragile]
  \frametitle{Reading data from the internet}
  \begin{lstlisting}[language=Python]
  import urllib.request

  req = urllib.request.Request("https://wikipedia.com")
  with urllib.request.urlopen(req) as response:
    the_page = response.read()

  print(the_page)
  \end{lstlisting} 
  \vfill
  \begin{enumerate}[A]
    \item From a Python program
    \item urllib module: Given a URL, returns the document at that URL
  \end{enumerate}
\end{frame}

\end{document}
