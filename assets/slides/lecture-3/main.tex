\documentclass{beamer}

\usepackage{graphicx}
\usepackage{textpos}
\usepackage{listings}
\usepackage{tikz}
\usepackage{listofitems}

\definecolor{codegreen}{rgb}{0,0.6,0}
\definecolor{codegray}{rgb}{0.5,0.5,0.5}
\definecolor{codepurple}{rgb}{0.58,0,0.82}
\definecolor{backcolour}{rgb}{0.95,0.95,0.92}

\lstdefinestyle{python}{
  backgroundcolor=\color{backcolour},   
  commentstyle=\color{codegreen},
  keywordstyle=\color{magenta},
  numberstyle=\tiny\color{codegray},
  stringstyle=\color{codepurple},
  basicstyle=\ttfamily\footnotesize,
  breakatwhitespace=false,         
  %breaklines=true,                 
  captionpos=b,                    
  keepspaces=true,                 
  numbers=left,                    
  numbersep=5pt,                  
  showspaces=false,                
  showstringspaces=false,
  showtabs=false,                  
  tabsize=2
}

\lstset{style=python}

\usetheme{Madrid}
\useoutertheme{miniframes} % Alternatively: miniframes, infolines, split

% Setup the university's color pallette
\definecolor{UIUCorange}{RGB}{19, 41, 75} 
\definecolor{UIUCblue}{RGB}{232, 74, 39} 


\setbeamercolor{palette primary}{bg=UIUCorange,fg=white}
\setbeamercolor{palette secondary}{bg=UIUCblue,fg=white}
\setbeamercolor{palette tertiary}{bg=UIUCblue,fg=white}
\setbeamercolor{palette quaternary}{bg=UIUCblue,fg=white}
\setbeamercolor{structure}{fg=UIUCorange} % itemize, enumerate, etc
\setbeamercolor{section in toc}{fg=UIUCblue} % TOC sections

\setbeamercolor{subsection in head/foot}{bg=UIUCorange,fg=UIUCblue}
\setbeamercolor{subsection in head/foot}{bg=UIUCorange,fg=UIUCblue}

\usepackage[utf8]{inputenc}
\usepackage{graphicx}


%Information to be included in the title page:
\title{\textbf{Topic 2: Variables and Expressions}}
\author{\textbf{David H Smith IV}}
\institute[\textbf{UIUC}]{\textbf{University of Illinois Urbana-Champaign}}
\date{\textbf{Tues, Aug 24 2021}}

\setbeamertemplate{title page}[default][colsep=-4bp,rounded=true]
\addtobeamertemplate{title page}{\vspace{3\baselineskip}}{}
\addtobeamertemplate{title page}{
  \begin{textblock*}{\paperwidth}(-1.0em, -1.2em)
    \includegraphics[width=\paperwidth, height=\paperheight]{imgs/uiuc.png}
  \end{textblock*} 
}{}

\begin{document}

\frame{\titlepage}

\section{Announcements}

\begin{frame}
  Things that are due tommorow night:
    \begin{itemize}
      \item Homework 2
      \item zyBooks Participation for Topic 2 (Part 2) 
      \item Post-reading for Topic 2 (Part 2)
    \end{itemize}
\end{frame}

\section{Review Polls}

%
%
%
\begin{frame}[fragile]
  \frametitle{Poll Question: Whitespace}
  Which of the following are considered \'whitespace\'?
  \vfill
  \begin{enumerate}
    \item Spaces
    \item Tabs
    \item Newlines
    \item Spaces and Tabs
    \item Spaces, Tabs, and Newlines
  \end{enumerate}
  \pause
  \vfill
  Whitespace is any character that takes up vertical or horizontal space bu does not produce an otherwise visible mark.
\end{frame}

%
%
%
\section{Topic Review: Escape Character}
\begin{frame}[fragile]
  \frametitle{Topic Review: Escape Character}
  \begin{enumerate}
    \item Treat backslash(\textbackslash) as a special character
    \item \textbackslash means that the character \textit{immediatly} following it should be treated differently.
      \begin{enumerate}
        \item \lstinline{\' and \"} escape quotes within a string.
        \item \lstinline{\t} encodes a tab.
        \item \lstinline{\n} encodes a new line.
        \item \lstinline{\\} encodes a slash.
      \end{enumerate}
  \end{enumerate}
\end{frame}

\section{Floating Points}
%
%
%
\begin{frame}[fragile]
  \frametitle{Poll Question: }
  What is printed to the screen when the user types in \lstinline|1|.
  \begin{lstlisting}[language=Python]
  x = input()
  y = float(x)
  print(y)
  \end{lstlisting}
  \vfill
  \begin{enumerate}[A]
    \item 1
    \item 1.0
    \item ValueError
    \item Other error
  \end{enumerate}
\end{frame}


%
%
%
\begin{frame}[fragile]
  \frametitle{Poll Question}
  What will the output be if the user types in \lstinline|4.5| then \lstinline|1|?
  \begin{lstlisting}[language=Python]
  x = int(input())
  y = int(input())
  z = x + y
  print(z)
  \end{lstlisting}
  \vfill
  \begin{enumerate}
    \item 5.5
    \item 6.0
    \item ValueError
    \item 6
  \end{enumerate}
\end{frame}

%
%
%
\begin{frame}[fragile]
  \frametitle{Poll Question: }
  What is the result of running the following code when the user types in \lstinline|4.5| and \lstinline|1|?
  \begin{lstlisting}[language=Python]
  x = int(float(input()))
  y = float(input())
  z = x + y
  print(z)
  \end{lstlisting}
  \vfill
  \begin{enumerate}
    \item
  \end{enumerate}
\end{frame}

%
%
%
\begin{frame}[fragile]
  \frametitle{Poll Question: }

  \vfill
  \begin{enumerate}
    \item
  \end{enumerate}
\end{frame}

%
%
%
\begin{frame}[fragile]
  \frametitle{Poll Question: }

  \vfill
  \begin{enumerate}
    \item
  \end{enumerate}
\end{frame}


%
%
%
\begin{frame}[fragile]
  \frametitle{Poll Question: Python Literals}
  Which of the following is not a valid Python literal?
  \begin{enumerate}[A]
    \item 1.00001
    \item 1E-7
    \item 1,097
    \item -3.00
    \item \lstinline{'\'\'''}
  \end{enumerate}
\end{frame}




\section{Variables}

%
%
%
\begin{frame}[fragile]
  \frametitle{Poll Question: Python Naming}
  \begin{minipage}{0.49\textwidth}
    Which are legal python names?
    \begin{itemize}
      \item \lstinline{____}
      \item \lstinline{12monkeys}
      \item \lstinline{l33tCoder42}
    \end{itemize}
  \end{minipage}
  \begin{minipage}{0.49\textwidth}
    \begin{enumerate}[A]
      \item 1 and 2
      \item 1 and 3
      \item 2 only
      \item 1, 2, and 3
    \end{enumerate}
  \end{minipage}
\end{frame}

%
%
%
\begin{frame}[fragile]
  \frametitle{Poll Question: }

  \vfill
  \begin{enumerate}
    \item
  \end{enumerate}
\end{frame}


%
%
%
\begin{frame}
  Things that are due tommorow night:
    \begin{itemize}
      \item Homework 2
      \item zyBooks Participation for Topic 2 (Part 2) 
      \item Post-reading for Topic 2 (Part 2)
    \end{itemize}
\end{frame}


\end{document}
