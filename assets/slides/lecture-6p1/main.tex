\documentclass{beamer}

\usepackage{graphicx}
\usepackage{textpos}
\usepackage{listings}
\usepackage{lstautogobble}

\usetheme{Madrid}
\useoutertheme{miniframes} % Alternatively: miniframes, infolines, split

% Setup the university's color pallette
\definecolor{UIUCorange}{RGB}{19, 41, 75} % UBC Blue (primary)
\definecolor{UIUCblue}{RGB}{232, 74, 39} % UBC Grey (secondary)

\definecolor{codegreen}{rgb}{0,0.6,0}
\definecolor{codegray}{rgb}{0.5,0.5,0.5}
\definecolor{codepurple}{rgb}{0.58,0,0.82}
\definecolor{backcolour}{rgb}{0.95,0.95,0.92}

\lstdefinestyle{python}{
  backgroundcolor=\color{backcolour},   
  commentstyle=\color{codegreen},
  keywordstyle=\color{magenta},
  numberstyle=\tiny\color{codegray},
  stringstyle=\color{codepurple},
  basicstyle=\ttfamily\footnotesize,
  breakatwhitespace=false,         
  belowskip=-0.5em,
  breaklines=true,                 
  captionpos=b,                    
  keepspaces=true,                 
  numbers=left,                    
  numbersep=5pt,                  
  showspaces=false,                
  showstringspaces=false,
  showtabs=false,                  
  tabsize=2
}

\lstset{style=python}

\AtBeginSection[]{
    \begin{frame}
        \vfill
        \centering
        \begin{beamercolorbox}[sep=8pt,center,shadow=true,rounded=true]{title}
            \usebeamerfont{title}\insertsectionhead\par%
        \end{beamercolorbox}
        \vfill
    \end{frame}
}
% Setup the university's color pallette
\definecolor{UIUCorange}{RGB}{19, 41, 75} % UBC Blue (primary)
\definecolor{UIUCblue}{RGB}{232, 74, 39} % UBC Grey (secondary)


\setbeamercolor{palette primary}{bg=UIUCorange,fg=white}
\setbeamercolor{palette secondary}{bg=UIUCblue,fg=white}
\setbeamercolor{palette tertiary}{bg=UIUCblue,fg=white}
\setbeamercolor{palette quaternary}{bg=UIUCblue,fg=white}
\setbeamercolor{structure}{fg=UIUCorange} % itemize, enumerate, etc
\setbeamercolor{section in toc}{fg=UIUCblue} % TOC sections

\setbeamercolor{subsection in head/foot}{bg=UIUCorange,fg=UIUCblue}
\setbeamercolor{subsection in head/foot}{bg=UIUCorange,fg=UIUCblue}

\usepackage[utf8]{inputenc}
\usepackage{graphicx}


%Information to be included in the title page:
\title{\textbf{Topic 6: Intro to For and While Loops}}
\author{\textbf{David H Smith IV}}
\institute[\textbf{UIUC}]{\textbf{University of Illinois Urbana-Champaign}}
\date{\textbf{Sat, 21 18 2021}}

\setbeamertemplate{title page}[default][colsep=-4bp,rounded=true]
\addtobeamertemplate{title page}{\vspace{3\baselineskip}}{}
\addtobeamertemplate{title page}{
  \begin{textblock*}{\paperwidth}(-1.0em, -1.2em)
    \includegraphics[width=\paperwidth, height=\paperheight]{imgs/uiuc.png}
  \end{textblock*} 
}{}

\begin{document}

\frame{\titlepage}

\section{Updates}

%
% Slide 2
%
\begin{frame}[fragile]
  \frametitle{Course Updates}
  \begin{enumerate}[A]
    \item Homework (PrairieLearn) will now have a 24 hours grace period with a 20\% penalty for late work.
    \item Lab 3 (Chromakey) - Due Sunday after next.
  \end{enumerate}
\end{frame}




\section{The \lstinline|range()| Function}

%
% Slide 2
%
\begin{frame}[fragile]
  \frametitle{Range Function Variations}
  The following are all variations of the range function:
  \begin{enumerate}[A]
    \item \lstinline|range(end)|
    \item \lstinline|range(start, end)|
    \item \lstinline|range(start, end, increment)|
  \end{enumerate}
  \vfill
  \textbf{Note: } $start \leq x < end$ for all \lstinline|x| in the range.
\end{frame}

%
% Slide 2
%
\begin{frame}[fragile]
  \frametitle{Poll Question: Range Function}
  What is printed to the screen?
  \begin{lstlisting}[language=Python, autogobble]
  x = range(1, 5)
  print(x)
  \end{lstlisting}
  \vfill
  \begin{enumerate}[A]
    \item \lstinline|range(1, 5)|
    \item \lstinline|[1, 2, 3, 4]|
    \item \lstinline|1, 2, 3, 4|
    \item \lstinline|1, 2, 3, 4, 5|
    \item \lstinline|[1, 2, 3, 4, 5]|
  \end{enumerate}
  \vfill \pause
  Why? What does the output of \lstinline|type(x)| tell us?
\end{frame}

%
% Slide 2
%
\begin{frame}[fragile]
  \frametitle{Poll Question: Range Function}
  What is printed to the screen?
  \begin{lstlisting}[language=Python, autogobble]
  x = range(1, 5)
  y = list(x)
  print(y)
  \end{lstlisting}
  \vfill
  \begin{enumerate}[A]
    \item \lstinline|[1, 2, 3, 4]|
    \item \lstinline|[1, 2, 3, 4, 5]|
    \item TypeError
    \item ValueError
  \end{enumerate}
\end{frame}

%
% Slide 2
%
\begin{frame}[fragile]
  \frametitle{Poll Question: Range Function}
  What is printed to the screen?
  \begin{lstlisting}[language=Python, autogobble]
  x = range(5)
  y = list(x)
  print(y)
  \end{lstlisting}
  \vfill
  \begin{enumerate}[A]
    \item \lstinline|[1, 2, 3, 4]|
    \item \lstinline|[1, 2, 3, 4, 5]|
    \item \lstinline|[0, 1, 2, 3, 4]|
    \item \lstinline|[0, 1, 2, 3, 4, 5]|
  \end{enumerate}
\end{frame}

%
% Slide 2
%
\begin{frame}[fragile]
  \frametitle{Poll Question: Range Function}
  What is printed to the screen?
  \begin{lstlisting}[language=Python, autogobble]
  x = range(0, 11, 2)
  y = list(x)
  print(y)
  \end{lstlisting}
  \vfill
  \begin{enumerate}[A]
    \item Some error
    \item \lstinline|[1, 3, 5, 7, 9]|
    \item \lstinline|[1, 3, 5, 7, 11]|
    \item \lstinline|[0, 2, 4, 6, 8, 10]|
  \end{enumerate}
\end{frame}

%
% Slide 2
%
\begin{frame}[fragile]
  \frametitle{Poll Question: Range Function}
  What is printed to the screen?
  \begin{lstlisting}[language=Python, autogobble]
  x = range(1, 11, 2)
  y = list(x)
  print(y)
  \end{lstlisting}
  \vfill
  \begin{enumerate}[A]
    \item Some error
    \item \lstinline|[1, 3, 5, 7, 9]|
    \item \lstinline|[1, 3, 5, 7, 11]|
    \item \lstinline|[0, 2, 4, 6, 8, 10]|
  \end{enumerate}
\end{frame}

%
% Slide 2
%
\begin{frame}[fragile]
  \frametitle{Poll Question: Range Function}
  What is printed to the screen?
  \begin{lstlisting}[language=Python, autogobble]
  x = range(5, 2)
  y = list(x)
  print(y)
  \end{lstlisting}
  \vfill
  \begin{enumerate}[A]
    \item Error
    \item \lstinline|[]|
    \item \lstinline|[5, 4, 3]|
    \item \lstinline|[4, 3, 2]|
    \item \lstinline|[5, 4, 3, 2]|
  \end{enumerate}
\end{frame}

%
% Slide 2
%
\begin{frame}[fragile]
  \frametitle{Poll Question: Range Function}
  What is printed to the screen?
  \begin{lstlisting}[language=Python, autogobble]
  x = range(5, 2, -1)
  y = list(x)
  print(y)
  \end{lstlisting}
  \vfill
  \begin{enumerate}[A]
    \item Error
    \item \lstinline|[]|
    \item \lstinline|[5, 4, 3]|
    \item \lstinline|[4, 3, 2]|
    \item \lstinline|[5, 4, 3, 2]|
  \end{enumerate}
\end{frame}

\section{For Loops}

%
% Slide 2
%
\begin{frame}[fragile]
  \frametitle{Poll Question: For Loops}
  How many lines are printed to the screen?
  \begin{lstlisting}[language=Python, autogobble]
  course_times = {'CS 105': 'F9-11', "CS 125": "MWF11-12"}
  for course in course_times:
    print(course, 'meets', course_times(course))
  \end{lstlisting}
  \vfill
  \begin{enumerate}[A]
    \item 0
    \item 2
    \item 4
    \item error
  \end{enumerate}
  \pause
  What do we need to do to fix it?
\end{frame}

%
% Slide 2
%
\begin{frame}[fragile]
  \frametitle{Poll Question: For Loops}
  How many lines are printed to the screen?
  \begin{lstlisting}[language=Python, autogobble]
  course_times = {'CS 105': 'F9-11', "CS 125": "MWF11-12"}
  for course in course_times:
    print(course, 'meets', course_times[course])
  \end{lstlisting}
  \vfill
  \begin{enumerate}[A]
    \item 0
    \item 2
    \item 4
  \end{enumerate}
\end{frame}

%
% Slide 2
%
\begin{frame}[fragile]
  \frametitle{Poll Question: For Loop}
  How many lines will be printed?
  \begin{lstlisting}[language=Python, autogobble]
  things = [22, [33, 44], 55, [66]]
  for thing in things:
    print(thing)
  \end{lstlisting}
  \vfill
  \begin{enumerate}[A]
    \item SyntaxError
    \item 5
    \item 4
    \item TypeError
  \end{enumerate}
  \pause
  What will be printed to the screen?
\end{frame}

%
% Slide 2
%
\begin{frame}[fragile]
  \frametitle{Poll Question: For Loop and Range}
  How many lines are printed to the screen?
  \begin{lstlisting}[language=Python, autogobble]
  for i in range(0, 10):
    print(i)
  \end{lstlisting}
  \vfill
  \begin{enumerate}[A]
    \item 11
    \item 10
    \item 9
    \item TypeError
  \end{enumerate}
\end{frame}

%
% Slide 2
%
\begin{frame}[fragile]
  \frametitle{Poll Question: For Loop and Range}
  How many lines are printed to the screen?
  \begin{lstlisting}[language=Python, autogobble]
  for i in range(-3, 9, 4):
    print(i)
  \end{lstlisting}
  \vfill
  \begin{enumerate}[A]
    \item 3
    \item 4
    \item 5
    \item 0
  \end{enumerate}
\end{frame}

%
% Slide 2
%
\begin{frame}[fragile]
  \frametitle{Poll Question: Nested For Loops}
  How many lines are printed to the screen?
  \begin{lstlisting}[language=Python, autogobble]
  list1 = ['lemon', 'orange', 'lime']
  list2 = ['banana', 'lemon']

  for thing1 in list1:
    for thing2 in list2:
      print(thing2)
  \end{lstlisting}
  \vfill
  \begin{enumerate}[A]
    \item 5
    \item 6
    \item 7
    \item SyntaxError
  \end{enumerate}
\end{frame}

\section{While Loops}

%
%
%
\begin{frame}[fragile]
  \frametitle{While Loops: If Statements that Just Keep Going}
    Key differences beween for loops and while loops:
    \begin{enumerate}
      \item For loops iterate over collection. While loops iterate while boolean expression is true.
      \item For loops terminate when they run out of values. While loops terminate when a boolean expression is false.
    \end{enumerate}
    \vfill
  \begin{lstlisting}
  while <cond>:
    # Function body code below
    ...
  \end{lstlisting}
\end{frame}

%
% Slide 2
%
\begin{frame}[fragile]
  \frametitle{Poll Question: While Loops}
  How many lines are printed?
  \begin{lstlisting}[language=Python, autogobble]
  num = 14
  while num >= 1:
    print(num)
    num = num // 2
  \end{lstlisting}
  \vfill
  \begin{enumerate}[A]
    \item 3
    \item 4
    \item 5
    \item 7
  \end{enumerate}
\end{frame}

%
% Slide 2
%
\begin{frame}[fragile]
  \frametitle{Poll Question: While Loops}
  How many times will this while loop iterate before terminating?
  \begin{lstlisting}[language=Python, autogobble]
  i = 0 
  while i < 3:
    i + 1
  \end{lstlisting}
  \vfill
  \begin{enumerate}[A]
    \item 2
    \item 3
    \item 4
    \item $\infty$
  \end{enumerate}
  \pause
  How do we fix this?
\end{frame}

%
% Slide 2
%
\begin{frame}[fragile]
  \frametitle{Poll Question: While Loops}
  What is the resulting value of \lstinline|sum| if the user attempts this sequence of responses: y, 10, y, 5, n, 3, y, 5.
  \begin{lstlisting}[language=Python, autogobble]
  sum = 0
  while input("Enter another number?") == "y":
    sum += int(input("Num: "))
  \end{lstlisting}
  \vfill
  \begin{enumerate}[A]
    \item This code is not valid 
    \item \lstinline|"y10y5n3y5"|
    \item \lstinline|15|
    \item \lstinline|20|
  \end{enumerate}
\end{frame}

%
% Slide 2
%
\begin{frame}[fragile]
  \frametitle{For vs While Loop: Side by Side}
  These two pieces of code are equivalant:
  \begin{minipage}{0.48\textwidth}
    \begin{lstlisting}[language=Python, autogobble]
    x = [1, 2, 3, 4]
    for item in x:
      print(item)
    \end{lstlisting}
  \end{minipage}
  \hfill
  \begin{minipage}{0.48\textwidth}
    \begin{lstlisting}[language=Python, autogobble]
    x = [1, 2, 3, 4]
    i = 0
    while i < len(x):
      item = x[i]
      print(item)
      i += 1
    \end{lstlisting}
  \end{minipage}
\end{frame}



\section{Loops in Functions}

%
% Slide 2
%
\begin{frame}[fragile]
	\frametitle{Poll Question: Constructing Conditionals}
  Which of the following functions correctly counts and returns the number of strings in a list of strings that are of an even length.
	\vfill
	\begin{minipage}{0.49\textwidth}
    1)
		\begin{lstlisting}[language=Python, autogobble,basicstyle=\tiny,numbers=none]
    def count_even_len(str_list):
      count = 0 
      for string in str_list:
        if len(string) % 2 == 0:
          count += 1
      return count
		\end{lstlisting}
	\end{minipage}
	\begin{minipage}{0.49\textwidth}
    2)
		\begin{lstlisting}[language=Python, autogobble,basicstyle=\tiny,numbers=none]
    def count_even_len(str_list):
      count = 0 
      i = 0
      while i < len(str_list):
        if len(str_list[i]) % 2 == 0:
          count += 1
        i += 1
      return count
		\end{lstlisting}
	\end{minipage}
  \vfill
  \centering
  \begin{minipage}{0.45\textwidth}
    3)
    \begin{lstlisting}[language=Python, autogobble,basicstyle=\tiny,numbers=none]
    def count_even_len(str_list):
      count = 0 
      i = 0
      while i <= len(str_list):
        if len(str_list[i]) % 2 == 0:
          count += 1
        i += 1
      return count
    \end{lstlisting}
  \end{minipage}
\end{frame}

%
% Slide 2
%
\begin{frame}[fragile]
  \frametitle{Poll Question: Constructing Conditionals}
  Which of the following will append vals from the parameter that are strings that contain a substring?
  \tiny
  \vfill
  1)
  \begin{lstlisting}[language=Python, autogobble,basicstyle=\tiny,numbers=none]
    def append_if_has_substring(my_list, substr, val1, val2, val3):
      for val in [val1, val2, val3]:
        if substr in val and type(val) is str:
          my_list.append(val)
  \end{lstlisting}
  \vfill
  2)
  \begin{lstlisting}[language=Python, autogobble,basicstyle=\tiny,numbers=none]
    def append_if_has_substring(my_list, substr, val1, val2, val3):
      for val in [val1, val2, val3]:
        if type(val) is str and substr in val:
          my_list.append(val)
  \end{lstlisting}
  \vfill
  3)
  \begin{lstlisting}[language=Python, autogobble,basicstyle=\tiny,numbers=none]
    def append_if_has_substring(my_list, substr, val1, val2, val3):
      for val in [val1, val2, val3]:
        if type(val) is str and substr in val:
          my_list = my_list.append(val)
  \end{lstlisting}
  \vfill
  4)
  \begin{lstlisting}[language=Python, autogobble,basicstyle=\tiny,numbers=none]
    def append_if_has_substring(my_list, substr, val1, val2, val3):
      for val in [val1, val2, val3]:
        if type(val) is str and substr in val:
          my_list.add(val)
  \end{lstlisting}
\end{frame}

%
% Slide 2
%
\begin{frame}[fragile]
	\frametitle{Poll Question: Constructing Conditionals}
  What function returns a list of keys that are associated with even values? You can assume that the values are of type int.
	\vfill
	\begin{minipage}{0.32\textwidth}
    1)
		\begin{lstlisting}[language=Python, autogobble,basicstyle=\tiny,numbers=none]
    def get_even_value_keys(d):
      l = []
      for foo in d:
        if d[foo] % 2 == 0:
          l.append(foo)
      return l
		\end{lstlisting}
	\end{minipage}
	\begin{minipage}{0.32\textwidth}
    2)
		\begin{lstlisting}[language=Python, autogobble,basicstyle=\tiny,numbers=none]
    def get_even_value_keys(d):
      l = []
      i = 0
      while i < len(d):
        if d[i] % 2 == 0:
          l.append(i)
      return l
		\end{lstlisting}
	\end{minipage}
	\begin{minipage}{0.32\textwidth}
    3)
		\begin{lstlisting}[language=Python, autogobble,basicstyle=\tiny,numbers=none]
    def get_even_value_keys(d):
      l = []
      for foo in d:
        if foo % 2 == 0:
          l.append(foo)
      return l
		\end{lstlisting}
	\end{minipage}
\end{frame}


\end{document}
