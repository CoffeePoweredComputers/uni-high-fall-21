\documentclass{beamer}

\usepackage{graphicx}
\usepackage{textpos}
\usepackage{listings}
\usepackage{lstautogobble}

\usetheme{Madrid}
\useoutertheme{miniframes} % Alternatively: miniframes, infolines, split

% Setup the university's color pallette
\definecolor{UIUCorange}{RGB}{19, 41, 75} % UBC Blue (primary)
\definecolor{UIUCblue}{RGB}{232, 74, 39} % UBC Grey (secondary)

\definecolor{codegreen}{rgb}{0,0.6,0}
\definecolor{codegray}{rgb}{0.5,0.5,0.5}
\definecolor{codepurple}{rgb}{0.58,0,0.82}
\definecolor{backcolour}{rgb}{0.95,0.95,0.92}

\lstdefinestyle{python}{
  backgroundcolor=\color{backcolour},   
  commentstyle=\color{codegreen},
  keywordstyle=\color{magenta},
  numberstyle=\tiny\color{codegray},
  stringstyle=\color{codepurple},
  basicstyle=\ttfamily\footnotesize,
  breakatwhitespace=false,         
  belowskip=-0.5em,
  breaklines=true,                 
  captionpos=b,                    
  keepspaces=true,                 
  numbers=left,                    
  numbersep=5pt,                  
  showspaces=false,                
  showstringspaces=false,
  showtabs=false,                  
  tabsize=2
}

\lstset{style=python}

\AtBeginSection[]{
    \begin{frame}
        \vfill
        \centering
        \begin{beamercolorbox}[sep=8pt,center,shadow=true,rounded=true]{title}
            \usebeamerfont{title}\insertsectionhead\par%
        \end{beamercolorbox}
        \vfill
    \end{frame}
}
% Setup the university's color pallette
\definecolor{UIUCorange}{RGB}{19, 41, 75} % UBC Blue (primary)
\definecolor{UIUCblue}{RGB}{232, 74, 39} % UBC Grey (secondary)


\setbeamercolor{palette primary}{bg=UIUCorange,fg=white}
\setbeamercolor{palette secondary}{bg=UIUCblue,fg=white}
\setbeamercolor{palette tertiary}{bg=UIUCblue,fg=white}
\setbeamercolor{palette quaternary}{bg=UIUCblue,fg=white}
\setbeamercolor{structure}{fg=UIUCorange} % itemize, enumerate, etc
\setbeamercolor{section in toc}{fg=UIUCblue} % TOC sections

\setbeamercolor{subsection in head/foot}{bg=UIUCorange,fg=UIUCblue}
\setbeamercolor{subsection in head/foot}{bg=UIUCorange,fg=UIUCblue}

\usepackage[utf8]{inputenc}
\usepackage{graphicx}

%Information to be included in the title page:
\title{\textbf{Programming Patterns - Reference Sheet}}
\author{\textbf{David H Smith IV}}
\institute[\textbf{UIUC}]{\textbf{University of Illinois Urbana-Champaign}}
\date{\textbf{}}

\setbeamertemplate{title page}[default][colsep=-4bp,rounded=true]
\addtobeamertemplate{title page}{\vspace{3\baselineskip}}{}
\addtobeamertemplate{title page}{
  \begin{textblock*}{\paperwidth}(-1.0em, -1.2em)
    \includegraphics[width=\paperwidth, height=\paperheight]{imgs/uiuc.png}
  \end{textblock*} 
}{}

\begin{document}

\frame{\titlepage}


%
% Slide 2
%
\begin{frame}[fragile]
  \frametitle{Counting Pattern}
  \begin{lstlisting}[language=Python, autogobble][language=Python, autogobble]
  def count(collection):
    counter = 0
    for item in collection:
      if <item meets condition>:
        counter += 1
    return counter
  \end{lstlisting}
\end{frame}

%
% Slide 2
%
\begin{frame}[fragile]
  \frametitle{Computing a Sum/Total}
  \begin{lstlisting}[language=Python, autogobble][language=Python, autogobble]
  def sum(collection):
    total = 0
    for item in collection:
      total += item
    return total
  \end{lstlisting}
\end{frame}

%
% Slide 2
%
\begin{frame}[fragile]
  \frametitle{Finding (single thing) in a Collection}
  \begin{lstlisting}[language=Python, autogobble][language=Python, autogobble]
  def find_thing(collection):
    for thing in collection:
      if <thing meets condition>:
        return thing
  \end{lstlisting}
  \vfill
  \begin{lstlisting}[language=Python, autogobble][language=Python, autogobble]
  def find_thing(collection):
    found = None
    for thing in collection:
      if <thing meets condition>:
        found = thing
        break
    return found
  \end{lstlisting}
\end{frame}

%
% Slide 2
%
\begin{frame}[fragile]
  \frametitle{Finding best in collection}
  \begin{lstlisting}[language=Python, autogobble][language=Python, autogobble]
  def find_best(collection):
    currentbest = ??
    for thing in collection:
      if <thing is better than current best>:
        currentbest = thing
    return currentbest 
  \end{lstlisting}
  \vfill
  \begin{itemize}
    \item If we're searching over a list and we want to return the largest or smaller number: \lstinline|currentbest = stufflist[0]| 
    \item If we're searching over a list of strings and we want to return the longest string: \lstinline|currentbest = stufflist[0]| or \lstinline|currentbest = ""|
    \item If you know the list contains only non-negative integers: \lstinline|currentbest = -1|
  \end{itemize}
\end{frame}

%
% Slide 2
%
\begin{frame}[fragile]
  \frametitle{Filtering a collection}
  \begin{lstlisting}[language=Python, autogobble][language=Python, autogobble]
  def filter(collection):
    new_list = []

    for thing in collection:
      if <thing meets criteria>:
        newlist.append(thing)

    return new_list
  \end{lstlisting}
\end{frame}

\end{document}
