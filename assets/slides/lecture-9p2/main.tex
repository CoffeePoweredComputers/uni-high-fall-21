\documentclass{beamer}

\usepackage{graphicx}
\usepackage{textpos}
\usepackage{listings}
\usepackage{lstautogobble}

\usetheme{Madrid}
\useoutertheme{miniframes} % Alternatively: miniframes, infolines, split

% Setup the university's color pallette
\definecolor{UIUCorange}{RGB}{19, 41, 75} % UBC Blue (primary)
\definecolor{UIUCblue}{RGB}{232, 74, 39} % UBC Grey (secondary)

\definecolor{codegreen}{rgb}{0,0.6,0}
\definecolor{codegray}{rgb}{0.5,0.5,0.5}
\definecolor{codepurple}{rgb}{0.58,0,0.82}
\definecolor{backcolour}{rgb}{0.95,0.95,0.92}

\lstdefinestyle{python}{
  backgroundcolor=\color{backcolour},   
  commentstyle=\color{codegreen},
  keywordstyle=\color{magenta},
  numberstyle=\tiny\color{codegray},
  stringstyle=\color{codepurple},
  basicstyle=\ttfamily\footnotesize,
  breakatwhitespace=false,         
  belowskip=-0.5em,
  breaklines=true,                 
  captionpos=b,                    
  keepspaces=true,                 
  numbers=left,                    
  numbersep=5pt,                  
  showspaces=false,                
  showstringspaces=false,
  showtabs=false,                  
  tabsize=2
}

\lstset{style=python}

\AtBeginSection[]{
    \begin{frame}
        \vfill
        \centering
        \begin{beamercolorbox}[sep=8pt,center,shadow=true,rounded=true]{title}
            \usebeamerfont{title}\insertsectionhead\par%
        \end{beamercolorbox}
        \vfill
    \end{frame}
}
% Setup the university's color pallette
\definecolor{UIUCorange}{RGB}{19, 41, 75} % UBC Blue (primary)
\definecolor{UIUCblue}{RGB}{232, 74, 39} % UBC Grey (secondary)


\setbeamercolor{palette primary}{bg=UIUCorange,fg=white}
\setbeamercolor{palette secondary}{bg=UIUCblue,fg=white}
\setbeamercolor{palette tertiary}{bg=UIUCblue,fg=white}
\setbeamercolor{palette quaternary}{bg=UIUCblue,fg=white}
\setbeamercolor{structure}{fg=UIUCorange} % itemize, enumerate, etc
\setbeamercolor{section in toc}{fg=UIUCblue} % TOC sections

\setbeamercolor{subsection in head/foot}{bg=UIUCorange,fg=UIUCblue}
\setbeamercolor{subsection in head/foot}{bg=UIUCorange,fg=UIUCblue}

\usepackage[utf8]{inputenc}
\usepackage{graphicx}

%Information to be included in the title page:
\title{\textbf{Files Continued}}
\author{\textbf{David H Smith IV}}
\institute[\textbf{UIUC}]{\textbf{University of Illinois Urbana-Champaign}}
\date{\textbf{Tues, Oct 12 2021}}

\setbeamertemplate{title page}[default][colsep=-4bp,rounded=true]
\addtobeamertemplate{title page}{\vspace{3\baselineskip}}{}
\addtobeamertemplate{title page}{
  \begin{textblock*}{\paperwidth}(-1.0em, -1.2em)
    \includegraphics[width=\paperwidth, height=\paperheight]{imgs/uiuc.png}
  \end{textblock*} 
}{}

\begin{document}

\frame{\titlepage}

\section{Reminders}

%
% Slide 1
%
\begin{frame}
  \frametitle{Reminders}
  \begin{itemize}
    \item Last homework of the quarter is due Thursday by the end of the day.
  \end{itemize}
\end{frame}

\section{OS Review}

%
% Slide 2
%
\begin{frame}[fragile]
  \frametitle{Poll Question: Opening a File}
  What is returned by os.getcwd() when called in myfile.py?
  \begin{lstlisting}
  home
  |__tracertong
     |---mydata.csv
     |___dev_website
         |---index.html
         |___elements
             |--- myfile.py   <-- You are here
  \end{lstlisting} 
  \vfill
  \begin{enumerate}[A]
    \item \lstinline|"/home/tracertong/dev_website/elements"|
    \item \lstinline|"/home/tracertong/dev_website/elements/myfile.py"|
    \item \lstinline|"dev_website/elements/myfile.py"|
    \item \lstinline|"elements/myfile.py"|
  \end{enumerate}
\end{frame}

\begin{frame}[fragile]
  \frametitle{Poll Question: Opening a File}
  How many times will this function call iterate if placed in a for loop: \lstinline|os.walk("dev_website")|?
  \begin{lstlisting}
  dev_website
  |---index.html
  |___elements
      |--- about.html
      |--- cv.html
      |--- myfile.py
      |___ projects.html
  \end{lstlisting} 
  \vfill
  \begin{enumerate}[A]
    \item 1
    \item 2
    \item 6
    \item 0
  \end{enumerate}
\end{frame}

\section{Files Review}
%
% Slide 2
%
\begin{frame}[fragile]
  \frametitle{Files}
  Which of the following reads all the contents of a file into a list of strings?
  \vfill
  \begin{enumerate}[A]
    \item readlines
    \item readall
    \item read
    \item readline
  \end{enumerate}
\end{frame}

%
% Slide 2
%
\begin{frame}[fragile]
  \frametitle{Poll Question: Read Characters}
  Given a variable named \lstinline|file_object| that contains a file object which of the following will read the next 15 character into a variable named \lstinline|title|.
  \vfill
  \begin{enumerate}[A]
    \item \lstinline| title = file_object.read(15)|
    \item \lstinline| title = file_object.read(14)|
    \item \lstinline| title = file_object.reads(15)|
    \item \lstinline| title = read(file_object, 15)|
  \end{enumerate}
\end{frame}

%
% Slide 2
%
\begin{frame}[fragile]
  \frametitle{Poll Question: Files}
  Read from a file that may not exist?
  \vfill
  \begin{enumerate}[A]
    \item \lstinline|outf = open('filename', 'r+w')|
    \item \lstinline|outf = open('filename', 'rw')|
    \item \lstinline|outf = open('filename', 'r')|
    \item \lstinline|outf = open('filename', 'r+')|
    \item \lstinline|outf = open('filename', 'w+')|
  \end{enumerate}
\end{frame}


%
% Slide 2
%
\begin{frame}[fragile]
  \frametitle{Poll Question: Opening a File}
  My data.csv contains three rows. What is the result of: \lstinline|len(open("mydata.csv").readlines())|.
  \begin{lstlisting}
  home
  |__tracertong
     |---mydata.csv
     |___dev_website
         |---index.html
         |___elements
             |___ myfile.py   <-- You are here
  \end{lstlisting} 
  \vfill
  \begin{enumerate}[A]
    \item 2
    \item 3
    \item 4
    \item FileNotFoundError
    \item SyntaxError
  \end{enumerate}
\end{frame}

%
% Slide 2
%
\begin{frame}[fragile]
  \frametitle{Poll Question: Opening a File}
  We are script myfile.py and we want to open mydata.csv.
  \begin{lstlisting}
  home
  |__tracertong
     |---mydata.csv
     |___dev_website
         |---index.html
         |___elements
             |___ myfile.py   <-- You are here
  \end{lstlisting} 
  \vfill
  \begin{enumerate}[A]
    \item \lstinline|fo = open("../../mydata.csv")|
    \item \lstinline|fo = open("/../../mydata.csv")|
    \item \lstinline|fo = open("/tracertong/mydata.csv")|
    \item \lstinline|fo = open("../../../mydata.csv")|
  \end{enumerate}
\end{frame}



\section{With-As vs Open-Close}
%
% Slide 2
%
\begin{frame}[fragile]
  \frametitle{Reading from Files}
  Method 1:
  \begin{lstlisting}[language=Python, autogobble]
  file_object = open('filename')
  lines = file_object.readlines()
  for line in lines:
    print(line)
  file_object.close()
  \end{lstlisting}
  \vfill
  Method 2:
  \begin{lstlisting}[language=Python, autogobble]
  with open('filename') as inf:
    lines = inf.readlines()
    for line in lines:
      print(line)
    #automatic file close
  \end{lstlisting}
\end{frame}

%
% Slide 2
%
\begin{frame}[fragile]
  \frametitle{Writing to Files}
  Method 1:
  \begin{lstlisting}[language=Python, autogobble]
  file_object = open('filename', 'w')
  file_object.write('thing to write')
  file_objet.close()  #automatic at program end
  file_object.flush() #optional
  \end{lstlisting}
  \vfill
  Method 2:
  \begin{lstlisting}[language=Python, autogobble]
  with open('filename', 'w') as outf:
    outf.write('thing to write')
    #automatic file close
  \end{lstlisting}
\end{frame}

\section{File Patterns}

%
% Slide 2
%
\begin{frame}[fragile]
  \frametitle{Patterns and Files}
  \begin{minipage}{0.38\textwidth}
    Usual Sum/Total:
    \begin{lstlisting}[language=Python, autogobble]
    def foo(some_list):
      total = 0
      for item in some_list
        total += item
      return total
    \end{lstlisting}
  \end{minipage}
  \hfill
  \begin{minipage}{0.58\textwidth}
    Sum/Total Pattern w/ File:
    \begin{lstlisting}[language=Python, autogobble]
    def foo(filename):
      file_object = open(filename)
      lines = file_object.readlines()
      total = 0
      for line in lines:
        total += int(line)
      return total
    \end{lstlisting}
  \end{minipage}
\end{frame}

\section{Argument Vector (argv)}

%
% Slide 2
%
\begin{frame}[fragile]
  \frametitle{Argument Vector (argv)}
  Example script time...
\end{frame}

\end{document}
