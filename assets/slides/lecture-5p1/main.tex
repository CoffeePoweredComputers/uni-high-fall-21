\documentclass{beamer}

\usepackage{graphicx}
\usepackage{textpos}
\usepackage{listings}
\usepackage{lstautogobble}

\usetheme{Madrid}
\useoutertheme{miniframes} % Alternatively: miniframes, infolines, split

% Setup the university's color pallette
\definecolor{UIUCorange}{RGB}{19, 41, 75} % UBC Blue (primary)
\definecolor{UIUCblue}{RGB}{232, 74, 39} % UBC Grey (secondary)

\definecolor{codegreen}{rgb}{0,0.6,0}
\definecolor{codegray}{rgb}{0.5,0.5,0.5}
\definecolor{codepurple}{rgb}{0.58,0,0.82}
\definecolor{backcolour}{rgb}{0.95,0.95,0.92}

\lstdefinestyle{python}{
	backgroundcolor=\color{backcolour},   
	commentstyle=\color{codegreen},
	keywordstyle=\color{magenta},
	numberstyle=\tiny\color{codegray},
	stringstyle=\color{codepurple},
	basicstyle=\ttfamily\footnotesize,
	breakatwhitespace=false,         
	belowskip=-0.5em,
	breaklines=true,                 
	captionpos=b,                    
	keepspaces=true,                 
	numbers=left,                    
	numbersep=5pt,                  
	showspaces=false,                
	showstringspaces=false,
	showtabs=false,                  
	tabsize=2
}

\lstset{style=python}

\AtBeginSection[]{
	\begin{frame}
		\vfill
		\centering
		\begin{beamercolorbox}[sep=8pt,center,shadow=true,rounded=true]{title}
			\usebeamerfont{title}\insertsectionhead\par%
		\end{beamercolorbox}
		\vfill
	\end{frame}
}


\setbeamercolor{palette primary}{bg=UIUCorange,fg=white}
\setbeamercolor{palette secondary}{bg=UIUCblue,fg=white}
\setbeamercolor{palette tertiary}{bg=UIUCblue,fg=white}
\setbeamercolor{palette quaternary}{bg=UIUCblue,fg=white}
\setbeamercolor{structure}{fg=UIUCorange} % itemize, enumerate, etc
\setbeamercolor{section in toc}{fg=UIUCblue} % TOC sections

\setbeamercolor{subsection in head/foot}{bg=UIUCorange,fg=UIUCblue}
\setbeamercolor{subsection in head/foot}{bg=UIUCorange,fg=UIUCblue}

\usepackage[utf8]{inputenc}
\usepackage{graphicx}


%Information to be included in the title page:
\title{\textbf{Topic 5: Conditionals}}
\author{\textbf{David H Smith IV}}
\institute[\textbf{UIUC}]{\textbf{University of Illinois Urbana-Champaign}}
\date{\textbf{Mon, July 05 2021}}

\setbeamertemplate{title page}[default][colsep=-4bp,rounded=true]
\addtobeamertemplate{title page}{\vspace{3\baselineskip}}{}
\addtobeamertemplate{title page}{
	\begin{textblock*}{\paperwidth}(-1.0em, -1.2em)
		\includegraphics[width=\paperwidth, height=\paperheight]{imgs/uiuc.png}
	\end{textblock*} 
}{}

\begin{document}

\frame{\titlepage}

%
% Slide 1
%
\begin{frame}
	\frametitle{Weekly Reminders}
	\begin{enumerate}[A]
		\item 
	\end{enumerate}
\end{frame}

\section{Truth Tables}

%
% Slide 2
%
\begin{frame}[fragile]
	\frametitle{The Basic Operations: OR, AND, NOT}
	\centering

	% Please add the following required packages to your document preamble:
	% \usepackage[table,xcdraw]{xcolor}
	% If you use beamer only pass "xcolor=table" option, i.e. \documentclass[xcolor=table]{beamer}
	\begin{table}[]
		\begin{tabular}{|ccc|ccc|cc|}
			{\color[HTML]{CD9934} \textbf{A}} & {\color[HTML]{3531FF} \textbf{B}} & \textbf{A OR B} & {\color[HTML]{CD9934} \textbf{A}} & {\color[HTML]{3531FF} \textbf{B}} & \textbf{A AND B} & {\color[HTML]{CD9934} \textbf{A}} & \textbf{NOT A} \\ \hline
			{\color[HTML]{CD9934} T}          & {\color[HTML]{3531FF} T}          & T               & {\color[HTML]{CD9934} T}          & {\color[HTML]{3531FF} T}          & T                & {\color[HTML]{CD9934} T}          & T              \\
			{\color[HTML]{CD9934} T}          & {\color[HTML]{3531FF} F}          & F               & {\color[HTML]{CD9934} T}          & {\color[HTML]{3531FF} F}          & F                & {\color[HTML]{CD9934} F}          & T              \\
			{\color[HTML]{CD9934} F}          & {\color[HTML]{3531FF} T}          & T               & {\color[HTML]{CD9934} F}          & {\color[HTML]{3531FF} T}          & T                & {\color[HTML]{CD9934} T}          & F              \\
			{\color[HTML]{CD9934} F}          & {\color[HTML]{3531FF} F}          & F               & {\color[HTML]{CD9934} F}          & {\color[HTML]{3531FF} F}          & F                & {\color[HTML]{CD9934} F}          & F             
		\end{tabular}
	\end{table}
\end{frame}

%
% Slide 2
%
\begin{frame}[fragile]
	\frametitle{Group Work: Combining Operations}
	\centering
	% Please add the following required packages to your document preamble:
	% \usepackage[table,xcdraw]{xcolor}
	% If you use beamer only pass "xcolor=table" option, i.e. \documentclass[xcolor=table]{beamer}
	\begin{table}[]
		\begin{tabular}{|cc|c|}
			{\color[HTML]{CD9934} \textbf{A}} & {\color[HTML]{3531FF} \textbf{B}} & \textbf{(A OR B) AND (NOT B)} \\ \hline
			{\color[HTML]{CD9934} T}          & {\color[HTML]{3531FF} T}          & ?                             \\
			{\color[HTML]{CD9934} T}          & {\color[HTML]{3531FF} F}          & ?                             \\
			{\color[HTML]{CD9934} F}          & {\color[HTML]{3531FF} T}          & ?                             \\
			{\color[HTML]{CD9934} F}          & {\color[HTML]{3531FF} F}          & ?                           
		\end{tabular}
	\end{table}
\end{frame}

%
% Slide 2
%
\begin{frame}[fragile]
	\frametitle{Group Work: Combining Operations}
	\centering
	% Please add the following required packages to your document preamble:
	% \usepackage[table,xcdraw]{xcolor}
	% If you use beamer only pass "xcolor=table" option, i.e. \documentclass[xcolor=table]{beamer}
	\begin{table}[]
		\begin{tabular}{|cc|c|}
			{\color[HTML]{CD9934} \textbf{A}} & {\color[HTML]{3531FF} \textbf{B}} & \textbf{(A OR B) AND (NOT B)} \\ \hline
			{\color[HTML]{CD9934} T}          & {\color[HTML]{3531FF} T}          & T                             \\
			{\color[HTML]{CD9934} T}          & {\color[HTML]{3531FF} F}          & F                             \\
			{\color[HTML]{CD9934} F}          & {\color[HTML]{3531FF} T}          & T                             \\
			{\color[HTML]{CD9934} F}          & {\color[HTML]{3531FF} F}          & F                            
		\end{tabular}
	\end{table}
\end{frame}


%
% Slide 2
%
\begin{frame}[fragile]
	\frametitle{De Morgan's Laws}
	\centering
	\begin{itemize}
		\item NOT(A AND B) = (NOT(A) OR NOT(B))
		\item NOT(A OR B) = (NOT(A) AND NOT(B))
	\end{itemize}
\end{frame}


\section{Boolean Expressions}

%
% Slide 2
%
\begin{frame}[fragile]
	\frametitle{Truth Table to Expressions}
	\vfill
	The As and Bs in the truth tables correspond to \textit{the result of boolean expressions}.
	\vfill
	\begin{minipage}{0.4\textwidth}
		\begin{lstlisting}[language=Python,autogobble]
	# Get some variables 
	x = int(input())
	y = int(input())

	# Construct the expr
  # and assign to A or B
	A = (x == 3)
	B = (y > 5)
		\end{lstlisting}
	\end{minipage}
	\begin{minipage}{0.59\textwidth}
		\begin{table}[]
			\begin{tabular}{|cc|c|c|c|}
				{\color[HTML]{CD9934} \textbf{A}} & {\color[HTML]{3531FF} \textbf{B}} & \textbf{A OR B} & \textbf{A AND B} & \textbf{NOT A} \\ \hline
				{\color[HTML]{CD9934} T}          & {\color[HTML]{3531FF} T}          & T               & T                & T              \\
				{\color[HTML]{CD9934} T}          & {\color[HTML]{3531FF} F}          & F               & F                & T              \\
				{\color[HTML]{CD9934} F}          & {\color[HTML]{3531FF} T}          & T               & T                & F              \\
				{\color[HTML]{CD9934} F}          & {\color[HTML]{3531FF} F}          & F               & F                & F             
			\end{tabular}
		\end{table}
	\end{minipage}
	\vfill
	\pause
	\begin{itemize}
		\item You can have \textbf{as many operands (e.g., A, B, C, $\cdots$) as you like}.
		\item The truth tables get BIG as you have to consider more permutations.
	\end{itemize}
\end{frame}

\section{Poll Questions}

%
% Slide 2
%
\begin{frame}[fragile]
	\frametitle{Poll Question: Boolean Expressions}
	Expressions that evaluate to \lstinline|True| or \lstinline|False|. 
	\begin{lstlisting}[language=Python, autogobble]
	(1 + 6) < (2 + 5)
	\end{lstlisting}
	\vfill
	\begin{enumerate}[A]
		\item \lstinline|True|
		\item \lstinline|False|
		\item TypeError
		\item SyntaxError
	\end{enumerate}
\end{frame}

%
% Slide 3
%
\begin{frame}[fragile]
	\frametitle{Poll Question: Boolean Expressions}
	Expressions that evaluate to \lstinline|True| or \lstinline|False|. 
	\begin{lstlisting}[language=Python, autogobble]
	"cat" < "Dog"
	\end{lstlisting}
	\vfill
	\begin{enumerate}[A]
		\item \lstinline|True|
		\item \lstinline|False|
		\item TypeError
		\item SyntaxError
	\end{enumerate}
\end{frame}

%
% Slide
%
\begin{frame}[fragile]
	\frametitle{Relational Ops on Non-numbers}
	\begin{enumerate}[A]
		\item \lstinline|ord("c")| \textrightarrow 99
			\pause
		\item \lstinline|ord("D")| \textrightarrow 68
			\pause
		\item Strings are compared based on the ASCII values of their characters.
			\pause
		\item People often normalize strings before comparisons: \lstinline|thing1.lower() < thing2.lower()|
	\end{enumerate}
\end{frame}

\section{Conditional Branching}

%
% Slide 2
%
\begin{frame}[fragile]
	\frametitle{Poll Question: If Statements}
	What does this code print?
	\begin{lstlisting}[language=Python, autogobble]
	x = 1
	if x < 7:
		print(x) 
	print(7)
	\end{lstlisting}
	\vfill
	\begin{enumerate}[A]
		\item 1
		\item 7
		\item \fbox{\parbox{0.02\textwidth}{1\\7}}
		\item SyntaxError
	\end{enumerate}
\end{frame}

%
% Slide 2
%
\begin{frame}[fragile]
	\frametitle{Poll Question: }
	What does this code print?
	\begin{lstlisting}[language=Python, autogobble]
	age = 17
	young = age < 30
	if young == true:
		print(age)
	\end{lstlisting}
	\vfill
	\begin{enumerate}[A]
		\item Nothing
		\item 17
		\item 30
		\item SyntaxError
	\end{enumerate}
\end{frame}

%
% Slide 2
%
\begin{frame}[fragile]
	\frametitle{Poll Question: If-Else Statements}
	What does this code print?
	\begin{lstlisting}[language=Python, autogobble]
	x = 2
	if x > 8:
		x = x - 2
		print(x)
	else:
		print(8)
	\end{lstlisting}
	\vfill
	\begin{enumerate}[A]
		\item 0
		\item 8
		\item \fbox{\parbox{0.02\textwidth}{0\\8}}
		\item SyntaxError
	\end{enumerate}
\end{frame}

%
% Slide 2
%
\section{More Poll Questions}
\begin{frame}[fragile]
	\frametitle{Poll Question: }
	What does \lstinline|test(7)| return?
	\begin{lstlisting}[language=Python, autogobble]
	def test(num):
		if num > 0:
			return True
		return False
	\end{lstlisting}
	\vfill
	\begin{enumerate}[A]
		\item \lstinline|True|
		\item \lstinline|False|
		\item SyntaxError
		\item \textbf{Always} \lstinline|True|
	\end{enumerate}
\end{frame}

%
% Slide 2
%
\begin{frame}[fragile]
	\frametitle{Poll Question: Constructing Conditionals}
	Which of the following will correctly report whether a student got an A, B, or something else?\\
	\vfill
	\begin{minipage}{0.32\textwidth}
		\begin{lstlisting}[language=Python, autogobble,basicstyle=\tiny,numbers=none]
		def print_grade(percent):
			if grade >= 90:
				print("You got an A!")
			elif grade >= 80:
				print("You got a B!")
			else:
				print("Other")
		\end{lstlisting}
	\end{minipage}
	\begin{minipage}{0.32\textwidth}
		\begin{lstlisting}[language=Python, autogobble,basicstyle=\tiny,numbers=none]
		def print_grade(percent):
			if grade >= 90:
				print("You got an A!")
			if grade >= 80:
				print("You got a B!")
			else:
				print("Other")
		\end{lstlisting}
	\end{minipage}
	\begin{minipage}{0.32\textwidth}
		\begin{lstlisting}[language=Python, autogobble,basicstyle=\tiny,numbers=none]
		def print_grade(percent):
			if grade >= 90:
				print("You got an A!")
			if grade >= 80:
				print("You got a B!")
			if grade < 80:
				print("Other")
		\end{lstlisting}
	\end{minipage}
	\vfill
	\begin{enumerate}[A]
		\item 1
		\item 2
		\item 3
		\item 1 and 2
		\item All of the above
	\end{enumerate}
\end{frame}


%
% Slide 2
%
\begin{frame}[fragile]
	\frametitle{Poll Question: Multi-way Branches}
	If you were choosing between 6 possibilities, what is the fewest \lstinline|elif| statements you could have?
	\vfill
	\begin{minipage}{0.49\textwidth}
		\begin{enumerate}[A]
			\item 1
			\item 2
			\item 3
			\item 4
			\item 5
		\end{enumerate}
	\end{minipage}
	\pause
	\begin{minipage}{0.49\textwidth}
		\begin{lstlisting}[language=Python, autogobble]
	if <cond>:
		...
	elif <cond>:
		...
	elif <cond>:
		...
	elif <cond>:
		...
	elif <cond>:
		...
	else:
		...
		\end{lstlisting}
	\end{minipage}
\end{frame}

%
% Slide 2
%
\begin{frame}[fragile]
	\frametitle{Poll Question: If Statements}
	What's the result of running the following code?
	\begin{lstlisting}[language=Python, autogobble]
	x = 5
  y = x == 3 or 4
	\end{lstlisting}
	\vfill
	\begin{enumerate}[A]
		\item True
		\item False
		\item SyntaxError
	\end{enumerate}
	\pause
	
\end{frame}

%
% Slide 2
%
\begin{frame}[fragile]
	\frametitle{Boolean Operators}
	\begin{enumerate}[A]
		\item Why is \lstinline|x == 3 or 4| always True?
		\item Alternatives:
			\begin{enumerate}
				\item \lstinline|x == 3 or x == 4|
				\item \lstinline|x in [3, 4]|
			\end{enumerate}
		\item Types of operators:
			\begin{enumerate}
				\item \textbf{Binary operators:} and, or
				\item \textbf{Unary Operators: } not
			\end{enumerate}
	\end{enumerate}
\end{frame}

%
% Slide 2
%
\begin{frame}[fragile]
	\frametitle{Truthy and Falsy}
	\begin{minipage}{0.69\textwidth}
		\begin{enumerate}[A]
			\item Python will convert non-Boolean types to Booleans. \\\lstinline|if "hello":|
			\item Accomplished via the use of the \lstinline|bool()| function. \\\lstinline|bool("hello")|
			\item All values are truthy (convert to \lstinline|True|) except those displayed to the right:
		\end{enumerate}
	\end{minipage}
	\begin{minipage}{0.29\textwidth}
		{\scriptsize
			\begin{itemize} 
				\item \lstinline|None|
				\item \lstinline|False|
				\item \lstinline|0|
				\item \lstinline|0.0|
				\item \lstinline|0j|
				\item \lstinline|Decimal(0)|
				\item \lstinline|Fraction(0, 1)|
				\item \lstinline|[]|
				\item \lstinline|{}|
				\item \lstinline|()|
				\item \lstinline|''|
				\item \lstinline|b''|
				\item \lstinline|set()|
				\item \lstinline|range(0)|
		\end{itemize}}
	\end{minipage}
\end{frame}

%
% Slide 2
%
\begin{frame}[fragile]
	\frametitle{Poll Question: Printing with Bools}
	What does the following segment of code produce?
	\begin{lstlisting}[language=Python, autogobble]
	print("George") and print("Boole")
	\end{lstlisting}
	\vfill
	\begin{enumerate}[A]
		\item \fbox{\parbox{0.1\textwidth}{George}}
		\item \fbox{\parbox{0.1\textwidth}{Boole}}
		\item \fbox{\parbox{0.1\textwidth}{George\\Boole}}
		\item SyntaxError
	\end{enumerate}
\end{frame}

%
% Slide 2
%
\section{Short Circuit}
\begin{frame}[fragile]
	\frametitle{Short Circuiting}
	\begin{itemize}
		\item Python is \textbf{lazy} (for efficiency reasons)
		\item It won't evaluate Boolean expressions it doesn't need to
	\end{itemize}
	\vfill
	\begin{lstlisting}[language=Python, autogobble]
	True or anything() # This is True 
	False and anything() # This is False
	\end{lstlisting}
	\vfill
	\begin{itemize}
		\item Python won't evaluate the \lstinline|anything()| part.
		\item You can use this to prevent errors from occurring in your code or having to next if statements:
	\end{itemize}
	\vfill
	\begin{lstlisting}[language=Python, autogobble]
	if (len(my_str) > 10) and (my_str[10] == 'a'):
		print("the tenth character of my string is ", my_str[10])
	\end{lstlisting}
	\vfill
\end{frame}

\end{document}
