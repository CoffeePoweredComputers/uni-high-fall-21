\documentclass{beamer}

\usepackage{graphicx}
\usepackage{textpos}
\usepackage{listings}
\usepackage{lstautogobble}

\usetheme{Madrid}
\useoutertheme{miniframes} % Alternatively: miniframes, infolines, split

% Setup the university's color pallette
\definecolor{UIUCorange}{RGB}{19, 41, 75} % UBC Blue (primary)
\definecolor{UIUCblue}{RGB}{232, 74, 39} % UBC Grey (secondary)

\definecolor{codegreen}{rgb}{0,0.6,0}
\definecolor{codegray}{rgb}{0.5,0.5,0.5}
\definecolor{codepurple}{rgb}{0.58,0,0.82}
\definecolor{backcolour}{rgb}{0.95,0.95,0.92}

\lstdefinestyle{python}{
  backgroundcolor=\color{backcolour},   
  commentstyle=\color{codegreen},
  keywordstyle=\color{magenta},
  numberstyle=\tiny\color{codegray},
  stringstyle=\color{codepurple},
  basicstyle=\ttfamily\footnotesize,
  breakatwhitespace=false,         
  belowskip=-0.5em,
  breaklines=true,                 
  captionpos=b,                    
  keepspaces=true,                 
  numbers=left,                    
  numbersep=5pt,                  
  showspaces=false,                
  showstringspaces=false,
  showtabs=false,                  
  tabsize=2
}

\lstset{style=python}

\AtBeginSection[]{
    \begin{frame}
        \vfill
        \centering
        \begin{beamercolorbox}[sep=8pt,center,shadow=true,rounded=true]{title}
            \usebeamerfont{title}\insertsectionhead\par%
        \end{beamercolorbox}
        \vfill
    \end{frame}
}
% Setup the university's color pallette
\definecolor{UIUCorange}{RGB}{19, 41, 75} % UBC Blue (primary)
\definecolor{UIUCblue}{RGB}{232, 74, 39} % UBC Grey (secondary)


\setbeamercolor{palette primary}{bg=UIUCorange,fg=white}
\setbeamercolor{palette secondary}{bg=UIUCblue,fg=white}
\setbeamercolor{palette tertiary}{bg=UIUCblue,fg=white}
\setbeamercolor{palette quaternary}{bg=UIUCblue,fg=white}
\setbeamercolor{structure}{fg=UIUCorange} % itemize, enumerate, etc
\setbeamercolor{section in toc}{fg=UIUCblue} % TOC sections

\setbeamercolor{subsection in head/foot}{bg=UIUCorange,fg=UIUCblue}
\setbeamercolor{subsection in head/foot}{bg=UIUCorange,fg=UIUCblue}

\usepackage[utf8]{inputenc}
\usepackage{graphicx}

%Information to be included in the title page:
\title{\textbf{Files Continued}}
\author{\textbf{David H Smith IV}}
\institute[\textbf{UIUC}]{\textbf{University of Illinois Urbana-Champaign}}
\date{\textbf{Tues, Oct 12 2021}}

\setbeamertemplate{title page}[default][colsep=-4bp,rounded=true]
\addtobeamertemplate{title page}{\vspace{3\baselineskip}}{}
\addtobeamertemplate{title page}{
  \begin{textblock*}{\paperwidth}(-1.0em, -1.2em)
    \includegraphics[width=\paperwidth, height=\paperheight]{imgs/uiuc.png}
  \end{textblock*} 
}{}

\begin{document}

\frame{\titlepage}

\section{Reminders}

%
% Slide 1
%
\begin{frame}
  \frametitle{Reminders}
  \begin{itemize}
    \item Last homework of the quarter is due Thursday by the end of the day.
  \end{itemize}
\end{frame}

\section{OS Review}

\begin{frame}[fragile]
  \frametitle{Poll Question: Opening a File}
  How many times will this function call iterate if placed in a for loop: \lstinline|os.walk("dev_website")|?
  \begin{minipage}{0.74\textwidth}
    \begin{lstlisting}
  dev_website
  |---index.html
  |___elements
  |   |--- about.html
  |   |--- cv.html
  |   |--- myfile.py
  |   |___ projects.html
  |__inner_dir
     |--- thing1.txt
     |___ thing2.txt
    \end{lstlisting} 
  \end{minipage}
  \begin{minipage}{0.24\textwidth}
    \begin{enumerate}[A]
      \item 1
      \item 3
      \item 10
      \item 9
    \end{enumerate}
  \end{minipage}
\end{frame}

%
% Slide 2
%
\begin{frame}[fragile]
  \frametitle{Poll Question: Opening a File}
  Imagine file.csv has four lines of text. What is the value of \lstinline|total_len| after the code has finished running?
  \begin{lstlisting}
  fo = open("file.csv")
  read1 = fo.readlines()
  read2 = fo.readlines()
  total_len = len(read1) + len(read2)
  \end{lstlisting} 
  \vfill
  \begin{enumerate}[A]
    \item 3
    \item 6
    \item 0
    \item Error
  \end{enumerate}
\end{frame}

%
% Slide 2
%
\begin{frame}[fragile]
  \frametitle{Poll Question: Opening a File}
  We are script myfile.py and we want to open mydata.csv.
  \begin{lstlisting}
  home
  |__tracertong
     |---mydata.csv
     |___dev_website
         |---index.html
         |___elements
             |___ myfile.py   <-- You are here
  \end{lstlisting} 
  \vfill
  \begin{enumerate}[A]
    \item \lstinline|fo = open("../../mydata.csv")|
    \item \lstinline|fo = open("/../../mydata.csv")|
    \item \lstinline|fo = open("/tracertong/mydata.csv")|
    \item \lstinline|fo = open("../../../mydata.csv")|
  \end{enumerate}
\end{frame}

%
% Slide 2
%
\begin{frame}[fragile]
  \frametitle{Poll Question: Reading a File}
  Imagine file.csv has four lines of text. What is the value of \lstinline|read1| after the code has finished running?
  \begin{lstlisting}
  fo = open("file.csv")
  fo.close()
  read1 = fo.readlines()
  \end{lstlisting} 
  \vfill
  \begin{enumerate}[A]
    \item 4
    \item Cannot determine from info given
    \item 0
    \item ValueError
  \end{enumerate}
\end{frame}


%
% Slide 2
%
\begin{frame}[fragile]
  \frametitle{Poll Question: Reading a File}
  Imagine file.csv has one line that says "This is a file". What is the result of the code?
  \begin{lstlisting}
  fo = open("file.csv")
  read1 = fo.read(1)
  read2 = fo.read(1)
  print(read1, read2)
  \end{lstlisting} 
  \vfill
  \begin{enumerate}[A]
    \item This is
    \item T h
    \item T T
    \item This This
    \item Error
  \end{enumerate}
\end{frame}


%
% Slide 2
%
\begin{frame}[fragile]
  \frametitle{Poll Question: Reading a File}
  Imagine file.csv has four lines of text. What is the value of \lstinline|total_len| after the code has finished running?
  \begin{lstlisting}
  fo = open("file.csv")
  read1 = fo.readlines()
  read2 = fo.readlines()
  total_len = len(read1) + len(read2)
  \end{lstlisting} 
  \vfill
  \begin{enumerate}[A]
    \item 3
    \item 6
    \item 0
    \item Error
  \end{enumerate}
\end{frame}


\section{With-As vs Open-Close}
%
% Slide 2
%
\begin{frame}[fragile]
  \frametitle{Reading from Files}
  Method 1:
  \begin{lstlisting}[language=Python, autogobble]
  file_object = open('filename')
  lines = file_object.readlines()
  for line in lines:
    print(line)
  file_object.close()
  \end{lstlisting}
  \vfill
  Method 2:
  \begin{lstlisting}[language=Python, autogobble]
  with open('filename') as inf:
    lines = inf.readlines()
    for line in lines:
      print(line)
    #automatic file close
  \end{lstlisting}
\end{frame}

%
% Slide 2
%
\begin{frame}[fragile]
  \frametitle{Writing to Files}
  Method 1:
  \begin{lstlisting}[language=Python, autogobble]
  file_object = open('filename', 'w')
  file_object.write('thing to write')
  file_objet.close()  #automatic at program end
  file_object.flush() #optional
  \end{lstlisting}
  \vfill
  Method 2:
  \begin{lstlisting}[language=Python, autogobble]
  with open('filename', 'w') as outf:
    outf.write('thing to write')
    #automatic file close
  \end{lstlisting}
\end{frame}

\section{Argument Vector (argv)}

%
% Slide 2
%
\begin{frame}[fragile]
  \frametitle{Poll Question: Argv}
  What value is printed to the screen if the file, test.py, is run as such: \lstinline|python test.py 1 2 3|?
  \begin{lstlisting}
  import sys
  print(sys.argv)
  \end{lstlisting} 
  \vfill
  \begin{enumerate}[A]
    \item 3
    \item 4
    \item 5
    \item Error
  \end{enumerate}
\end{frame}

%
% Slide 2
%
\begin{frame}[fragile]
  \frametitle{Poll Question: Argv}
  What value is printed to the screen if the file, test.py, is run as such: \lstinline|python test.py 1 2 3|?
  \begin{lstlisting}
  import sys
  print(sum(sys.argv[1:]))
  \end{lstlisting} 
  \vfill
  \begin{enumerate}[A]
    \item 6
    \item 1
    \item 5
    \item TypeError
  \end{enumerate}
\end{frame}


%
% Slide 2
%
\begin{frame}[fragile]
  \frametitle{Argument Vector (argv)}
  Example script time...
\end{frame}

\end{document}
