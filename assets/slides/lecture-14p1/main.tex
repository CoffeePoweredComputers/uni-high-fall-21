\documentclass{beamer}

\usepackage{graphicx}
\usepackage{textpos}
\usepackage{listings}
\usepackage{lstautogobble}

\usetheme{Madrid}
\useoutertheme{miniframes} % Alternatively: miniframes, infolines, split

% Setup the university's color pallette
\definecolor{UIUCorange}{RGB}{19, 41, 75} % UBC Blue (primary)
\definecolor{UIUCblue}{RGB}{232, 74, 39} % UBC Grey (secondary)

\definecolor{codegreen}{rgb}{0,0.6,0}
\definecolor{codegray}{rgb}{0.5,0.5,0.5}
\definecolor{codepurple}{rgb}{0.58,0,0.82}
\definecolor{backcolour}{rgb}{0.95,0.95,0.92}

\lstdefinestyle{python}{
    backgroundcolor=\color{backcolour},   
    commentstyle=\color{codegreen},
    keywordstyle=\color{magenta},
    numberstyle=\tiny\color{codegray},
    stringstyle=\color{codepurple},
    basicstyle=\ttfamily\footnotesize,
    breakatwhitespace=false,         
    belowskip=-0.5em,
    breaklines=true,                 
    captionpos=b,                    
    keepspaces=true,                 
    %numbers=left,                    
    numbersep=5pt,                  
    showspaces=false,                
    showstringspaces=false,
    showtabs=false,                  
    tabsize=2
}

\lstset{style=python}

\AtBeginSection[]{
    \begin{frame}
        \vfill
        \centering
        \begin{beamercolorbox}[sep=8pt,center,shadow=true,rounded=true]{title}
            \usebeamerfont{title}\insertsectionhead\par%
        \end{beamercolorbox}
        \vfill
    \end{frame}
}
% Setup the university's color pallette
\definecolor{UIUCorange}{RGB}{19, 41, 75} % UBC Blue (primary)
\definecolor{UIUCblue}{RGB}{232, 74, 39} % UBC Grey (secondary)


\setbeamercolor{palette primary}{bg=UIUCorange,fg=white}
\setbeamercolor{palette secondary}{bg=UIUCblue,fg=white}
\setbeamercolor{palette tertiary}{bg=UIUCblue,fg=white}
\setbeamercolor{palette quaternary}{bg=UIUCblue,fg=white}
\setbeamercolor{structure}{fg=UIUCorange} % itemize, enumerate, etc
\setbeamercolor{section in toc}{fg=UIUCblue} % TOC sections

\setbeamercolor{subsection in head/foot}{bg=UIUCorange,fg=UIUCblue}
\setbeamercolor{subsection in head/foot}{bg=UIUCorange,fg=UIUCblue}

\usepackage[utf8]{inputenc}
\usepackage{graphicx}

%Information to be included in the title page:
\title{\textbf{Classes}}
\author{\textbf{David H Smith IV}}
\institute[\textbf{UIUC}]{\textbf{University of Illinois Urbana-Champaign}}
\date{\textbf{Tues, Nov 30 2021}}

\setbeamertemplate{title page}[default][colsep=-4bp,rounded=true]
\addtobeamertemplate{title page}{\vspace{3\baselineskip}}{}
\addtobeamertemplate{title page}{
    \begin{textblock*}{\paperwidth}(-1.0em, -1.2em)
        \includegraphics[width=\paperwidth, height=\paperheight]{imgs/uiuc.png}
    \end{textblock*} 
}{}

\begin{document}

\frame{\titlepage}

\section{Reminders}

%
% Slide 1
%
\begin{frame}
    \frametitle{Reminders}
    \begin{itemize}
        \item Usual mix of homework (see the website)
        \item Quiz next Thursday, practice quiz will be up Saturday
        \item Practice final will be up next Tuesday.
    \end{itemize}
\end{frame}


\section{Review + New Stuff: Class Constructors}

%
% Slide 2
%
\begin{frame}[fragile]
    \frametitle{Poll Question: Constructors}
    Which of the following are valid constructor function defs?
    \begin{enumerate}
        \item \lstinline|def __init__(self):|
        \item \lstinline|def __init__(self, foo):|
        \item \lstinline|def _init_(self):|
        \item \lstinline|def __init__(self, foo, *args, **kwargs):|
        \item \lstinline|def __init__(self, foo, bar=0, *baz, **qux):|
        \item \lstinline|def __init__(self, bar=0, foo, **qux, *baz):|
    \end{enumerate}
    \vfill
    \textbf{Consider each individually in groups and I will go through and ask which are valid.}
\end{frame}

%
% Slide 2
%
\begin{frame}[fragile]
    \frametitle{Poll Question: Setting Things Up}
    What is printed in when the following code is run?
    \begin{lstlisting}[language=Python, autogobble]
    class Person:
        def __init__(self, name, age):
            myname = name
            myage = age
        def get_name(self):
            print(myname)

    p1 = Person("Dave", 22)
    print(p1.get_name())
    \end{lstlisting}
    \vfill
    \begin{enumerate}[A]
        \item NameError
        \item 22
        \item Dave
        \item self
    \end{enumerate}
\end{frame}

%
% Slide 2
%
\begin{frame}[fragile]
    \frametitle{Poll Question: Setting Things Up}
    What is printed in when the following code is run?
    \begin{lstlisting}[language=Python, autogobble]
    class Person:
        def __init__(self, name, age):
            myname = name
            myage = age
        def get_date_of_birth(current_year):
            print(current_year - self.age)

    p1 = Person("Dave", 22)
    print(p1.get_date_of_birth(2021))
    \end{lstlisting}
    \vfill
    \begin{enumerate}[A]
        \item 1999
        \item -1977
        \item NameError
        \item TypeError
    \end{enumerate}
\end{frame}

%
% Slide 2
%
\begin{frame}[fragile]
    \frametitle{Poll Question: Calling Class Functions in the Class}
    What code should replace the question marks?
    \begin{lstlisting}[language=Python, autogobble, basicstyle=\tiny]
    class People:
        def __init__(self, *people):
            self.people_list = people

        def _get_age_difference(self, p2):
            return abs(self.age - p2.age)
            
        def get_age_differences(self, person, current_year):
            for p in self.people_list:
                if p is not self:
                    n1 = self.name
                    n2 = p.name
                    diff = ??
                    print("{} and {} are {} years apart".format(n2, n1, diff))
                        
    \end{lstlisting}
    \vfill
    \begin{enumerate}[A]
        \item \lstinline|self._get__age_difference(p2)|
        \item \lstinline|_get__age_difference(p2)|
        \item \lstinline|_get__age_difference(self, p2)|
    \end{enumerate}
\end{frame}

%
% Slide 2
%
\begin{frame}[fragile]
    \frametitle{Key Takeaways}
    \begin{enumerate}[A]
        \item 
    \end{enumerate}
\end{frame}


\section{Class Interfaces, Internal Methods, and Encapsulation}

%
% Slide 2
%
\begin{frame}[fragile]
    \frametitle{Encapsulation: Private vs Public}
    \vfill
    \begin{enumerate}[A]
        \item Encapsulation is a fundamental concept in object oriented programming (OOP).
        \item Private vs Public Variables
        \begin{enumerate}[A]
            \item \textbf{Public: } The values/methods can be used outside of the class.
            \item \textbf{Private } The values/methods can't be used outside of the class.
        \end{enumerate}
    \item \textbf{Python doesn't have encapsulation!} (I have opinions on this...)
    \end{enumerate}
    \vfill
\end{frame}

%
% Slide 2
%
\begin{frame}[fragile]
    \frametitle{}
    \begin{lstlisting}[language=Python, autogobble]
    def Foo:
        _private_func(self):
            #...

        public_func(self):
            self._private_func() #Good practice
            #...

    f = Foo()
    f._private_func() #Not good practice
    f.pubic_func() #Good practice
    \end{lstlisting}
    \vfill
    \begin{enumerate}[A]
        \item Functions/variables \textit{intended} to only be used in the classes start with an "\_". Python's version of private.
        \item Functions/variables \textit{intended} to be used anywhere don't start with an "\_".
        \item \textbf{This is only by convention and has no impact on how the code runs or is interpreted.}
    \end{enumerate}
\end{frame}

\section{Overloading}

%
% Slide 2
%
\begin{frame}[fragile]
    \frametitle{Common Operators to Overload}
    \begin{lstlisting}[language=Python, autogobble, basicstyle=\tiny]
    # Example
    class Person:
        def __init__(self, name, age):
            self.name = name
            self.age = age
        def __sub__(self, other):
            return self.age - other.age

    p1 = Person("Alice", 22)
    p2 = Person("Bob", 27)
    age_difference = p2 - p1
    \end{lstlisting}
    \vfill
    \begin{minipage}{0.49\textwidth}
    \begin{itemize}
        \item \lstinline|>| \ \textrightarrow \ \lstinline|__gt__(self, other)| 
        \item \lstinline|>=| \textrightarrow \ \lstinline|__ge__(self, other)|
        \item \lstinline|<| \ \textrightarrow \ \lstinline|__lt__(self, other)|
        \item \lstinline|<=| \textrightarrow \ \lstinline|__le__(self, other)|
        \item \lstinline|==| \textrightarrow \ \lstinline|__eq__(self, other)|
    \end{itemize}
    \end{minipage}
    \hfill
    \begin{minipage}{0.49\textwidth}
    \begin{itemize}
        \item \lstinline|-| \ \textrightarrow \ \lstinline|__sub__(self, other)|
        \item \lstinline|+| \ \textrightarrow \ \lstinline|__add__(self, other)|
        \item \lstinline|*| \ \textrightarrow \ \lstinline|__mul__(self, other)|
        \item \lstinline|/| \ \textrightarrow \ \lstinline|__truediv__(self, other)|
        \item \lstinline|%| \ \textrightarrow \ \lstinline|__mod__(self, other)|
        \item \lstinline|__str__(self)|
    \end{itemize}
    \end{minipage}
\end{frame}

%
% Slide 2
%
\begin{frame}[fragile]
    \frametitle{Iteration and Subscripting Overloads }
    To define the behaviour of the class when (1) a collection are being iterated over or a (2) single instance is being unpacked use the \lstinline|__iter__(self)| function.
    \begin{lstlisting}[language=Python, autogobble]
    class Foo:
        def __init__(self, end):
            self.end = end

        def __iter__(self):
            """Commonly uses a generator"""
            c = 0
            while c < self.end:
                yield c
                c += 1

    x = Foo(10)
    for i in foo:
        print(i)
    \end{lstlisting}
\end{frame}


%
% Slide 2
%
\begin{frame}[fragile]
    \frametitle{Subscripting Overloads}
    To define the behaviour of the \lstinline|[]| operator when using variable unpacking or iteration use the \lstinline|__getitem__(self)| function.

    \begin{lstlisting}[language=Python, autogobble]
    class Foo:
        def __init__(self, start, end):
            self.start = start
            self.end = end

        def __getitem__(self, item):
            """Commonly uses a generator"""
            value = self.start + item 
            if item < value:
                return value

    x = Foo(10)
    print(x[3])
    \end{lstlisting}
\end{frame}


\end{document}
