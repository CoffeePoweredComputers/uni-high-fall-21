\documentclass{beamer}

\usepackage{graphicx}
\usepackage{textpos}
\usepackage{listings}
\usepackage{lstautogobble}

\usetheme{Madrid}
\useoutertheme{miniframes} % Alternatively: miniframes, infolines, split

% Setup the university's color pallette
\definecolor{UIUCorange}{RGB}{19, 41, 75} % UBC Blue (primary)
\definecolor{UIUCblue}{RGB}{232, 74, 39} % UBC Grey (secondary)

\definecolor{codegreen}{rgb}{0,0.6,0}
\definecolor{codegray}{rgb}{0.5,0.5,0.5}
\definecolor{codepurple}{rgb}{0.58,0,0.82}
\definecolor{backcolour}{rgb}{0.95,0.95,0.92}

\lstdefinestyle{python}{
    backgroundcolor=\color{backcolour},   
    commentstyle=\color{codegreen},
    keywordstyle=\color{magenta},
    numberstyle=\tiny\color{codegray},
    stringstyle=\color{codepurple},
    basicstyle=\ttfamily\footnotesize,
    breakatwhitespace=false,         
    belowskip=-0.5em,
    breaklines=true,                 
    captionpos=b,                    
    keepspaces=true,                 
    %numbers=left,                    
    numbersep=5pt,                  
    showspaces=false,                
    showstringspaces=false,
    showtabs=false,                  
    tabsize=2
}

\lstset{style=python}

\AtBeginSection[]{
    \begin{frame}
        \vfill
        \centering
        \begin{beamercolorbox}[sep=8pt,center,shadow=true,rounded=true]{title}
            \usebeamerfont{title}\insertsectionhead\par%
        \end{beamercolorbox}
        \vfill
    \end{frame}
}
% Setup the university's color pallette
\definecolor{UIUCorange}{RGB}{19, 41, 75} % UBC Blue (primary)
\definecolor{UIUCblue}{RGB}{232, 74, 39} % UBC Grey (secondary)


\setbeamercolor{palette primary}{bg=UIUCorange,fg=white}
\setbeamercolor{palette secondary}{bg=UIUCblue,fg=white}
\setbeamercolor{palette tertiary}{bg=UIUCblue,fg=white}
\setbeamercolor{palette quaternary}{bg=UIUCblue,fg=white}
\setbeamercolor{structure}{fg=UIUCorange} % itemize, enumerate, etc
\setbeamercolor{section in toc}{fg=UIUCblue} % TOC sections

\setbeamercolor{subsection in head/foot}{bg=UIUCorange,fg=UIUCblue}
\setbeamercolor{subsection in head/foot}{bg=UIUCorange,fg=UIUCblue}

\usepackage[utf8]{inputenc}
\usepackage{graphicx}

%Information to be included in the title page:
\title{\textbf{Classes}}
\author{\textbf{David H Smith IV}}
\institute[\textbf{UIUC}]{\textbf{University of Illinois Urbana-Champaign}}
\date{\textbf{Tues, Nov 30 2021}}

\setbeamertemplate{title page}[default][colsep=-4bp,rounded=true]
\addtobeamertemplate{title page}{\vspace{3\baselineskip}}{}
\addtobeamertemplate{title page}{
    \begin{textblock*}{\paperwidth}(-1.0em, -1.2em)
        \includegraphics[width=\paperwidth, height=\paperheight]{imgs/uiuc.png}
    \end{textblock*} 
}{}

\begin{document}

\frame{\titlepage}

\section{Reminders}

%
% Slide 1
%
\begin{frame}
    \frametitle{Reminders}
    \begin{itemize}
        \item Usual mix of homework (see the website)
        \item Lab is extended to Dec 6th
    \end{itemize}
\end{frame}

\section{The Birds Eye View}

%
% Slide 2
%
\begin{frame}[fragile]
    \frametitle{Classes, Objects, Instances, oh my!}
    \begin{enumerate}[A]
        \item \textbf{Classes:} The actual Python code that provides instructions on how to build a class (\lstinline|__init__()|), the attributes in the class, and definitions for the class functions.
            \pause
            \begin{itemize}
                \item \textbf{Class Attribute: } A value in the class that is accessible to all instances of that class.
                    \pause
                \item \textbf{Instance Attribute: } A value that is only accessible to a given instance.
                    \pause
                \item \textbf{Instance Method: } A function that is callable from within a given instance. The words `method' and `function' mean the same thing.
            \end{itemize}
            \pause
        \item \textbf{Instance:} An object that was created using a given class. We can have multiple instances of the same class.
            \pause
        \item \textbf{Object:} Just another name for the thing we instantiated.
    \end{enumerate}
\end{frame}

\section{Making a Class}

%
% Slide 2
%
\begin{frame}[fragile]
    \frametitle{Making a Class: Poll Question}
    Which of the following is the correct way of instantiating the class \lstinline|Foo|?
    \begin{lstlisting}[language=Python, autogobble]
    class Foo:
        def __init__(self):
            print("I'm a class!")
    \end{lstlisting}
    \vfill
    \begin{enumerate}[A]
        \item \lstinline|foo = Foo()|
        \item \lstinline|foo = Foo(self)|
        \item \lstinline|foo = Foo.__init__()|
        \item \lstinline|foo = __init__()|
    \end{enumerate}
\end{frame}

%
% Slide 2
%
\begin{frame}[fragile]
    \frametitle{Making a Class: Poll Question}
    What is the result of the following code?
    \begin{lstlisting}[language=Python, autogobble]
    class Foo:
        def bar(self, x, y):
            return x + y

    foo = Foo()
    x = foo.bar(1, 2)
    print(x)
    \end{lstlisting}
    \vfill
    \begin{enumerate}[A]
        \item SyntaxError
        \item NameError
        \item 3
        \item TypeError
    \end{enumerate}
\end{frame}


%
% Slide 2
%
\begin{frame}[fragile]
    \frametitle{Making a Class: \lstinline|self|}
    What is the result of the following code?
    \begin{lstlisting}[language=Python, autogobble]
    class Foo:

        def __init__(self):
            print("I'm a class!")

        def get_id(self):
            return id(self)

    foo = Foo()
    print(id(foo) == foo.get_id())
    \end{lstlisting}
    \vfill
    \begin{enumerate}[A]
        \item True
        \item False
        \item Trick question
    \end{enumerate}
\end{frame}

%
% slide 2
%
\begin{frame}[fragile]
    \frametitle{\lstinline|self| as an automatic first argument}
    Given this class...\\
    \vfill
    \begin{lstlisting}[language=python, autogobble]
    class Foo:
        def __init__(self):
            print("class created!")

        def bar(self, x, y):
            return x + y
    \end{lstlisting}
    \pause
    \vspace{0.5cm}
    This...
    \vfill
    \begin{lstlisting}[language=python, autogobble]
    f = Foo()
    x = f.bar(5, 6)
    \end{lstlisting}
    \pause
    \vspace{0.5cm}
    Is the same as this...\\
    \vfill
    \begin{lstlisting}[language=python, autogobble]
    f= Foo()
    x = Foo.get_id(f, 5, 6)
    \end{lstlisting}
\end{frame}

%
% slide 2
%
\begin{frame}[fragile]
    \frametitle{Classses in General}
    You've seen this before, you just didn't know it... Any thoughts?
    \pause
    \begin{lstlisting}[language=python, autogobble]
    lst = list()
    s = set()
    d1 = dict()
    \end{lstlisting}
    \vfill
    \begin{minipage}{0.49\textwidth}
        \pause
        These...
        \begin{enumerate}
            \item \lstinline|lst.append(x)|
            \item \lstinline|s.add(x)|
            \item \lstinline|d1.update(d2)|
        \end{enumerate}
    \end{minipage}
    \hfill
    \begin{minipage}{0.49\textwidth}
        \pause
        Are the same as these
        \begin{enumerate}
            \item \lstinline|list.append(lst, x)|
            \item \lstinline|set.add(s, x)|
            \item \lstinline|dict.update(d1, d2)|
        \end{enumerate}
    \end{minipage}
    \vfill
    \pause
    \textbf{This is more syntactic sugar brought to you by Python. The key takeaway is that \lstinline|self| is automatically added in front and refers to the object before the dot (\lstinline|.|).}
\end{frame}


%
% slide 2
%
\begin{frame}[fragile]
    \frametitle{Making Classes: Key Takeaways}

    \begin{enumerate}
        \item Classes are a list of instructions for how to instantiate an object just as functions are a list of instructions on how to perform an operation given some data.
            \pause
        \item Classes are abstract descriptions, objects are concrete and actually exist. Much like function definitions vs function calls().
            \pause
        \item \lstinline|__init__| Is called when you create a function but is never explicitly called.
            \begin{itemize}
                \item Example: \lstinline|foo = Foo()|
            \end{itemize}
            \pause
        \item Self is automatically passed in and refers to the object bound to the variable before the dot.
            \begin{itemize}
                \item Example: \lstinline|foo.call_function()|
            \end{itemize}
            \pause
        \item \lstinline|__init__| is not required. If it is not present in a class definition a default one will be provided and used to instantiate the object.
    \end{enumerate}

\end{frame}

\section{Attributes and Methods}

%
%
%
\begin{frame}[fragile]
    \frametitle{Poll Question: }
    \begin{minipage}{0.35\textwidth}
        \begin{lstlisting}[language=Python, autogobble, basicstyle=\tiny]
        class Name:
            name_count = 0
            def __init__(self, name):
                ??
                self.name = name

        n1 = Name("foo")
        n2 = Name("bar")
        n3 = Name("baz")
        \end{lstlisting}
    \end{minipage}
    \hfill
    \begin{minipage}{0.59\textwidth}
        Which of the following lines can be used to increment the class attribute count of Name instances that have been instantiated?
        \begin{itemize}
            \item \lstinline|Name.name_count += 1|
            \item \lstinline|self.name_count += 1|
            \item \lstinline|name_count += 1|
            \item \lstinline|self.name_count = Name.name_count + 1|
            \item \lstinline|Name.name_count = self.name_count + 1|
        \end{itemize}
    \end{minipage}
    \vfill
    \textbf{Consider each and then I'll go through and ask true (A) or false (B) for each.}
\end{frame}

%
%
%
\begin{frame}[fragile]
    \frametitle{Poll Question: }
    \begin{minipage}{0.49\textwidth}
        \begin{lstlisting}[language=Python, autogobble]
        class Name:
            name_count = 0
            def __init__(self, name):
                self.name_count += 1
                self.name = name

        n1 = Name("foo")
        n2 = Name("bar")
        n3 = Name("baz")
        print(n1.name_count, n2.name_count, n3.name_count)
        \end{lstlisting}
    \end{minipage}
    \hfill
    \begin{minipage}{0.49\textwidth}
        What is the output of the program on the right?
        \begin{enumerate}[A]
            \item \lstinline|1 1 1|
            \item \lstinline|1 2 3|
            \item \lstinline|3 3 3|
            \item NameError
        \end{enumerate}
    \end{minipage}
\end{frame}

%
%
%
\begin{frame}[fragile]
    \frametitle{Scoping in Python Sucks (I Hate it Very Much)}
    \begin{minipage}{0.47\textwidth}
        This...
        \begin{lstlisting}[language=Python, autogobble, basicstyle=\tiny]
        class Name:
            name_count = 0
            def __init__(self, name):
                self.name_count += 1
                self.name = name
        \end{lstlisting}
    \end{minipage}
    \hfill
    \begin{minipage}{0.52\textwidth}
        Is the same as this.
        \begin{lstlisting}[language=Python, autogobble, basicstyle=\tiny]
        class Name:
            name_count = 0
            def __init__(self, name):
                self.name_count = self.name_count + 1
                self.name = name
        \end{lstlisting}
    \end{minipage}
    \vfill
    \pause
    Looking at the one on the right:
    \begin{enumerate}
        \item Python starts by evaluating the expression on the right.
            \pause
        \item \lstinline|self.name_count + 1|:  \lstinline|self.name_count| isn't an instance attribute so it resolves to the class attribute.
            \pause
        \item \lstinline|self.name_count = 1|: \lstinline|self.name_count| doesn't exist as an instance level attribute so scope resolution decides to create a new instance attribute, thus leaving the class attribute unaffected.
    \end{enumerate}
    \pause
    \textbf{Always use the class name to change class attributes. Bad confusing things happen otherwise.}
\end{frame}

%
%
%
\begin{frame}[fragile]
    \frametitle{Poll Question: }
    \begin{minipage}{0.49\textwidth}
        \begin{lstlisting}[language=Python, autogobble]
        class Name:
            name_count = 0
            def __init__(self, name):
                Name.name_count += 1
                self.name = name

        n1 = Name("foo")
        n2 = Name("bar")
        n3 = Name("baz")
        print(??)
        \end{lstlisting}
    \end{minipage}
    \hfill
    \begin{minipage}{0.49\textwidth}
        Which of the following lines CANNOT be used to identify how many name_count were created?
        \begin{enumerate}[A]
            \item \lstinline|Name.name_count|
            \item Either \lstinline|n1.name_count| or \lstinline|n2.name_count| or \lstinline|n3.name_count|
            \item \lstinline|Name().name_count|
        \end{enumerate}
    \end{minipage}
    \vfill
    \textbf{Which should be used?}
\end{frame}


%
%
%
\begin{frame}[fragile]
    \frametitle{Poll Question: The Race Class}
    \begin{minipage}{0.69\textwidth}
        \begin{lstlisting}[language=Python, autogobble, basicstyle=\tiny]
        class Racer:

            finished_list = []

            def __init__(self, name, number):
                self.name = name
                self.number = number

            def finished(self):
                Racer.finished_list.append(self)
                print("finished in", len(Racer.finished_list))

        \end{lstlisting}
        \pause
        \vspace{0.5cm}
        What is produced by the following code?
        \vfill
        \begin{lstlisting}[language=Python, autogobble, basicstyle=\tiny]

        r1 = Racer("David", 13)
        r2 = Racer("Dipti", 142)
        print(Racer.finished_list)
        r2.finished()
        r1.finished()
        print([r.name for r in Racer.finished_list])
        \end{lstlisting}
    \end{minipage}
    \hfill
    \begin{minipage}{0.29\textwidth}
        \begin{enumerate}[A]
            \item 
                \begin{lstlisting}[autogobble, basicstyle=\tiny]
                []
                finished in 2
                finished in 1
                ['David', 'Dipti']
                \end{lstlisting}
            \item AttributeError
            \item 
                \begin{lstlisting}[autogobble, basicstyle=\tiny]
                []
                finished in 1
                finished in 2
                ['Dipti', 'David']
                \end{lstlisting}
            \item 
                \begin{lstlisting}[autogobble, basicstyle=\tiny]
                []
                finished in 1
                finished in 1
                []
                \end{lstlisting}
        \end{enumerate}
    \end{minipage}
\end{frame}

%
%
%
\begin{frame}[fragile]
    \frametitle{Poll Question: Ready-to-Go}
    \begin{minipage}{0.69\textwidth}
        \begin{lstlisting}[language=Python, autogobble, basicstyle=\tiny]
        class ReadyToGo:

                ready = 0
                instances = 0

                def __init__(self, name):
                        self.name = name
                        self.ready = False
                        ReadyToGo.instances += 1

                def set_ready(self):
                        ReadyToGo.ready += 1
                        self.ready = True
        \end{lstlisting}
        \pause
        \vspace{0.1cm}
        What is produced by the following code?
        \vfill
        \begin{lstlisting}[language=Python, autogobble, basicstyle=\tiny]
        p1 = ReadyToGo("Alice")
        p2 = ReadyToGo("Bob")
        p3 = ReadyToGo("Charlie")
        players = [p1, p2, p3]

        p1.set_ready()
        p3.set_ready()

        for player in players:
                if not player.ready:
                        print(player.name, "is not ready")
        \end{lstlisting}
    \end{minipage}
    \hfill
    \begin{minipage}{0.29\textwidth}
        \begin{enumerate}[A]
            \item 
                \begin{lstlisting}[autogobble, basicstyle=\tiny]
                Bob is not ready
                \end{lstlisting}
            \item SyntaxError
            \item NameError
            \item AttributeError
            \item 
                \begin{lstlisting}[autogobble, basicstyle=\tiny]
                Alice is not ready
                Bob is not ready
                Charlie is not ready
                \end{lstlisting}
        \end{enumerate}
    \end{minipage}
\end{frame}

\section{Modifying Attributes after Instantiation}
%
%
%
\begin{frame}[fragile]
    \frametitle{Poll Question: }
    \begin{minipage}{0.69\textwidth}
        What is produced by the following code?
        \begin{lstlisting}[language=Python, autogobble, basicstyle=\tiny]
        class Name:
            name_count = 0
            def __init__(self, name):
                self.name = name
                Name.name_count += 1

        foo = Name("foo")
        bar = Name("bar")
        foo.name = "Fred"
        print(bar.name, foo.name)
        \end{lstlisting}
    \end{minipage}
    \hfill
    \begin{minipage}{0.29\textwidth}
        \begin{enumerate}[A]
            \item AttributeError
            \item NameError
            \item \lstinline|bar foo|
            \item \lstinline|bar Fred|
        \end{enumerate}
    \end{minipage}
\end{frame}

%
%
%
\begin{frame}[fragile]
    \frametitle{Poll Question: }
    \begin{minipage}{0.69\textwidth}
        What is produced by the following code?
        \begin{lstlisting}[language=Python, autogobble, basicstyle=\tiny]
        class Name:
            name_count = 0
            def __init__(self, name):
                self.name = name
                Name.name_count += 1

        foo = Name("foo")
        bar = Name("bar")
        Name.name_count = 1000
        print(bar.name_count)
        \end{lstlisting}
    \end{minipage}
    \hfill
    \begin{minipage}{0.29\textwidth}
        \begin{enumerate}[A]
            \item AttributeError
            \item NameError
            \item 1000
            \item 2
        \end{enumerate}
    \end{minipage}
\end{frame}

%
%
%
\begin{frame}[fragile]
    \frametitle{Poll Question: }
    \begin{minipage}{0.69\textwidth}
        What is produced by the following code?
        \begin{lstlisting}[language=Python, autogobble, basicstyle=\tiny]
        class Name:
            name_count = 0
            def __init__(self, name):
                self.name = name
                Name.name_count += 1

        foo = Name("foo")
        bar = Name("bar")
        foo.name_count = 1000
        print(bar.name_count)
        \end{lstlisting}
    \end{minipage}
    \hfill
    \begin{minipage}{0.29\textwidth}
        \begin{enumerate}[A]
            \item AttributeError
            \item NameError
            \item 1000
            \item 2
        \end{enumerate}
    \end{minipage}
\end{frame}

%
% slide 2
%
\begin{frame}[fragile]
    \frametitle{Key Takeaways}
    \begin{enumerate}
        \item How to reference attributes best practices:
            \begin{itemize}
                \item Reference class attributes using the class name:
                    \vspace{-0.2cm}
                    \begin{lstlisting}[language=python, autogobble, basicstyle=\tiny]
                class Foo:
                    count = 0
                    def __init__(self):
                        Foo.count += 1
                    \end{lstlisting}
                    \vfill
                \item Reference instance attributes and methods using self \textit{when inside the class}.
                    \vspace{-0.2cm}
                    \begin{lstlisting}[language=python, autogobble, basicstyle=\tiny]
                class Foo:
                    def __init__(self, name):
                        self.name = name
                    def print_name(self):
                        print(self.name)
                    \end{lstlisting}
                    \vfill
                \item Reference instance attributes and methods using instance's variable name \textit{when outside the class}.
                    \vspace{-0.2cm}
                    \begin{lstlisting}[language=python, autogobble, basicstyle=\tiny]
                class Foo:
                    def __init__(self, name):
                        self.name = name
                x = Foo("bar")
                print(x.name)
                    \end{lstlisting}
                    \vfill
            \end{itemize}
        \item \lstinline|self| must be the first parameter in every instance methods arguments.
    \end{enumerate}
\end{frame}

\end{document}
