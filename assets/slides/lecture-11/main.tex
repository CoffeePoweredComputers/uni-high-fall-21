\documentclass{beamer}

\usepackage{graphicx}
\usepackage{textpos}
\usepackage{listings}
\usepackage{lstautogobble}

\usetheme{Madrid}
\useoutertheme{miniframes} % Alternatively: miniframes, infolines, split

% Setup the university's color pallette
\definecolor{UIUCorange}{RGB}{19, 41, 75} % UBC Blue (primary)
\definecolor{UIUCblue}{RGB}{232, 74, 39} % UBC Grey (secondary)

\definecolor{codegreen}{rgb}{0,0.6,0}
\definecolor{codegray}{rgb}{0.5,0.5,0.5}
\definecolor{codepurple}{rgb}{0.58,0,0.82}
\definecolor{backcolour}{rgb}{0.95,0.95,0.92}

\lstdefinestyle{python}{
  backgroundcolor=\color{backcolour},   
  commentstyle=\color{codegreen},
  keywordstyle=\color{magenta},
  numberstyle=\tiny\color{codegray},
  stringstyle=\color{codepurple},
  basicstyle=\ttfamily\footnotesize,
  breakatwhitespace=false,         
  belowskip=-0.5em,
  breaklines=true,                 
  captionpos=b,                    
  keepspaces=true,                 
  %numbers=left,                    
  numbersep=5pt,                  
  showspaces=false,                
  showstringspaces=false,
  showtabs=false,                  
  tabsize=2
}

\lstset{style=python}

\AtBeginSection[]{
    \begin{frame}
        \vfill
        \centering
        \begin{beamercolorbox}[sep=8pt,center,shadow=true,rounded=true]{title}
            \usebeamerfont{title}\insertsectionhead\par%
        \end{beamercolorbox}
        \vfill
    \end{frame}
}
% Setup the university's color pallette
\definecolor{UIUCorange}{RGB}{19, 41, 75} % UBC Blue (primary)
\definecolor{UIUCblue}{RGB}{232, 74, 39} % UBC Grey (secondary)


\setbeamercolor{palette primary}{bg=UIUCorange,fg=white}
\setbeamercolor{palette secondary}{bg=UIUCblue,fg=white}
\setbeamercolor{palette tertiary}{bg=UIUCblue,fg=white}
\setbeamercolor{palette quaternary}{bg=UIUCblue,fg=white}
\setbeamercolor{structure}{fg=UIUCorange} % itemize, enumerate, etc
\setbeamercolor{section in toc}{fg=UIUCblue} % TOC sections

\setbeamercolor{subsection in head/foot}{bg=UIUCorange,fg=UIUCblue}
\setbeamercolor{subsection in head/foot}{bg=UIUCorange,fg=UIUCblue}

\usepackage[utf8]{inputenc}
\usepackage{graphicx}

%Information to be included in the title page:
\title{\textbf{Adv. Dictionaries}}
\author{\textbf{David H Smith IV}}
\institute[\textbf{UIUC}]{\textbf{University of Illinois Urbana-Champaign}}
\date{\textbf{Tues, Nov 02 2021}}

\setbeamertemplate{title page}[default][colsep=-4bp,rounded=true]
\addtobeamertemplate{title page}{\vspace{3\baselineskip}}{}
\addtobeamertemplate{title page}{
  \begin{textblock*}{\paperwidth}(-1.0em, -1.2em)
    \includegraphics[width=\paperwidth, height=\paperheight]{imgs/uiuc.png}
  \end{textblock*} 
}{}

\begin{document}

\frame{\titlepage}

\section{Reminders}

%
% Slide 1
%
\begin{frame}
  \frametitle{Reminders}
  \begin{itemize}
    \item Checkpoint 1 was due last night. Please be sure to push your changes.
    \item Quiz 3 is Thursday
    \item Homework is to attempt the practice quiz (50 pts)
    \item Today: Dictionaries and Checkpoint 2 of the game of life.
  \end{itemize}
\end{frame}


\section{Dictionaries}

%
% Slide 2
%
\begin{frame}[fragile]
  \frametitle{In Review} 
  \vfill
  \begin{minipage}{0.60\textwidth}
    \begin{itemize}
      \item Consists of \lstinline|key:value| pairs.
      \item Why do we care? Tracking relationships between things. 
    \end{itemize}
  \end{minipage}
  \begin{minipage}{0.39\textwidth} 
    \begin{lstlisting}[language=Python, autogobble]
  name_map = {
    "dhsmith2" : {
      "first" : "David",
      "second" : "Smith"
    },
    "mflwr" : {
      "first" : "Max",
      "second" : "Fowler"
    }
  }\end{lstlisting}
  \end{minipage}
  \vfill
  \begin{lstlisting}[language=Python, autogobble]
    def informal_email(names_dict, netid):
      email = "Dear {0}, I wanted you to know ..."
      return email.format(names_dict[netid]['first'])

    email_text = informal_email(name_map, "dhsmith2")
  \end{lstlisting}
\end{frame}

%
% Slide 2
%
\section{Computing a Histogram}
\begin{frame}[fragile]
  \frametitle{Poll Question: Dictionaries}
  What is the value of \lstinline|x| after the following function is called?
  \begin{lstlisting}[language=Python, autogobble]
  def get_item_counts(some_list):
    counts = {}
    for item in some_list:
        counts[item] += 1

  x = get_item_counts(["This", "This", "This", "Is", "A"])
  \end{lstlisting}
  \vfill
  \begin{enumerate}
    \item \lstinline|{"This": 3, "Is": 1, "A": 1}|
    \item \lstinline|{This: 3, Is: 1, A: 1}|
    \item \lstinline|{"This": "3", "Is": "1", "A": "1"}|
    \item KeyError
  \end{enumerate}
  \pause
  Why, and how do we fix this?
\end{frame}

%
% Slide 2
%
\begin{frame}[fragile]
  \frametitle{Dictionaries: Computing a Histogram}
  Creating a count map of items in a collection is a common dictionary pattern:
  \begin{lstlisting}[language=Python, autogobble]
  def get_item_counts(some_list):
    counts = {}
    for item in some_list:
      if item not in counts:
        counts[item] = 1
      else:
        counts[item] += 1
  \end{lstlisting}
\end{frame}

%
% Slide 2
%
\section{Key, Item, Value Functions}
\begin{frame}[fragile]
  \frametitle{Poll Question: Dictionary Functions}
  What will be printed to the screen after the following has run?
  \begin{lstlisting}[language=Python, autogobble]
  dict_1 = {"foo": 5, "bar": 10, "baz": 12}
  for i in keys(dict_1):
    print(i, end=" ")
  \end{lstlisting}
  \vfill
  \begin{enumerate}[A]
    \item foo bar baz
    \item 5 10 12 
    \item NameError
    \item SyntaxError
  \end{enumerate}
\end{frame}

%
% Slide 2
%
\begin{frame}[fragile]
  \frametitle{Poll Question: Dictionary Functions}
  What will be printed to the screen after the following has run?
  \begin{lstlisting}[language=Python, autogobble]
  dict_1 = {"foo": 5, "bar": 10, "baz": 12}
  for i in dict_1.keys():
    print(i, end=" ")
  \end{lstlisting}
  \vfill
  \begin{enumerate}[A]
    \item foo bar baz
    \item 5 10 12 
    \item NameError
    \item SyntaxError
  \end{enumerate}
\end{frame}

%
% Slide 2
%
\begin{frame}[fragile]
  \frametitle{Poll Question: Dictionary Functions}
  What will be printed to the screen after the following has run?
  \begin{lstlisting}[language=Python, autogobble]
  dict_1 = {"foo": 5, "bar": 10, "baz": 12}
  for i in dict_1.keys():
    print(i, end=" ")
  \end{lstlisting}
  \vfill
  \begin{enumerate}[A]
    \item foo bar baz
    \item 5 10 12 
    \item NameError
    \item SyntaxError
  \end{enumerate}
\end{frame}

%
% Slide 2
%
\begin{frame}[fragile]
  \frametitle{Poll Question: Dictionary Functions}
  What will be printed to the screen after the following has run?
  \begin{lstlisting}[language=Python, autogobble]
  dict_1 = {"foo": 5, "bar": 10, "baz": 12}
  for i in dict_1.items():
    print(i, end=" ")
  \end{lstlisting}
  \vfill
  \begin{enumerate}[A]
    \item ('foo', 5) ('bar', 3) ('baz', 10)
    \item (5, 'foo') (3, 'bar') (10, 'baz')
    \item NameError
    \item Something else...?
  \end{enumerate}
\end{frame}

%
% Slide 2
%
\begin{frame}[fragile]
  \frametitle{Poll Question: Dictionary Functions}
  What will be printed to the screen after the following has run?
  \begin{lstlisting}[language=Python, autogobble]
  dict_1 = {"foo": 5, "bar": 10, "baz": 12}
  all_keys = []
  total_val = 0
  for foo, bar in dict_1.items():
    all_keys.append(foo)
    total_val += bar
  print(all_keys, total_val)
  \end{lstlisting}
  \vfill
  \begin{enumerate}[A]
    \item \lstinline|[5, 10, 12] "foobarbaz"|
    \item \lstinline|["foo", "bar", "baz"] 27|
    \item TypeError
    \item Something else...?
  \end{enumerate}
\end{frame}

%
% Slide 2
%
\begin{frame}[fragile]
  \frametitle{Poll Question: Dictionary Functions}
  What will be printed to the screen after the following has run?
  \begin{lstlisting}[language=Python, autogobble]
  dict_1 = {"foo": 5, "bar": 10, "baz": 12}
  for i in dict_1.values():
    print(i, end=" ")
  \end{lstlisting}
  \vfill
  \begin{enumerate}[A]
    \item foo bar baz
    \item 5 5 5
    \item 5 10 12
    \item Error 
  \end{enumerate}
\end{frame}

%
% Slide 2
%
\begin{frame}[fragile]
  \frametitle{Dictionary Functions}
  Functions for iteration:
  \begin{enumerate}
    \item \lstinline|dict.items()| \textrightarrow \ Generates tuples of all of the key value pairs in the dictionary.
    \item \lstinline|dict.keys()| \textrightarrow \ Generates all of the keys in the dictionary.
    \item \lstinline|dict.values()| \textrightarrow \ Generates all of the values in the dictionary.
  \end{enumerate}
  \vfill
  Functions for modification:
  \begin{enumerate}
    \item \lstinline|dict.clear()| \textrightarrow \ Clears all the key value pairs from the dictionary.
    \item \lstinline|dict.get(key, default)| \textrightarrow \ Tries to lookup the value associated with a key and gives default if key not found.
    \item \lstinline|dict1.update(dict2)| \textrightarrow \ Merges the key:value pairs from dict1 into dict2. 
    \item \lstinline|dict.pop(key, default)| \textrightarrow \ Removes the key:value pair, returns the value, default if key not found.
  \end{enumerate}
\end{frame}

\end{document}
