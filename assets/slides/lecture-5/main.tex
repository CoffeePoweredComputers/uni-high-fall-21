\documentclass{beamer}

\usepackage{graphicx}
\usepackage{textpos}
\usepackage{listings}
\usepackage{lstautogobble}

\usetheme{Madrid}
\useoutertheme{miniframes} % Alternatively: miniframes, infolines, split

% Setup the university's color pallette
\definecolor{UIUCorange}{RGB}{19, 41, 75} % UBC Blue (primary)
\definecolor{UIUCblue}{RGB}{232, 74, 39} % UBC Grey (secondary)

\definecolor{codegreen}{rgb}{0,0.6,0}
\definecolor{codegray}{rgb}{0.5,0.5,0.5}
\definecolor{codepurple}{rgb}{0.58,0,0.82}
\definecolor{backcolour}{rgb}{0.95,0.95,0.92}

\lstdefinestyle{python}{
  backgroundcolor=\color{backcolour},   
  commentstyle=\color{codegreen},
  keywordstyle=\color{magenta},
  numberstyle=\tiny\color{codegray},
  stringstyle=\color{codepurple},
  basicstyle=\ttfamily\footnotesize,
  breakatwhitespace=false,         
  belowskip=-0.5em,
  breaklines=true,                 
  captionpos=b,                    
  keepspaces=true,                 
  numbers=left,                    
  numbersep=5pt,                  
  showspaces=false,                
  showstringspaces=false,
  showtabs=false,                  
  tabsize=2
}

\lstset{style=python}

\AtBeginSection[]{
    \begin{frame}
        \vfill
        \centering
        \begin{beamercolorbox}[sep=8pt,center,shadow=true,rounded=true]{title}
            \usebeamerfont{title}\insertsectionhead\par%
        \end{beamercolorbox}
        \vfill
    \end{frame}
}


\setbeamercolor{palette primary}{bg=UIUCorange,fg=white}
\setbeamercolor{palette secondary}{bg=UIUCblue,fg=white}
\setbeamercolor{palette tertiary}{bg=UIUCblue,fg=white}
\setbeamercolor{palette quaternary}{bg=UIUCblue,fg=white}
\setbeamercolor{structure}{fg=UIUCorange} % itemize, enumerate, etc
\setbeamercolor{section in toc}{fg=UIUCblue} % TOC sections

\setbeamercolor{subsection in head/foot}{bg=UIUCorange,fg=UIUCblue}
\setbeamercolor{subsection in head/foot}{bg=UIUCorange,fg=UIUCblue}

\usepackage[utf8]{inputenc}
\usepackage{graphicx}


%Information to be included in the title page:
\title{\textbf{Topic 4: Functions}}
\author{\textbf{David H Smith IV}}
\institute[\textbf{UIUC}]{\textbf{University of Illinois Urbana-Champaign}}
\date{\textbf{Mon, June 28 2021}}

\setbeamertemplate{title page}[default][colsep=-4bp,rounded=true]
\addtobeamertemplate{title page}{\vspace{3\baselineskip}}{}
\addtobeamertemplate{title page}{
  \begin{textblock*}{\paperwidth}(-1.0em, -1.2em)
    \includegraphics[width=\paperwidth, height=\paperheight]{imgs/uiuc.png}
  \end{textblock*} 
}{}

\begin{document}

\frame{\titlepage}

\section{Announcements}

%
% Slide 1
%
\begin{frame}
  \frametitle{Weekly Reminders}
  \begin{enumerate}[A]
    \item Don't forget to register with the CBTF to take the quiz this week.
    \item Homework 3 is posted.
    \item Make sure you're doing the correct post-readings/participation activities.
    \item Slightly different schedule for 4th of July weekend.
  \end{enumerate}
\end{frame}


\section{Poll Questions}
%
% Slide 2
%
\begin{frame}[fragile]
  \frametitle{Poll Question: Calling Functions}
  What is the result of running the following code?
  \begin{lstlisting}[language=Python, autogobble]
  def test():
    print('first')

  print('second')
  test()
  \end{lstlisting}
  \vfill
  \begin{enumerate}[A]
    \item SyntaxError
    \item \fbox{\parbox{0.1\textwidth}{\lstinline|first|}}
    \item \fbox{\parbox{0.1\textwidth}{\lstinline|first|\\\lstinline|second|}}
    \item \fbox{\parbox{0.15\textwidth}{\lstinline|\ \ \ \ first|\\\lstinline|second|}}
    \item \fbox{\parbox{0.1\textwidth}{\lstinline|second|\\\lstinline|first|}} %correct
  \end{enumerate}
\end{frame}

%
% Slide 2
%
\begin{frame}[fragile]
  \centering
  \begin{enumerate}[A]
    \item In Python indentation is both \textbf{syntactic} and \textbf{semantic}.
    \item These three programs are all different even through they have the same text
  \end{enumerate}
  \vfill
  \begin{minipage}{0.29\textwidth}
    \centering
    \begin{lstlisting}[language=Python, autogobble]
    def test():
    print('first')
    print('second')

    test()
    \end{lstlisting}
  \end{minipage}
  \hfill
  \begin{minipage}{0.29\textwidth}
    \centering
    \begin{lstlisting}[language=Python, autogobble]
    def test():
      print('first')
    print('second')

    test()
    \end{lstlisting}
  \end{minipage}
  \hfill
  \begin{minipage}{0.29\textwidth}
    \centering
    \begin{lstlisting}[language=Python, autogobble]
    def test():
      print('first')
      print('second')

    test()
    \end{lstlisting}
  \end{minipage}
\end{frame}

%
% Slide 2
%
\begin{frame}[fragile]
  \frametitle{Poll Question: Function Parameters}
  What value is printed?
  \begin{lstlisting}[language=Python, autogobble]
  def do_thing(v1, v2, v3):
    a = v2
    b = v1 + 1
    print(a * b)

  do_thing(3, 2, 0)
  \end{lstlisting}
  \vfill
  \begin{enumerate}[A]
    \item 0
    \item 6
    \item 8 %correct
    \item 9
  \end{enumerate}
\end{frame}

%
% Slide 2
%
\begin{frame}[fragile]
  \frametitle{Poll Question: Function Scope}
  What value is printed?
  \begin{lstlisting}[language=Python, autogobble]
  def do_thing(var1):
    var1 = 2

  var1 = 1
  do_thing(var1)
  print(var1)
  \end{lstlisting}
  \vfill
  \begin{enumerate}[A]
    \item 0
    \item 1 %correct
    \item 2
    \item SyntaxError
  \end{enumerate}
\end{frame}


%
% Slide 2
%
\begin{frame}[fragile]
  \frametitle{Poll Question: Function Scope}
  What value is printed?
  \begin{lstlisting}[language=Python, autogobble]
  def do_thing(var1):
    var1.append(4)

  var1 = [1, 2, 3]
  do_thing(var1)
  print(len(var1))
  \end{lstlisting}
  \vfill
  \begin{enumerate}[A]
    \item 0
    \item 3
    \item 4 %correct
    \item SyntaxError
  \end{enumerate}
\end{frame}

%
% Slide 2
%
\begin{frame}[fragile]
  \frametitle{Poll Questions: Function Scope}
  How many values does the following function return?
  \begin{lstlisting}[language=Python, autogobble]
  def do_thing(var1):
    var1 = [1, 2, 3, 4]

  var1 = [1, 2, 3]
  do_thing(var1)
  print(len(var1))
  \end{lstlisting}
  \vfill
  \begin{enumerate}[A]
    \item 0
    \item 3 %correct
    \item 4
    \item TypeError
    \item SyntaxError
  \end{enumerate}
\end{frame}

%
% Slide 2
%
\begin{frame}[fragile]
  \frametitle{Poll Questions: Making Functions}
  \begin{minipage}{0.29\textwidth}
    \centering
    \begin{lstlisting}[language=Python, autogobble]
    def f1():
      return 5
    \end{lstlisting}
  \end{minipage}
  \hfill
  \begin{minipage}{0.29\textwidth}
    \centering
    \begin{lstlisting}[language=Python, autogobble]
    def f2():
      print(5)
    \end{lstlisting}
  \end{minipage}
  \hfill
  \begin{minipage}{0.29\textwidth}
    \centering
    \begin{lstlisting}[language=Python, autogobble]
    def f3():
      return print(5)
    \end{lstlisting}
  \end{minipage}
  \vfill
  Considering the previous functions, which of the following assigns \lstinline|x| to 5?
  \centering
  \begin{enumerate}[A]
    \item \lstinline|x = f1()|
    \item \lstinline|x = f2()|
    \item \lstinline|x = f3()|
    \item All of the above
    \item None of the above
  \end{enumerate}
\end{frame}

%
% Slide 2
%
\begin{frame}[fragile]
  \frametitle{Poll Questions: Nesting Functions}
  Given the following function, what value is returned by \lstinline|f(f(2))|?
  \begin{lstlisting}[language=Python, autogobble]
  def f(x):
    return 3 * x
  \end{lstlisting}
  \vfill
  \begin{enumerate}[A]
    \item 3
    \item 6
    \item 9
    \item 12
    \item 18
  \end{enumerate}
\end{frame}

%
% Slide 2
%
\begin{frame}[fragile]
  \frametitle{Poll Questions: Making Functions}
  What bugs are in the following code?
  \begin{lstlisting}[language=Python, autogobble]
  def add_one(x):
  return x + 1

  x = 2
  x = x + add_one(x)
  \end{lstlisting}
  \vfill
  \begin{enumerate}[A]
    \item No bugs. The code is fine.
    \item The function body is not indented.
    \item We use \lstinline|x| as both a parameter and a variable, but we are not allowed to do that.
    \item B and C
  \end{enumerate}
\end{frame}

%
% Slide 2
%
\begin{frame}[fragile]
  \frametitle{Poll Question: Printing in Functions} 
  What is the value of z after the following code has run?
  \begin{lstlisting}[language=Python, autogobble]
  def add_one():
    x = 5 * 7
    print(x)

  z = add_one()
  \end{lstlisting}
  \vfill
  \begin{enumerate}[A]
    \item \lstinline|35|
    \item \lstinline|"35"|
    \item \lstinline|None|
    \item SyntaxError
  \end{enumerate}
\end{frame}

%
% Slide 2
%
\begin{frame}[fragile]
  \frametitle{Poll Question: Return Count} 
  How many values does the following function return?
  \begin{lstlisting}[language=Python, autogobble]
  def return_stuff():
    return "hello"
    return "world"
    return "foo"
    return "bar"
  \end{lstlisting}
  \vfill
  \begin{enumerate}[A]
    \item 1
    \item 2
    \item 3
    \item 4
  \end{enumerate}
\end{frame}

%
% Slide 2
%
\begin{frame}
  \frametitle{Key Takeaways}
  Most important takeaways from today's lecture:
  \begin{enumerate}[A]
    \item \lstinline|print()| is not the same as \lstinline|return|
      \begin{enumerate}[A]
        \item  \lstinline|print()| \textrightarrow Prints to the monitor. It does not give you a value you can work with.
        \item  \lstinline|return| \textrightarrow Isn't used to print things to the screen. It's used to give data back after a function has finished doing stuff.
      \end{enumerate}
    \item A function only returns \textit{\textbf{once}}. No matter how many return statements you put in the program the only one that matters is the first one that's reached.
  \end{enumerate}
\end{frame}

\section{Reminders}
%
% Slide 1
%
\begin{frame}
  \frametitle{Weekly Reminders}
  \begin{enumerate}[A]
    \item Don't forget to register with the CBTF to take the quiz this week.
    \item Homework 3 is posted and due this Friday at 11PM (CST).
    \item Make sure you're doing the correct post-readings/participation activities.
    \item Slightly different schedule for 4th of July weekend.
  \end{enumerate}
\end{frame}


\end{document}
