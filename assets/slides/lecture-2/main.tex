\documentclass{beamer}

\usepackage{graphicx}
\usepackage{textpos}
\usepackage{listings}
\usepackage{tikz}
\usepackage{xcolor}
\usepackage{listofitems}

\definecolor{codegreen}{rgb}{0,0.6,0}
\definecolor{codegray}{rgb}{0.5,0.5,0.5}
\definecolor{codepurple}{rgb}{0.58,0,0.82}
\definecolor{backcolour}{rgb}{0.95,0.95,0.92}


\lstdefinestyle{py}{
  backgroundcolor=\color{backcolour},   
  commentstyle=\color{codegreen},
  keywordstyle=\color{magenta},
  numberstyle=\tiny\color{codegray},
  stringstyle=\color{codepurple},
  basicstyle=\ttfamily\footnotesize,
  breakatwhitespace=false,         
  breaklines=true,                 
  captionpos=b,                    
  keepspaces=true,                 
  numbers=left,                    
  numbersep=5pt,                  
  showspaces=false,                
  showstringspaces=false,
  showtabs=false,                  
  tabsize=2
}

\lstset{style=py}

\usetheme{Madrid}
\useoutertheme{miniframes} % Alternatively: miniframes, infolines, split

% Setup the university's color pallette
\definecolor{UIUCorange}{RGB}{19, 41, 75} 
\definecolor{UIUCblue}{RGB}{232, 74, 39} 


\setbeamercolor{palette primary}{bg=UIUCorange,fg=white}
\setbeamercolor{palette secondary}{bg=UIUCblue,fg=white}
\setbeamercolor{palette tertiary}{bg=UIUCblue,fg=white}
\setbeamercolor{palette quaternary}{bg=UIUCblue,fg=white}
\setbeamercolor{structure}{fg=UIUCorange} % itemize, enumerate, etc
\setbeamercolor{section in toc}{fg=UIUCblue} % TOC sections

\setbeamercolor{subsection in head/foot}{bg=UIUCorange,fg=UIUCblue}
\setbeamercolor{subsection in head/foot}{bg=UIUCorange,fg=UIUCblue}

\usepackage[utf8]{inputenc}
\usepackage{graphicx}

\AtBeginSection[]{
    \begin{frame}
        \vfill
        \centering
        \begin{beamercolorbox}[sep=8pt,center,shadow=true,rounded=true]{title}
            \usebeamerfont{title}\insertsectionhead\par%
        \end{beamercolorbox}
        \vfill
    \end{frame}
}


%Information to be included in the title page:
\title{\textbf{Topic 1: Intro to Python}}
\author{\textbf{David H Smith IV}}
\institute[\textbf{UIUC}]{\textbf{University of Illinois Urbana-Champaign}}
\date{\textbf{Wed, June 16 2021}}

\setbeamertemplate{title page}[default][colsep=-4bp,rounded=true]
\addtobeamertemplate{title page}{\vspace{3\baselineskip}}{}
\addtobeamertemplate{title page}{
  \begin{textblock*}{\paperwidth}(-1.0em, -1.2em)
    \includegraphics[width=\paperwidth, height=\paperheight]{imgs/uiuc.png}
  \end{textblock*} 
}{}



\begin{document}

\frame{\titlepage}

\begin{frame}
  \frametitle{Updates}
  \begin{enumerate}
    \item \textbf{Homework 1:} Posted and due this Friday.
    \item \textbf{Topic 2 (Part 1) and Post-reading}: Posted and due this Friday.
    \item Feel free to drop by office hours if you have any questions.
    \item As a reminder, I encourage you all to work together to complete the homeworks. Don't share exact answers but feel free to discuss things at a high level.
  \end{enumerate}
\end{frame}

\section{A Brief Review}

\begin{frame}
  \frametitle{Computability}
  \begin{itemize}
    \item Complexity $\neq$ Non-computable (undecidable)
      \pause
    \item \textbf{The Halting Problem: }Given a description of a Turing Machine and it's initial input (the paper strip), determine whether the program, when executed on this input, ever halts.
      \pause
    \item There is no \textit{general solution} to the halting problem.
  \end{itemize}
\end{frame}

\begin{frame}
  \frametitle{Anatomy of a Function Call}
  \centering
  \includegraphics[width=\textwidth]{./imgs/parameters.png}
\end{frame}


\begin{frame}
  \frametitle{Input and Print}
  \begin{itemize}
    \item \underline{\textbf{The Input Function}}: A builtin function that gets a string from the user.
      \pause
      \begin{itemize}
        \item \textbf{input()} \textrightarrow Doesn't give a prompt.
          \pause
        \item \textbf{input(`A test input: ')} \textrightarrow Will output the message \textit{"A test input:"} to the screen and let the user type their input in after it.
          \pause
      \end{itemize}
    \item \underline{\textbf{The Print Function}}: A builtin function that takes a string as a parameter (in between the parentheses) and outputs that string
      \pause
      \begin{itemize}
        \item \textbf{print(`Hello, World!')} \textrightarrow Will output the `Hello, World!''. 
      \end{itemize}
  \end{itemize}
\end{frame}

%
% Slide 
%
\begin{frame}
  \frametitle{Data Types in Python}
  \begin{itemize}
    \item Two types we'll need to know now:
      \begin{itemize}
        \item \textbf{Strings: } A long list of characters accompanied by surrounding quotes (e.g., `Hello, CS 105!').
        \item \textbf{Integers: } Whole numbers, both positive and negative.
      \end{itemize}
    \item You can check the type of a variable or expression with the \lstinline{type()} function.
      \pause
    \item You can convert between them with the \lstinline{str()} and \lstinline{int()} functions.
      \pause
    \item It's important to keep the type of your variables in mind when programming.
      \pause
  \end{itemize}
\end{frame}




\section{Poll Questions}

%
% 
%
\begin{frame}[fragile]
  \frametitle{Poll Question: Input and Print}
  Assuming there is a variable value with the value 7, which of the following statements prints: count = 7.
  \begin{enumerate}[A]
    \item \lstinline{print("count = " value)}
    \item \lstinline{print("count = ", end="value")}
    \item \lstinline{print("count = ", value)}
    \item \lstinline{print("count = $value")}
  \end{enumerate}
\end{frame}

%
%
%
\begin{frame}
  \frametitle{Poll Question: Addition Operation}
  In Python x + y is:
  \begin{enumerate}[A]
    \item an assignment
    \item a statement
    \item a variable
    \item an expression
  \end{enumerate}
\end{frame}

\section{Errors}

%
%
%
\begin{frame}[fragile]
  \frametitle{Poll Question: Errors}
  What happens if I type the following code into a \textbf{brand new} Python interpreter:
  \begin{lstlisting}[language=Python]
  x + y \end{lstlisting}
  \begin{enumerate}[A]
    \item Nothing
    \item SyntaxError
    \item NameError
    \item ValueError
  \end{enumerate}
\end{frame}

%
%
%
\begin{frame}[fragile]
  \frametitle{Poll Question: Error}
  What happens if I type the following code into a Python interpreter?
  \begin{lstlisting}[language=Python]
  print(input("Enter a number: "))\end{lstlisting}
  \begin{enumerate}[A]
    \item No error
    \item SyntaxError
    \item NameError
    \item TypeError
  \end{enumerate}
\end{frame}

%
%
%
\begin{frame}[fragile]
  \frametitle{Poll Question: Errors}
  What happens if I type the following code into a Python interpreter?
  \begin{lstlisting}[language=Python]
  value = input("Input your favorite number!\n")
  print("Your new favorite number is", value + 1)\end{lstlisting}
  \begin{enumerate}[A]
    \item No error
    \item SyntaxError
    \item ValueError
    \item TypeError
  \end{enumerate}
\end{frame}

%
%
%
\begin{frame}[fragile]
  \frametitle{Poll Question: The Print Function}
  What type of error will Python produce when this segment of code is run?
  \begin{lstlisting}[language=Python]
  x = "Hello, world'
  y = 5
  foo = x + y\end{lstlisting}
  \begin{enumerate}[A]
    \item SyntaxError
    \item NameError
    \item ValueError
    \item TypeError
  \end{enumerate}
\end{frame}


%
%
%
\begin{frame}[fragile]
  \frametitle{Poll Question: The Print Function}
  What type of error will Python produce when this segment of code is run?
  \begin{lstlisting}[language=Python]
  x = "Hello, world"
  y = 5
  foo = x + y + z\end{lstlisting}
  \begin{enumerate}[A]
    \item SyntaxError
    \item NameError
    \item ValueError
    \item TypeError
  \end{enumerate}
\end{frame}

%
%
%
\begin{frame}[fragile]
  \frametitle{Poll Question: Error}
  What happens if I type the following into a Python interpreter?
  \begin{lstlisting}[language=Python]
  a = input()
  b = input()
  a + b = x \end{lstlisting}
  \begin{enumerate}
    \item No error
    \item SyntaxError
    \item NameError
    \item TypeError
  \end{enumerate}
\end{frame}

\section{Data Types and Errors}

%
%
%
\begin{frame}[fragile]
  \frametitle{Poll Question: Data Types}
  What is the type of x after the following segment of code runs if the user inputs a 5 and then a 4?
  \begin{lstlisting}[language=Python]
  a = input()
  b = input()
  x = a + b\end{lstlisting}
  \begin{enumerate}[A]
    \item String
    \item Integer
  \end{enumerate}
\end{frame}

%
%
%
\begin{frame}[fragile]
  \frametitle{Poll Question: Data Types}
  What is the result of running the following code if the user types in \lstinline|world| and then \lstinline|hello|?
  \begin{lstlisting}[language=Python]
  a = int(input())
  b = int(input())
  x = a + b\end{lstlisting}
  \begin{enumerate}[A]
    \item No error
    \item SyntaxError
    \item NameError
    \item ValueError
    \item TypeError
  \end{enumerate}
\end{frame}

%
%
%
\begin{frame}[fragile]
  \frametitle{Converting from string to integer}
  \begin{lstlisting}[language=Python]
  x = input() #type(x) is str
  x = int(x)  #type of x is int
  \end{lstlisting}
  is the same as
  \begin{lstlisting}[language=Python]
  x = int(input()) 
  \end{lstlisting}
\end{frame}

\begin{frame}
  \frametitle{Your Questions!}
\end{frame}

\begin{frame}
  \frametitle{Two options for remaining time}
  \begin{itemize}
    \item Get Python/text editor setup using the instructions on under the lab section on the course website's about page.
    \item Get started on homework 1
  \end{itemize}
\end{frame}

\begin{frame}
  \frametitle{Updates}
  \begin{enumerate}
    \item \textbf{Homework 1:} Posted and due this Friday.
    \item \textbf{Topic 2 (Part 1) and Post-reading}: Posted and due this Friday.
    \item Feel free to drop by office hours if you have any questions.
    \item As a reminder, I encourage you all to work together to complete the homeworks. Don't share exact answers but feel free to discuss things at a high level.
  \end{enumerate}
\end{frame}

\end{document}
